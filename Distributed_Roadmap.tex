\section{Emergent Roadmap} \label{sec:Distributed_Roadmap}

Table \ref{tab:WhatsNextSummary} reprises the ``What's Next'' steps
from all of the previous patterns, offering another view on the
Peeragogy project's \patternname{Roadmap} in a concrete emergent form.
%
This table has been vetted by project participants, who suggested
revisions.  This led to an outline for a new pattern,
`\patternnameext{Onboarding}', a facilitative process that would
complement \patternname{Newcomer} and \patternname{Wrapper}.

We also tagged many of the items in our list of upcoming tasks with the names of
patterns in the pattern catalog.\footnote{\url{https://goo.gl/tcN3q6}}
However, the techniques that we have described in these patterns reach
far beyond our project.  In many cases, the ideas are ancient, or even
primordial (like \patternname{Heartbeat}).  The patterns can be
applied with or without high technology.  The Peeragogy project is one
of
\begin{quote}
 ``[T]ens of thousands of projects in the traditions of world
improvement \'elan -- without any central committee that would have
to, or even could, tell the active what their next operations should
be.'' \cite[p. 402]{sloterdijk2013change}
\end{quote}
When we talk about ``next steps,'' we aim to show what can be
realistically expected from us.
%% ; one can compare, for instance, the
%% pre-Columbian communal working practice of \emph{minga} or
%% \emph{mink'a}.\footnote{\url{https://es.wikipedia.org/wiki/Minka} (in
%%   Spanish).}
And yet, the emergent \patternname{Roadmap} also goes beyond specific
individuals in the project, since they will come and go.

Despite its tendency to longevity, robustness, and
widespread applicability, the outlook for peeragogy can also be
illuminated by examining some of its limitations.  Celebrated mathematician
Michael Atiyah points out that, benefits to collaboration
notwithstanding, ``many mathematicians do not like or are incapable of
collaborating with other mathematicians,'' and he further remarks that
``when it comes to the crunch, there is no substitute for really hard
thinking on your own'' \cite{atiyah1974research}.
%% \patternname{Peeragogy} is only useful in limited forms at times when
%% sharing your concerns with others brings little advantage.  For
%% instance, there may be cultural or even geographical barriers that
%% make the cost of building trust prohibitive.
More broadly, when involving others in your work would have more costs
than benefits \cite{coase1937nature}, it is better to work on your
own.

\begin{wrapfigure}{l}{.52\textwidth}
\vspace{-2.5cm}
\hspace{-.15cm}\resizebox{.55\textwidth}{!}{
\begin{tikzpicture}[dot/.style={circle,inner sep=1pt,fill,name=#1}]
%\draw[step=1cm,gray,very thin] (0,0) grid (10,10);
\node (assess) at (5, 9.75) {{\Large {\sc Assess}}};
\node (organize) at (5, 1) {{\Large {\sc Organize}}};
\node (convene)[text width=2cm,align=center,rotate=270] at (9.4, 5.4) {{\Large {\sc Convene}}};
\node (cooperate)[text width=15cm,align=center,rotate=90] at (.5, 5.2) {{\Large {\sc Cooperate}}};

%% \node(legend)[draw,rectangle,text width=3cm] at (9.25,.75) {\begin{tabular}{p{1.1in}}
%% \textbf{Legend}\\ \hline\vspace{-2mm} \textbf{A}\hspace{.41in}\textbf{B}\\
%% if pattern \textbf{A} refers to pattern \textbf{B}.
%%   \end{tabular}};

%% \draw[-{Latex[width=2mm]},draw=gray] ([xshift=5mm,yshift=1.75mm]legend.west) -- ([xshift=-18mm,yshift=1.75mm]legend.east);
%%%%%%%%%%%%%%%%%%%%%%%%%%%%%%%%%%%%%%%%%%%%%%%%%%%%%%%%%%%%%%%%%%%%%%%%%%%%%%%%%%%%%%%%%%%%%%%%%%%%%
\node[below = 3.3cm of assess] (roadmap) {{\icon \symbol{"0021D4}}  {\icon \symbol{"0021A6}} {\icon \symbol{"0021C2}}};
\node (reduce) at (5, 8.8) {{\icon \symbol{"002159}} {\icon \symbol{"00219B}}};
\node (carryingcapacity) at (1.25, 7.15) {{\icon \symbol{"002194}} {\icon \symbol{"0021D7}}};
\node[below = 3.2cm of carryingcapacity] (heartbeat) {{\icon \symbol{"002185}} {\icon \symbol{"0021A8}}};
\node (aspecificproject) at (8.5, 6.5) {{\icon \symbol{"0021A2}} {\icon \symbol{"00213C}}};
\node[below = 1cm of roadmap] (wrapper) {{\icon \symbol{"002136}} {\icon \symbol{"0021B2}} {\icon \symbol{"0021BD}}};
\node (newcomer) at (8.5, 3.25) {{\icon \symbol{"0021E5}} {\icon \symbol{"002180}}};
\node[below = 1.7cm of wrapper] (scrapbook) {{\icon \symbol{"002168}} {\icon \symbol{"0021B9}} {\icon \symbol{"00214C}}};
\node[above = 1cm of aspecificproject] (peeragogyproject) {{\icon \symbol{"00220A}} {\icon \symbol{"002158}}};
%%%%%%%%%%%%%%%%%%%%%%%%%%%%%%%%%%%%%%%%%%%%%%%%%%%%%%%%%%%%%%%%%%%%%%%%%%%%%%%%%%%%%%%%%%%%%%%%%%%%%
\draw[-{Latex[width=2mm]},draw=gray] (peeragogyproject) -- (aspecificproject);
% \draw[-{Latex[width=2mm]},draw=gray] (aspecificproject) -- (par);
\draw[-{Latex[width=2mm]},draw=gray] (aspecificproject) -- (roadmap);
\draw[-{Latex[width=2mm]},draw=gray] (aspecificproject.230) to[out=250,in=40] (scrapbook);
\draw[-{Latex[width=2mm]},draw=gray] (aspecificproject) -- (carryingcapacity);
\draw[-{Latex[width=2mm]},draw=gray] (carryingcapacity.335) -- (newcomer.160);
\draw[-{Latex[width=2mm]},draw=gray] (carryingcapacity.343) -- ([xshift=2mm]roadmap.160);
\draw[-{Latex[width=2mm]},draw=gray] (carryingcapacity) -- (peeragogyproject);
\draw[-{Latex[width=2mm]},draw=gray] ([xshift=1mm]carryingcapacity.south) -- (scrapbook.140);
\draw[-{Latex[width=2mm]},draw=gray] (heartbeat) -- (aspecificproject.210);
\draw[-{Latex[width=2mm]},draw=gray] (heartbeat) -- (carryingcapacity);
\draw[-{Latex[width=2mm]},draw=gray] (heartbeat) -- (scrapbook.155);
\draw[-{Latex[width=2mm]},draw=gray] (heartbeat) -- (reduce.215);
\draw[-{Latex[width=2mm]},draw=gray] (newcomer) -- ([xshift=4mm]reduce.south);
\draw[-{Latex[width=2mm]},draw=gray] (newcomer) -- (aspecificproject);
% \draw[-{Latex[width=2mm]},draw=gray] (newcomer) -- (creatingaguide.north);
\draw[-{Latex[width=2mm]},draw=gray] (newcomer) -- (roadmap.330);
\draw[-{Latex[width=2mm]},draw=gray] (newcomer) -- (scrapbook.24);
% \draw[-{Latex[width=2mm]},draw=gray] (par) -- (scrapbook);
\draw[-{Latex[width=2mm]},draw=gray] (roadmap) -- (peeragogyproject.205);
\draw[-{Latex[width=2mm]},draw=gray] (roadmap.330) -- (newcomer);
\draw[-{Latex[width=2mm]},draw=gray] (roadmap) -- (wrapper);
\draw[-{Latex[width=2mm]},draw=gray] ([yshift=.3mm]roadmap.187) -- (heartbeat);
\draw[-{Latex[width=2mm]},draw=gray] (roadmap) -- (aspecificproject);
% \draw[-{Latex[width=2mm]},draw=gray] (scrapbook) -- (par);
\draw[-{Latex[width=2mm]},draw=gray] (scrapbook) -- (wrapper);
\draw[-{Latex[width=2mm]},draw=gray] (scrapbook.110) to[out=123,in=250] (reduce.245);
\draw[-{Latex[width=2mm]},draw=gray] (scrapbook.70) to[out=43,in=305] (roadmap.315);
% \draw[-{Latex[width=2mm]},draw=gray] ([xshift=2mm,yshift=-.4mm]reduce.south) -- (creatingaguide);
\draw[-{Latex[width=2mm]},draw=gray] (reduce) -- (carryingcapacity);
\draw[-{Latex[width=2mm]},draw=gray] (reduce) -- (roadmap);
\draw[-{Latex[width=2mm]},draw=gray] ([xshift=.7mm]wrapper.180) -- (heartbeat.10);
\draw[-{Latex[width=2mm]},draw=gray] ([xshift=-.7mm,yshift=-.3mm]wrapper.360) -- (newcomer.178);
\draw[-{Latex[width=2mm]},draw=gray] (wrapper) -- ([xshift=2.3mm]carryingcapacity.south);
\draw[-{Latex[width=2mm]},draw=gray] (wrapper) -- (roadmap);

\end{tikzpicture}

}
\hspace{.4cm}
\vspace{-2.75cm}
\caption{Mnemonic \label{mnemonic}}
\vspace{-.5cm}
\end{wrapfigure}

Indeed, the very idea of a \patternname{Roadmap} in peeragogy is something
of a paradox.  We can't dictate the behavior of other participants,
and we often can't even guess ourselves what's coming up.  A
peeragogical \patternname{Roadmap} should prepare people for the
\emph{absence} of clear step-by-step direction, the \emph{presence} of
different viewpoints and priorities, and the consequent
\emph{requirement} to be reasonably self-directed.  Many features of the
peeragogical approach will be irrelevant to a project that is managed
in a top-down fashion, and that can rely on other coordination
mechanisms (like contracts) to manage work.

%%%%%%%%%%%%%%%%%%%%%%%%%%%%%%%%%%%%%%%%%%%%%%%%%%%%%%%%%%%%%%%%%%%%%%%%%%%%%%%%%%%%%%%%%%%%%%%%%%%%

\begin{table}
{\footnotesize
\newcolumntype{C}{>{\centering\arraybackslash}X}
\begin{tabularx}{\textwidth}{|C|}
\hline
%\rule{\textwidth}{0mm}
\cellcolor{Gray!30} \color{Black} \ref{sec:Peeragogy}. \patternname{Peeragogy}\vspace{.25em}\\
\hline
\multicolumn{1}{|@{\hspace{1mm}}p{.98\textwidth}|}{
\vspace{.01em}
% \textbf{How can we solve problems together?}\\
% Get really concrete about what the problems are.
\PeeragogyWN
\vspace{.28em}}\\
\hline 
%%%%%%%%%%%%%%%%%%%%
\cellcolor{Gray!30} \color{Black} \ref{sec:Roadmap}. \patternname{Roadmap}\vspace{.28em}\\
\hline
\multicolumn{1}{|@{\hspace{1mm}}p{.98\textwidth}|}{
\vspace{.01em}
%\textbf{How can we keep track of what everyone is doing?}\\
%Build a plan that we keep updating as we go along.
\RoadmapWN
\vspace{.28em}}\\
\hline
%%%%%%%%%%%%%%%%%%%%
\cellcolor{Gray!30} \color{Black} \ref{sec:Reduce, reuse, recycle}. \patternname{Reduce, reuse, recycle}\vspace{.28em}\\
\hline
\multicolumn{1}{|@{\hspace{1mm}}p{.98\textwidth}|}{
\vspace{.01em}
%\textbf{How can we avoid undue isolation?}\\
%Use what's there and share what we make.
\ReduceWN
\vspace{.28em}}\\
\hline
%%%%%%%%%%%%%%%%%%%%
\cellcolor{Gray!30} \color{Black} \ref{sec:Carrying capacity}. \patternname{Carrying capacity}\vspace{.28em}\\
\hline
\multicolumn{1}{|@{\hspace{1mm}}p{.98\textwidth}|}{
\vspace{.01em}
%\textbf{How can we avoid becoming overwhelmed?}\\
%Clearly express when we're frustrated.
\CarryingWN
\vspace{.28em}}\\
\hline
%%%%%%%%%%%%%%%%%%%%
\cellcolor{Gray!30} \color{Black} \ref{sec:A specific project}. \patternname{A specific project}\vspace{.28em}\\
\hline
\multicolumn{1}{|@{\hspace{1mm}}p{.98\textwidth}|}{
\vspace{.01em}
%\textbf{How can we avoid becoming perplexed?}\\
%Focus on concrete, doable tasks.
\SpecificWN
\vspace{.28em}}\\
\hline
%%%%%%%%%%%%%%%%%%%%
\cellcolor{Gray!30} \color{Black} \ref{sec:Wrapper}. \patternname{Wrapper}\vspace{.28em}\\
\hline
\multicolumn{1}{|@{\hspace{1mm}}p{.98\textwidth}|}{
\vspace{.01em}
%\textbf{How can people stay in touch with the project?}\\
%Maintain a coherent public surface.
\WrapperWN
\vspace{.28em}}\\
\hline
%%%%%%%%%%%%%%%%%%%%
\cellcolor{Gray!30} \color{Black} \ref{sec:Heartbeat}. \patternname{Heartbeat}\vspace{.28em}\\
\hline
\multicolumn{1}{|@{\hspace{1mm}}p{.98\textwidth}|}{
\vspace{.01em}
%\textbf{How can we make the project ``real'' for participants?}\\
%Keep up a regular, sustaining rhythm.
\HeartbeatWN
\vspace{.28em}}\\
\hline
%%%%%%%%%%%%%%%%%%%%
\cellcolor{Gray!30} \color{Black} \ref{sec:Newcomer}. \patternname{Newcomer}\vspace{.28em}\\
\hline
\multicolumn{1}{|@{\hspace{1mm}}p{.98\textwidth}|}{
\vspace{.01em}
%\textbf{How can we make the project accessible to new people?}\\
%Let's learn from newcomers.
\NewcomerWN
\vspace{.28em}}\\
\hline
%%%%%%%%%%%%%%%%%%%%
\cellcolor{Gray!30} \color{Black} \ref{sec:Scrapbook}. \patternname{Scrapbook}\vspace{.28em}\\
\hline
\multicolumn{1}{|@{\hspace{1mm}}p{.98\textwidth}|}{
\vspace{.01em}
%\textbf{How can we maintain focus as time goes by?}\\
%Move things that are not of immediate use out of focus.
\ScrapbookWN
\vspace{.28em}}\\
\hline
\end{tabularx}
}
\caption{What's next in the Peeragogy project\label{tab:WhatsNextSummary}}
\end{table}

%% \begin{table}
%% {\footnotesize
%% \begin{tabular}{|p{\textwidth}|}
%% \hline
%% \rowcolor{Gray!30} \multicolumn{1}{|l|}{\color{Black} \ref{sec:Peeragogy}. \patternname{Peeragogy}}\\
%% \hline
%% \vspace{-.5em}
%% \PeeragogyWN\\
%% \hline 
%% %%%%%%%%%%%%%%%%%%%%
%% \rowcolor{Gray!30} \multicolumn{1}{|l|}{\color{Black} \ref{sec:Roadmap}. \patternname{Roadmap}}\\
%% \hline
%% \vspace{-.5em}
%% \RoadmapWN\\[.1cm]
%% \hline
%% %%%%%%%%%%%%%%%%%%%%
%% \rowcolor{Gray!30} \multicolumn{1}{|l|}{\color{Black} \ref{sec:Reduce, reuse, recycle}. \patternname{Reduce, reuse, recycle}}\\
%% \hline
%% \vspace{-.5em}
%% \ReduceWN
%% \\[.1cm]
%% \hline
%% %%%%%%%%%%%%%%%%%%%%
%% \rowcolor{Gray!30} \multicolumn{1}{|l|}{\color{Black} \ref{sec:Carrying capacity}. \patternname{Carrying capacity}}\\
%% \hline
%% \vspace{-.5em}
%% \CarryingWN
%% \\[.1cm]
%% \hline
%% %%%%%%%%%%%%%%%%%%%%
%% \rowcolor{Gray!30} \multicolumn{1}{|l|}{\color{Black} \ref{sec:A specific project}. \patternname{A specific project}}\\
%% \hline
%% \vspace{-.5em}
%% \SpecificWN
%% \\[.1cm]
%% \hline
%% %%%%%%%%%%%%%%%%%%%%
%% \rowcolor{Gray!30} \multicolumn{1}{|l|}{\color{Black} \ref{sec:Wrapper}. \patternname{Wrapper}}\\
%% \hline
%% \vspace{-.5em}
%% \WrapperWN
%% \\[.1cm]
%% \hline
%% %%%%%%%%%%%%%%%%%%%%
%% \rowcolor{Gray!30} \multicolumn{1}{|l|}{\color{Black} \ref{sec:Heartbeat}. \patternname{Heartbeat}}\\
%% \hline
%% \vspace{-.5em}
%% \HeartbeatWN
%% \\[.1cm]
%% \hline
%% %%%%%%%%%%%%%%%%%%%%
%% \rowcolor{Gray!30} \multicolumn{1}{|l|}{\color{Black} \ref{sec:Newcomer}. \patternname{Newcomer}}\\
%% \hline
%% \vspace{-.5em}
%% \NewcomerWN
%% \\[.1cm]
%% \hline
%% %%%%%%%%%%%%%%%%%%%%
%% \rowcolor{Gray!30} \multicolumn{1}{|l|}{\color{Black} \ref{sec:Scrapbook}. \patternname{Scrapbook}}\\
%% \hline
%% \vspace{-.5em}
%% \ScrapbookWN
%% \\[.1cm]
%% \hline

%% \end{tabular}
%% }
%% \caption{What's next for the Peeragogy project\label{tab:WhatsNextSummary}}
%% \end{table}

%% \subsubsection*{\hyperref[sec:Peeragogy]{Peeragogy}} 
%% \PeeragogyWN

%% \subsubsection*{\hyperref[sec:Roadmap]{Roadmap}} 
%% \RoadmapWN

%% \subsubsection*{\hyperref[sec:Reduce, reuse, recycle]{Reduce, reuse, recycle}}
%% \ReduceWN

%% \subsubsection*{\hyperref[sec:Carrying capacity]{Carrying capacity}} 
%% \CarryingWN

%% \subsubsection*{\hyperref[sec:A specific project]{A specific project}}
%% \SpecificWN

%% \subsubsection*{\hyperref[sec:Wrapper]{Wrapper}}
%% \WrapperWN

%% \subsubsection*{\hyperref[sec:Heartbeat]{Heartbeat}}
%% \HeartbeatWN

%% \subsubsection*{\hyperref[sec:Newcomer]{Newcomer}}
%% \NewcomerWN

%% \subsubsection*{\hyperref[sec:Scrapbook]{Scrapbook}} 
%% \ScrapbookWN

\newpage
