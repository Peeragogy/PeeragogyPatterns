\section{Emergent Roadmap} \label{sec:Distributed_Roadmap}

Table \ref{tab:WhatsNextSummary} reprises the ``What's Next'' steps
from all of the previous patterns, offering another view on the
Peeragogy project's \patternname{Roadmap} in a concrete emergent form.
%
This table has been vetted by project participants, who suggested
revisions.  This led to an outline for a new pattern,
`\patternnameext{Onboarding}', a facilitative process that would
complement \patternname{Newcomer} and \patternname{Wrapper}.

We also tagged many of the items in our list of upcoming tasks with the names of
patterns in the pattern catalog.\footnote{\url{https://goo.gl/tcN3q6}}
However, the techniques that we have described in these patterns reach
far beyond our project.  In many cases, the ideas are ancient, or even
primordial (like \patternname{Heartbeat}).  The patterns can be
applied with or without high technology.  The Peeragogy project is one
of
\begin{quote}
 ``[T]ens of thousands of projects in the traditions of world
improvement \'elan -- without any central committee that would have
to, or even could, tell the active what their next operations should
be.'' \cite[p. 402]{sloterdijk2013change}
\end{quote}
When we talk about ``next steps,'' we aim to show what can be
realistically expected from us.
%% ; one can compare, for instance, the
%% pre-Columbian communal working practice of \emph{minga} or
%% \emph{mink'a}.\footnote{\url{https://es.wikipedia.org/wiki/Minka} (in
%%   Spanish).}
And yet, the emergent \patternname{Roadmap} also goes beyond specific
individuals in the project, since they will come and go.

Despite its tendency to longevity, robustness, and
widespread applicability, peeragogy does have some limits. For example,
the benefits to collaboration
notwithstanding, 
``there is no substitute for really hard
thinking on your own'' \cite{atiyah1974research}.
%% \patternname{Peeragogy} is only useful in limited forms at times when
%% sharing your concerns with others brings little advantage.  For
%% instance, there may be cultural or even geographical barriers that
%% make the cost of building trust prohibitive.
Nevertheless, even the most independent knowledge worker reads and
publishes: ``There is no such thing as private property in
language''~\cite[p.~559]{jakobson1971selected}.  The question is
usually how best -- not whether -- to involve others
\cite{coase1937nature,coases-penguin}.  This relates to another
prospective pattern: `\patternnameext{Don't quit your day job}'.

\begin{wrapfigure}{l}{.52\textwidth}
\vspace{-2.8cm}
\hspace{-.15cm}\resizebox{.55\textwidth}{!}{
\begin{tikzpicture}[dot/.style={circle,inner sep=1pt,fill,name=#1}]
\draw[thick] (0,.5) rectangle (10,10);
\node (assess) at (5, 9.55) {{\Large {\sc Assess}}};
\node (organize) at (5, 1.2) {{\Large {\sc Organize}}};
\node (convene)[text width=2cm,align=center,rotate=270] at (8.8, 5.4) {{\Large {\sc Convene}}};
\node (cooperate)[text width=15cm,align=center,rotate=90] at (1, 5.2) {{\Large {\sc Cooperate}}};

%% \node(legend)[draw,rectangle,text width=3cm] at (9.25,.75) {\begin{tabular}{p{1.1in}}
%% \textbf{Legend}\\ \hline\vspace{-2mm} \textbf{A}\hspace{.41in}\textbf{B}\\
%% if pattern \textbf{A} refers to pattern \textbf{B}.
%%   \end{tabular}};

%% \draw[draw=none,dashed] ([xshift=5mm,yshift=1.75mm]legend.west) -- ([xshift=-18mm,yshift=1.75mm]legend.east);
%%%%%%%%%%%%%%%%%%%%%%%%%%%%%%%%%%%%%%%%%%%%%%%%%%%%%%%%%%%%%%%%%%%%%%%%%%%%%%%%%%%%%%%%%%%%%%%%%%%%%
\node[below = 3cm of assess] (roadmap) {\hyperref[sec:Roadmap]{\raisebox{.3mm}{{\icon \symbol{"0021D4}}} \hspace{-.2mm}{\icon \symbol{"0021A6}} {\icon \symbol{"0021C2}}}};
\node (reduce) at (5, 8.8) {\hyperref[sec:Reduce, reuse, recycle]{{\icon \symbol{"002159}} {\icon \symbol{"00219B}} {\icon \symbol{"00219E}}}};
\node (carryingcapacity) at (1.75, 7.15) {\hyperref[sec:Carrying capacity]{{\icon \symbol{"002194}} {\icon \symbol{"0021D7}}}};
\node[below = 3.2cm of carryingcapacity] (heartbeat) {\hyperref[sec:Heartbeat]{{\icon \symbol{"002185}} {\icon \symbol{"0021A8}}}};
\node (aspecificproject) at (8, 6.5) {\hyperref[sec:A specific project]{{\icon \symbol{"0021A2}} {\icon \symbol{"00213C}}}};
\node[below = 1.2cm of roadmap] (wrapper) {\hyperref[sec:Wrapper]{{\icon \symbol{"002136}} {\icon \symbol{"0021B2}} {\icon \symbol{"0021BD}}}};
\node (newcomer) at (8, 3.25) {\hyperref[sec:Newcomer]{{\icon \symbol{"0021E5}} {\icon \symbol{"002180}}}};
\node[below = 1.9cm of wrapper] (scrapbook) {\hyperref[sec:Scrapbook]{{\icon \symbol{"002168}} {\icon \symbol{"0021B9}} {\icon \symbol{"00214C}}}};
\node[above = 1cm of aspecificproject] (peeragogyproject) {\hyperref[sec:Peeragogy]{{\icon \symbol{"00220A}} {\icon \symbol{"002158}}}};
%%%%%%%%%%%%%%%%%%%%%%%%%%%%%%%%%%%%%%%%%%%%%%%%%%%%%%%%%%%%%%%%%%%%%%%%%%%%%%%%%%%%%%%%%%%%%%%%%%%%%
\draw[draw=black,dashed,name path=line 1] (peeragogyproject) -- (aspecificproject)node[text opacity=0]                    {1};
\draw[draw=black,dashed,name path=line 2] (aspecificproject) -- (roadmap)node[text opacity=0]                             {2};
\draw[draw=black,dashed,name path=line 3] (aspecificproject.230) to[out=250,in=40] ([xshift=1mm]scrapbook.60)node[text opacity=0]        {3};
\draw[draw=black,dashed,name path=line 4] (aspecificproject) -- (carryingcapacity)node[text opacity=0]                    {4};
\draw[draw=black,dashed,name path=line 5] (carryingcapacity.335) -- (newcomer.160)node[text opacity=0]                    {5};
\draw[draw=black,dashed,name path=line 6] (carryingcapacity.343) -- ([xshift=1mm,yshift=-1mm]roadmap.150)node[text opacity=0]         {6};
\draw[draw=black,dashed,name path=line 7] (carryingcapacity) -- ([xshift=.4mm]peeragogyproject.west)node[text opacity=0]                    {7};
\draw[draw=black,dashed,name path=line 8] ([xshift=1mm]carryingcapacity.south) -- (scrapbook.140)node[text opacity=0]     {8};
\draw[draw=black,dashed,name path=line 9] (heartbeat) -- ([xshift=.4mm]aspecificproject.210)node[text opacity=0]                       {9};
\draw[draw=black,dashed,name path=line 10] (heartbeat) -- (carryingcapacity)node[text opacity=0]                          {10};
\draw[draw=black,dashed,name path=line 11] (heartbeat) -- (scrapbook.155)node[text opacity=0]                             {11};
\draw[draw=black,dashed,name path=line 12] (heartbeat) -- (reduce.225)node[text opacity=0]                                {12};
\draw[draw=black,dashed,name path=line 13] (newcomer) -- ([xshift=4mm]reduce.south)node[text opacity=0]                   {13};
\draw[draw=black,dashed,name path=line 14] (newcomer) -- (aspecificproject)node[text opacity=0]                           {14};
\draw[draw=black,dashed,name path=line 15] (newcomer) -- (roadmap.330)node[text opacity=0]                                {15};
\draw[draw=black,dashed,name path=line 16] (newcomer) -- (scrapbook.24)node[text opacity=0]                               {16};
\draw[draw=black,dashed,name path=line 17] (roadmap) -- ([xshift=1.2mm]peeragogyproject.205)node[text opacity=0]                        {17};
\draw[draw=black,dashed,name path=line 18] ([xshift=1.5mm,yshift=.5mm]roadmap.200) -- (heartbeat)node[text opacity=0]                               {18};
\draw[draw=black,dashed,name path=line 19] (scrapbook) -- (wrapper)node[text opacity=0]                                   {19};
\draw[draw=black,dashed,name path=line 20] (scrapbook.110) to[out=123,in=240] ([yshift=.4mm,xshift=.2mm]reduce.245) node[text opacity=0]            {20};
\draw[draw=black,dashed,name path=line 21] (scrapbook.70) to[out=43,in=310] (roadmap.315) node[text opacity=0]             {21};
\draw[draw=black,dashed,name path=line 22] (reduce) -- (carryingcapacity)node[text opacity=0]                             {22};
\draw[draw=black,dashed,name path=line 23] ([xshift=.7mm]reduce.270) -- ([xshift=.7mm,yshift=-.4mm]roadmap.90)node[text opacity=0]                                      {23};
\draw[draw=black,dashed,name path=line 24] ([xshift=.7mm]wrapper.180) -- (heartbeat.10)node[text opacity=0]               {24};
\draw[draw=black,dashed,name path=line 25] (wrapper.355) -- (newcomer.180)node[text opacity=0] {25};
\draw[draw=black,dashed,name path=line 26] (wrapper) -- ([xshift=2.3mm]carryingcapacity.south)node[text opacity=0]        {26};
\draw[draw=black,dashed,name path=line 27] (wrapper) -- (roadmap)node[text opacity=0]                                     {27};

%% \fill[fill=none,name intersections={of=line 7 and line 23,total=\t}]
%%     \foreach \s in {1,...,\t}{(intersection-\s) circle (2pt) node {$\star$}};

%% \fill[fill=none,name intersections={of=line 12 and line 7,total=\t}]
%%     \foreach \s in {1,...,\t}{(intersection-\s) circle (2pt) node {$\star$}};

%% \fill[fill=none,name intersections={of=line 20 and line 7,total=\t}]
%%     \foreach \s in {1,...,\t}{(intersection-\s) circle (2pt) node {$\star$}};

%% \fill[fill=none,name intersections={of=line 13 and line 7,total=\t}]
%%     \foreach \s in {1,...,\t}{(intersection-\s) circle (2pt) node {$\star$}};

%% \fill[fill=none,name intersections={of=line 12 and line 4,total=\t}]
%%     \foreach \s in {1,...,\t}{(intersection-\s) circle (2pt) node {$\star$}};

%% \fill[fill=none,name intersections={of=line 12 and line 6,total=\t}]
%%     \foreach \s in {1,...,\t}{(intersection-\s) circle (2pt) node {$\star$}};

%% \fill[fill=none,name intersections={of=line 12 and line 26,total=\t}]
%%     \foreach \s in {1,...,\t}{(intersection-\s) circle (2pt) node {$\star$}};

%% \fill[fill=none,name intersections={of=line 12 and line 8,total=\t}]
%%     \foreach \s in {1,...,\t}{(intersection-\s) circle (2pt) node {$\star$}};

%% \fill[fill=none,name intersections={of=line 18 and line 8,total=\t}]
%%     \foreach \s in {1,...,\t}{(intersection-\s) circle (2pt) node {$\star$}};

%% \fill[fill=none,name intersections={of=line 24 and line 8,total=\t}]
%%     \foreach \s in {1,...,\t}{(intersection-\s) circle (2pt) node {$\star$}};

%% \fill[fill=none,name intersections={of=line 9 and line 8,total=\t}]
%%     \foreach \s in {1,...,\t}{(intersection-\s) circle (2pt) node {$\star$}};

%% \fill[fill=none,name intersections={of=line 4 and line 23,total=\t}]
%%     \foreach \s in {1,...,\t}{(intersection-\s) circle (2pt) node {$\star$}};

%% \fill[fill=none,name intersections={of=line 4 and line 17,total=\t}]
%%     \foreach \s in {1,...,\t}{(intersection-\s) circle (2pt) node {$\star$}};

%% \fill[fill=none,name intersections={of=line 4 and line 13,total=\t}]
%%     \foreach \s in {1,...,\t}{(intersection-\s) circle (2pt) node {$\star$}};

%% \fill[fill=none,name intersections={of=line 17 and line 13,total=\t}]
%%     \foreach \s in {1,...,\t}{(intersection-\s) circle (2pt) node {$\star$}};

%% \fill[fill=none,name intersections={of=line 2 and line 13,total=\t}]
%%     \foreach \s in {1,...,\t}{(intersection-\s) circle (2pt) node {$\star$}};

%% \fill[fill=none,name intersections={of=line 9 and line 13,total=\t}]
%%     \foreach \s in {1,...,\t}{(intersection-\s) circle (2pt) node {$\star$}};

%% \fill[fill=none,name intersections={of=line 3 and line 13,total=\t}]
%%     \foreach \s in {1,...,\t}{(intersection-\s) circle (2pt) node {$\star$}};

%% \fill[fill=none,name intersections={of=line 3 and line 15,total=\t}]
%%     \foreach \s in {1,...,\t}{(intersection-\s) circle (2pt) node {$\star$}};

%% \fill[fill=none,name intersections={of=line 3 and line 5,total=\t}]
%%     \foreach \s in {1,...,\t}{(intersection-\s) circle (2pt) node {$\star$}};

%% \fill[fill=none,name intersections={of=line 3 and line 25,total=\t}]
%%     \foreach \s in {1,...,\t}{(intersection-\s) circle (2pt) node {$\star$}};

%% \fill[fill=none,name intersections={of=line 26 and line 18,total=\t}]
%%     \foreach \s in {1,...,\t}{(intersection-\s) circle (2pt) node {$\star$}};

%% \fill[fill=none,name intersections={of=line 15 and line 9,total=\t}]
%%     \foreach \s in {1,...,\t}{(intersection-\s) circle (2pt) node {$\star$}};

%% \fill[fill=none,name intersections={of=line 20 and line 4,total=\t}]
%%     \foreach \s in {1,...,\t}{(intersection-\s) circle (2pt) node {$\star$}};

%% \fill[fill=none,name intersections={of=line 20 and line 6,total=\t}]
%%     \foreach \s in {1,...,\t}{(intersection-\s) circle (2pt) node {$\star$}};

%% \fill[fill=none,name intersections={of=line 20 and line 26,total=\t}]
%%     \foreach \s in {1,...,\t}{(intersection-\s) circle (2pt) node {$\star$}};

%% \fill[fill=none,name intersections={of=line 20 and line 24,total=\t}]
%%     \foreach \s in {1,...,\t}{(intersection-\s) circle (2pt) node {$\star$}};

%% \fill[fill=none,name intersections={of=line 20 and line 9,total=\t}]
%%     \foreach \s in {1,...,\t}{(intersection-\s) circle (2pt) node {$\star$}};

%% \fill[fill=none,name intersections={of=line 20 and line 18,total=\t}]
%%     \foreach \s in {1,...,\t}{(intersection-\s) circle (2pt) node {$\star$}};

%% \fill[fill=none,name intersections={of=line 5 and line 9,total=\t}]
%%     \foreach \s in {1,...,\t}{(intersection-\s) circle (2pt) node {$\star$}};

%% \fill[fill=none,name intersections={of=line 5 and line 27,total=\t}]
%%     \foreach \s in {1,...,\t}{(intersection-\s) circle (2pt) node {$\star$}};

%% \fill[fill=none,name intersections={of=line 9 and line 27,total=\t}]
%%     \foreach \s in {1,...,\t}{(intersection-\s) circle (2pt) node {$\star$}};

%% \fill[fill=none,name intersections={of=line 21 and line 25,total=\t}]
%%     \foreach \s in {1,...,\t}{(intersection-\s) circle (2pt) node {$\star$}};

%% \fill[fill=none,name intersections={of=line 21 and line 5,total=\t}]
%%     \foreach \s in {1,...,\t}{(intersection-\s) circle (2pt) node {$\star$}};

%% \fill[fill=none,name intersections={of=line 21 and line 9,total=\t}]
%%     \foreach \s in {1,...,\t}{(intersection-\s) circle (2pt) node {$\star$}};

%% \fill[fill=none,name intersections={of=line 26 and line 9,total=\t}]
%%     \foreach \s in {1,...,\t}{(intersection-\s) circle (2pt) node {$\star$}};
\end{tikzpicture}

}
\hspace{.4cm}
\vspace{-2.95cm}
\caption{Mnemonic \label{mnemonic}}
\vspace{-.6cm}
\end{wrapfigure}

But there is a further issue that presents more of a paradox.  We
can't dictate the behavior of other participants, and we often can't
guess ourselves what's coming up.  A peeragogical
\patternname{Roadmap} should prepare people for the \emph{absence} of
clear step-by-step direction, the \emph{presence} of different
viewpoints and priorities, and the consequent \emph{requirement} to be
reasonably self-directed.  Many features of the peeragogical approach
would be irrelevant to a project that is managed in a top-down
fashion, and that can rely on other coordination mechanisms (like
contracts) to manage work.  Peeragogy is relevant when we must define
or redefine the problems together.

%%%%%%%%%%%%%%%%%%%%%%%%%%%%%%%%%%%%%%%%%%%%%%%%%%%%%%%%%%%%%%%%%%%%%%%%%%%%%%%%%%%%%%%%%%%%%%%%%%%%

\begin{table}
{\footnotesize
\newcolumntype{C}{>{\centering\arraybackslash}X}
\begin{tabularx}{\textwidth}{|C|}
\hline
%\rule{\textwidth}{0mm}
\vspace{-.4em}\cellcolor{Gray!30} \color{Black} \ref{sec:Peeragogy}. \patternname{Peeragogy}\vspace{.25em}\\
\hline
\multicolumn{1}{|@{\hspace{1mm}}p{.98\textwidth}|}{
\vspace{.01em}
% \textbf{How can we solve problems together?}\\
% Get really concrete about what the problems are.
\PeeragogyWN
\vspace{.28em}}\\
\hline 
%%%%%%%%%%%%%%%%%%%%
\vspace{-.4em}\cellcolor{Gray!30} \color{Black} \ref{sec:Roadmap}. \patternname{Roadmap}\vspace{.28em}\\
\hline
\multicolumn{1}{|@{\hspace{1mm}}p{.98\textwidth}|}{
\vspace{.01em}
%\textbf{How can we keep track of what everyone is doing?}\\
%Build a plan that we keep updating as we go along.
\RoadmapWN
\vspace{.28em}}\\
\hline
%%%%%%%%%%%%%%%%%%%%
\vspace{-.4em}\cellcolor{Gray!30} \color{Black} \ref{sec:Reduce, reuse, recycle}. \patternname{Reduce, reuse, recycle}\vspace{.28em}\\
\hline
\multicolumn{1}{|@{\hspace{1mm}}p{.98\textwidth}|}{
\vspace{.01em}
%\textbf{How can we avoid undue isolation?}\\
%Use what's there and share what we make.
\ReduceWN
\vspace{.28em}}\\
\hline
%%%%%%%%%%%%%%%%%%%%
\vspace{-.4em}\cellcolor{Gray!30} \color{Black} \ref{sec:Carrying capacity}. \patternname{Carrying capacity}\vspace{.28em}\\
\hline
\multicolumn{1}{|@{\hspace{1mm}}p{.98\textwidth}|}{
\vspace{.01em}
%\textbf{How can we avoid becoming overwhelmed?}\\
%Clearly express when we're frustrated.
\CarryingWN
\vspace{.28em}}\\
\hline
%%%%%%%%%%%%%%%%%%%%
\vspace{-.4em}\cellcolor{Gray!30} \color{Black} \ref{sec:A specific project}. \patternname{A specific project}\vspace{.28em}\\
\hline
\multicolumn{1}{|@{\hspace{1mm}}p{.98\textwidth}|}{
\vspace{.01em}
%\textbf{How can we avoid becoming perplexed?}\\
%Focus on concrete, doable tasks.
\SpecificWN
\vspace{.28em}}\\
\hline
%%%%%%%%%%%%%%%%%%%%
\vspace{-.4em}\cellcolor{Gray!30} \color{Black} \ref{sec:Wrapper}. \patternname{Wrapper}\vspace{.28em}\\
\hline
\multicolumn{1}{|@{\hspace{1mm}}p{.98\textwidth}|}{
\vspace{.01em}
%\textbf{How can people stay in touch with the project?}\\
%Maintain a coherent public surface.
\WrapperWN
\vspace{.28em}}\\
\hline
%%%%%%%%%%%%%%%%%%%%
\vspace{-.4em}\cellcolor{Gray!30} \color{Black} \ref{sec:Heartbeat}. \patternname{Heartbeat}\vspace{.28em}\\
\hline
\multicolumn{1}{|@{\hspace{1mm}}p{.98\textwidth}|}{
\vspace{.01em}
%\textbf{How can we make the project ``real'' for participants?}\\
%Keep up a regular, sustaining rhythm.
\HeartbeatWN
\vspace{.28em}}\\
\hline
%%%%%%%%%%%%%%%%%%%%
\vspace{-.4em}\cellcolor{Gray!30} \color{Black} \ref{sec:Newcomer}. \patternname{Newcomer}\vspace{.28em}\\
\hline
\multicolumn{1}{|@{\hspace{1mm}}p{.98\textwidth}|}{
\vspace{.01em}
%\textbf{How can we make the project accessible to new people?}\\
%Let's learn from newcomers.
\NewcomerWN
\vspace{.28em}}\\
\hline
%%%%%%%%%%%%%%%%%%%%
\vspace{-.4em}\cellcolor{Gray!30} \color{Black} \ref{sec:Scrapbook}. \patternname{Scrapbook}\vspace{.28em}\\
\hline
\multicolumn{1}{|@{\hspace{1mm}}p{.98\textwidth}|}{
\vspace{.01em}
%\textbf{How can we maintain focus as time goes by?}\\
%Move things that are not of immediate use out of focus.
\ScrapbookWN
\vspace{.28em}}\\
\hline
\end{tabularx}
}
\smallskip
\caption{What's next in the Peeragogy project\label{tab:WhatsNextSummary}}
\end{table}



\newpage
