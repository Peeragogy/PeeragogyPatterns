\section{Emergent Roadmap} \label{sec:Distributed_Roadmap}

This section reprises the ``What's Next'' steps from all of the previous
patterns, offering another view on the Peeragogy project's
\patternname{Roadmap} in a concrete emergent form.

\subsubsection*{\hyperref[sec:Peeragogy]{Peeragogy}} 
 We intend to revise and extend the patterns and methods of peeragogy to make it a workable model for learning, inside or outside of institutions.

\subsubsection*{\hyperref[sec:Roadmap]{Roadmap}} 
If we sense that something needs to change about the project, that is a clue that we might need to record a new pattern, or revise our existing patterns.

\subsubsection*{\hyperref[sec:Reduce, reuse, recycle]{Reduce, reuse, recycle}}
We've converted our old pattern catalog from the \emph{Peeragogy Handbook} into this paper, sharing it with a new community and gaining new perspectives.  Can we repeat that for other things we've made?

\subsubsection*{\hyperref[sec:Carrying capacity]{Carrying capacity}} 
Making it easy and fruitful for others to get involved is one of the best ways to redistribute the load.  This often requires skill development among those involved; compare the \patternname{Newcomer} pattern.

\subsubsection*{\hyperref[sec:A specific project]{A specific project}}
We need to build specific, tangible ``what's next'' steps and connect them with concrete action. Use the \patternname{Scrapbook} to organize that process. 

\subsubsection*{\hyperref[sec:Wrapper]{Wrapper}}
We have prototyped and deployed a visual ``dashboard'' that people can use to get involved with the ongoing work in the project.  Let's improve it, and match it with an improved interaction design for peeragogy.org.

\subsubsection*{\hyperref[sec:Heartbeat]{Heartbeat}} Identifying and fostering new \patternnameplural{Heartbeat} and new working groups is a task that can help make the community more robust.  This is the time dimension of spin off projects described in \patternname{Reduce, reuse, recycle}.

\subsubsection*{\hyperref[sec:Newcomer]{Newcomer}} A more detailed (but non-limiting) ``How to Get Involved'' walk-through or ``DIY Toolkit'' would be good to develop. We can start by listing some of the things we're currently learning about.

\subsubsection*{\hyperref[sec:Scrapbook]{Scrapbook}} 
After pruning back our pattern catalog, we want it to grow again: new patterns are needed.
One strategy would be to ``patternize'' the rest of the \emph{Peeragogy Handbook.}

