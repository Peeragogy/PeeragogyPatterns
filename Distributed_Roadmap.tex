\section{Emergent Roadmap} \label{sec:Distributed_Roadmap}

This section reprises the ``What's Next'' steps in all the previous
patterns, offering another view on the Peeragogy Project's
\patternname{Roadmap} in a concrete emergent form.

\subsubsection*{\hyperref[sec:Peeragogy_Project]{Peeragogy}} 
 We intend to revise and extend the patterns and methods of peeragogy to make it a workable model for education.

\subsubsection*{\hyperref[sec:Roadmap]{Roadmap}} If we sense that something needs to change about the project, that is a clue that we might need to record a new pattern.

\subsubsection*{\hyperref[sec:Use_or_make]{Reduce, Reuse, Recycle}}
We've spun off the pattern catalog from the \emph{Peeragogy Handbook} into this paper, sharing it with a new community and gaining new perspectives.  Let's look for other parts of the handbook we can spin off!

\subsubsection*{\hyperref[sec:Carrying_capacity]{Carrying capacity}} 
Making it easy and fruitful for others to get involved is one of the best ways to
redistribute the load (compare the
\patternname{\href{http://peeragogy.org/practice/heuristics/newcomer/}{Newcomer}}
pattern).

\subsubsection*{\hyperref[sec:A_specific_project]{A specific project}}
We need to build specific, tangible ``what's next'' steps and connect them with concrete action. Use the \patternname{Scrapbook} to organize that process. 

\subsubsection*{\hyperref[sec:Wrapper]{Wrapper}}
We have prototyped a visual ``dashboard'' that people can access to immediately get an idea of what work is ongoing in the project with links to ways to get further engaged.  Let's deploy it.

\subsubsection*{\hyperref[sec:Heartbeat]{Heartbeat}} Identifying and fostering new \patternname{Heartbeats} and new working groups is a task that can help make the community more robust.  This is the temporal dimension of spin off projects described in \patternname{Use or Make}.

\subsubsection*{\hyperref[sec:Newcomer]{Newcomer}} A more detailed (but non-limiting) ``How to Get Involved'' walk-through in text or video form would be good to develop. We can start by listing some of the things we're currently learning about, including: business issues relevant to the Peeragogy project, how to run a MOOC, and hot-syncing our website from Git.

\subsubsection*{\hyperref[sec:Scrapbook]{Scrapbook}} 
After significantly pruning back the pattern catalog, we want it to grow again: new patterns are needed.  Reviewing the contents of the \patternname{Scrapbook} will be one place to look for inspiration, but there are many others.




