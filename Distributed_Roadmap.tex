



Table \ref{tab:WhatsNextSummary} reprises the ``What's Next'' steps
from all of the previous patterns, offering another view on the
Peeragogy project's \patternname{Roadmap} in a concrete emergent form.
%
This table has been vetted by project participants, who suggested
revisions.  This led to an outline for a new pattern,
`\patternnameext{Onboarding}', a facilitative process that would
complement \patternname{Newcomer} and \patternname{Wrapper}.

We also tagged many of the items in our list of upcoming tasks with the names of
patterns in the pattern catalog.\footnote{\url{https://goo.gl/tcN3q6}}
However, the techniques that we have described in these patterns reach
far beyond our project.  In many cases, the ideas are ancient, or even
primordial (like \patternname{Heartbeat}).  The patterns can be
applied with or without high technology.  The Peeragogy project is one
of
\begin{quote}
 ``[T]ens of thousands of projects in the traditions of world
improvement \'elan -- without any central committee that would have
to, or even could, tell the active what their next operations should
be.'' \cite[p. 402]{sloterdijk2013change}
\end{quote}
When we talk about ``next steps,'' we aim to show what can be
realistically expected from us.
%% ; one can compare, for instance, the
%% pre-Columbian communal working practice of \emph{minga} or
%% \emph{mink'a}.\footnote{\url{https://es.wikipedia.org/wiki/Minka} (in
%%   Spanish).}
And yet, the emergent \patternname{Roadmap} also goes beyond specific
individuals in the project, who will come and go.


%%%%%%%%%%%%%%%%%%%%%%%%%%%%%%%%%%%%%%%%%%%%%%%%%%%%%%%%%%%%%%%%%%%%%%%%%%%%%%%%%%%%%%%%%%%%%%%%%%%%

\begin{table}
{\footnotesize
\newcolumntype{C}{>{\centering\arraybackslash}X}
\begin{tabularx}{\textwidth}{|C|}
\hline
%\rule{\textwidth}{0mm}
\vspace{-.4em}\cellcolor{Gray!30} \color{Black} \ref{sec:Peeragogy}. \patternname{Peeragogy}\vspace{.25em}\\
\hline
\multicolumn{1}{|@{\hspace{1mm}}p{.98\textwidth}|}{
\vspace{.01em}
% \textbf{How can we solve problems together?}\\
% Get really concrete about what the problems are.
\PeeragogyWN
\vspace{.28em}}\\
\hline 
%%%%%%%%%%%%%%%%%%%%
\vspace{-.4em}\cellcolor{Gray!30} \color{Black} \ref{sec:Roadmap}. \patternname{Roadmap}\vspace{.28em}\\
\hline
\multicolumn{1}{|@{\hspace{1mm}}p{.98\textwidth}|}{
\vspace{.01em}
%\textbf{How can we keep track of what everyone is doing?}\\
%Build a plan that we keep updating as we go along.
\RoadmapWN
\vspace{.28em}}\\
\hline
%%%%%%%%%%%%%%%%%%%%
\vspace{-.4em}\cellcolor{Gray!30} \color{Black} \ref{sec:Reduce, reuse, recycle}. \patternname{Reduce, reuse, recycle}\vspace{.28em}\\
\hline
\multicolumn{1}{|@{\hspace{1mm}}p{.98\textwidth}|}{
\vspace{.01em}
%\textbf{How can we avoid undue isolation?}\\
%Use what's there and share what we make.
\ReduceWN
\vspace{.28em}}\\
\hline
%%%%%%%%%%%%%%%%%%%%
\vspace{-.4em}\cellcolor{Gray!30} \color{Black} \ref{sec:Carrying capacity}. \patternname{Carrying capacity}\vspace{.28em}\\
\hline
\multicolumn{1}{|@{\hspace{1mm}}p{.98\textwidth}|}{
\vspace{.01em}
%\textbf{How can we avoid becoming overwhelmed?}\\
%Clearly express when we're frustrated.
\CarryingWN
\vspace{.28em}}\\
\hline
%%%%%%%%%%%%%%%%%%%%
\vspace{-.4em}\cellcolor{Gray!30} \color{Black} \ref{sec:A specific project}. \patternname{A specific project}\vspace{.28em}\\
\hline
\multicolumn{1}{|@{\hspace{1mm}}p{.98\textwidth}|}{
\vspace{.01em}
%\textbf{How can we avoid becoming perplexed?}\\
%Focus on concrete, doable tasks.
\SpecificWN
\vspace{.28em}}\\
\hline
%%%%%%%%%%%%%%%%%%%%
\vspace{-.4em}\cellcolor{Gray!30} \color{Black} \ref{sec:Wrapper}. \patternname{Wrapper}\vspace{.28em}\\
\hline
\multicolumn{1}{|@{\hspace{1mm}}p{.98\textwidth}|}{
\vspace{.01em}
%\textbf{How can people stay in touch with the project?}\\
%Maintain a coherent public surface.
\WrapperWN
\vspace{.28em}}\\
\hline
%%%%%%%%%%%%%%%%%%%%
\vspace{-.4em}\cellcolor{Gray!30} \color{Black} \ref{sec:Heartbeat}. \patternname{Heartbeat}\vspace{.28em}\\
\hline
\multicolumn{1}{|@{\hspace{1mm}}p{.98\textwidth}|}{
\vspace{.01em}
%\textbf{How can we make the project ``real'' for participants?}\\
%Keep up a regular, sustaining rhythm.
\HeartbeatWN
\vspace{.28em}}\\
\hline
%%%%%%%%%%%%%%%%%%%%
\vspace{-.4em}\cellcolor{Gray!30} \color{Black} \ref{sec:Newcomer}. \patternname{Newcomer}\vspace{.28em}\\
\hline
\multicolumn{1}{|@{\hspace{1mm}}p{.98\textwidth}|}{
\vspace{.01em}
%\textbf{How can we make the project accessible to new people?}\\
%Let's learn from newcomers.
\NewcomerWN
\vspace{.28em}}\\
\hline
%%%%%%%%%%%%%%%%%%%%
\vspace{-.4em}\cellcolor{Gray!30} \color{Black} \ref{sec:Scrapbook}. \patternname{Scrapbook}\vspace{.28em}\\
\hline
\multicolumn{1}{|@{\hspace{1mm}}p{.98\textwidth}|}{
\vspace{.01em}
%\textbf{How can we maintain focus as time goes by?}\\
%Move things that are not of immediate use out of focus.
\ScrapbookWN
\vspace{.28em}}\\
\hline
\end{tabularx}
}
\smallskip
\caption{What's next in the Peeragogy project\label{tab:WhatsNextSummary}}
\end{table}

Despite its tendency to longevity, robustness, and
widespread applicability, peeragogy does have some limits. For example,
the benefits to collaboration
notwithstanding, 
``there is no substitute for really hard
thinking on your own'' \cite{atiyah1974research}.
%% \patternname{Peeragogy} is only useful in limited forms at times when
%% sharing your concerns with others brings little advantage.  For
%% instance, there may be cultural or even geographical barriers that
%% make the cost of building trust prohibitive.
Nevertheless, even the most independent knowledge worker reads and
publishes: ``There is no such thing as private property in
language''~\cite[p.~559]{jakobson1971selected}.  The question is
usually how best -- not whether -- to involve others
\cite{coase1937nature,coases-penguin}.  This relates to another
prospective pattern: `\patternnameext{Don't quit your day job}'.

But there is a further issue that presents more of a paradox.  We
can't dictate the behavior of other participants, and we often can't
guess ourselves what's coming up.  A peeragogical
\patternname{Roadmap} should prepare people for the \emph{absence} of
clear step-by-step direction, the \emph{presence} of different
viewpoints and priorities, and the consequent \emph{requirement} to be
reasonably self-directed.  Many features of the peeragogical approach
would be irrelevant to a project that is managed in a top-down
fashion, and that can rely on other coordination mechanisms (like
contracts) to manage work.  Peeragogy is relevant when we must define
or redefine the problems together.

%\newpage
