\section{Distributed Roadmap} \label{sec:Distributed_Roadmap}

\begin{quote}
This section summarizes the ``What's Next'' steps in all the previous
patterns, reprising the \patternname{Roadmap} in a distributed, emergent form.
\end{quote}

\subsubsection*{Roadmap} Adding ``What's Next'' steps to our patterns gives us a ``distributed roadmap'' for the \patternname{Peeragogy Project}.  And this works both ways:  
If we sense that something needs to change about the project, that's a
clue that we might need to record a new pattern.

\subsubsection*{Use or make} 
We've spun off the pattern catalog from the \emph{Peeragogy Handbook} into this paper, sharing it with a new community and gaining new perspectives.  Let's look for other parts of the handbook we can spin off!

\subsubsection*{Carrying capacity} This pattern catalog has been rewritten in a way that should make it
easy for anyone to add new patterns. Making it easy and fruitful for
others to get involved is one of the best ways to redistribute the load
(compare the Newcomer pattern).

\subsubsection*{A specific project} 
 Each project connected with the \patternname{Peeragogy Project} should be described with one or more patterns, each with specific, tangible ``what's next'' steps.  The \patternname{Pattern Audit Routine} can help make these ``what's next'' steps concrete.

\subsubsection*{Wrapper}  We need better practices for wrapping things up at
various levels.  One of the latest ideas is to develop a simple visual
``dashboard'' for the project.

\subsubsection*{Heartbeat} When the project is bigger than more than just a few people, it's likely to have several \patternname{Heartbeats}.  Identifying and fostering new \patternname{Heartbeats} and new working groups is a task that can help make the community more robust.  This is the temporal dimension of spin off projects described in \patternname{Use or Make}.

\subsubsection*{Creating A Guide} We’ve been talking with collaborators in the Commons Abundance Network
about how to make a Pattern Language for the Commons. One of the
challenges that arises is how to support ongoing development of the
Pattern Language itself: a “living” map for a living territory. We’re
refining the Peeragogy Pattern language and template as a seed for this.

\subsubsection*{Newcomer} A more detailed (but non-limiting) ``How to Get Involved'' walk-through in text or video form would be good to develop. We can start by listing some of the things we're learning about.

\subsubsection*{Pattern audit routine}  TBA

\subsubsection*{Scrapbook} After significantly pruning back the pattern catalog, we want it to grow again: new patterns are needed.  Reviewing the contents of the \patternname{Scrapbook} will be one place to look for inspiration, but there are others.




