\section{Emergent Roadmap} \label{sec:Distributed_Roadmap}

Table \ref{tab:WhatsNextSummary} reprises the ``What's Next'' steps from all of the previous
patterns, offering another view on the Peeragogy project's
\patternname{Roadmap} in a concrete emergent form.

Architectual maverick Christopher Alexander asked the following questions to an audience of computer programmers \cite{alexander1999origins}: 
\begin{quote}
``What is the Chartres of programming? What task is at a high enough level to inspire people writing programs, to reach for the stars?''
\end{quote}
In order for humanity to pull itself up by its bootstraps, on this planet or any other, we need to continue to learn and adapt.

People who learn actively together talk to each other about material problems, share practical solutions, and constructively critique works-in-progress.  There are many different ways to go about this -- bug reports, mailing lists, writers workshops, Q\&A forums, watercoolers and skateparks are all places where peeragogy can happen.  In the Peeragogy project have found that the necessary ``reflection'' aspects of the process are particularly well-matched to Christopher Alexander's idea of a \emph{pattern language}, in which commonly occurring,  interconnected, elements of an optative design are refined until they can be described in terms of a simple template.  Indeed, thought of as a design pattern, 

Peeragogy typically takes place in mostly-horizontal relationships between people who have different but compatible objectives.  The techniques we describe with these patterns are in many cases ancient; one can compare, for instance, the Quechua communal working practice of \emph{mink'a}.  The underlying practices can be pursued with or without high technology and with or without the technique of design patterns.

The Peeragogy project is one of ``tens of thousands of projects in the traditions of world improvement \'elan -- without any central committee that would have to, or even could, tell the active what their next operations should be'' \cite[p. 402]{sloterdijk2013change}.  When we talk about ``next steps,'' we aim to clarify our own commitments, and show what can be realistically expected from us.  

Unless the project's plan is easy for people to see and to update,
they are not likely to use it, and are less likely to get involved.
The key point of the roadmap is to help support involvement by those
who \emph{are} involved.  The level of detail in the roadmap (and the
existence of a roadmap at all) should correspond to the felt need for
sharing information and to the tolerance of uncertainty among
participants.

\begin{table}
{\footnotesize
\begin{tabular}{|p{\textwidth}|}
\hline
\rowcolor{Gray!30} \multicolumn{1}{|l|}{\color{Black} \ref{sec:Peeragogy}. \patternname{Peeragogy}}\\
\hline
\vspace{-.5em}
\PeeragogyWN\\
\hline 
%%%%%%%%%%%%%%%%%%%%
\rowcolor{Gray!30} \multicolumn{1}{|l|}{\color{Black} \ref{sec:Roadmap}. \patternname{Roadmap}}\\
\hline
\vspace{-.5em}
\RoadmapWN
\\[.1cm]
\hline
%%%%%%%%%%%%%%%%%%%%
\rowcolor{Gray!30} \multicolumn{1}{|l|}{\color{Black} \ref{sec:Reduce, reuse, recycle}. \patternname{Reduce, reuse, recycle}}\\
\hline
\vspace{-.5em}
\ReduceWN
\\[.1cm]
\hline
%%%%%%%%%%%%%%%%%%%%
\rowcolor{Gray!30} \multicolumn{1}{|l|}{\color{Black} \ref{sec:Carrying capacity}. \patternname{Carrying capacity}}\\
\hline
\vspace{-.5em}
\CarryingWN
\\[.1cm]
\hline
%%%%%%%%%%%%%%%%%%%%
\rowcolor{Gray!30} \multicolumn{1}{|l|}{\color{Black} \ref{sec:A specific project}. \patternname{A specific project}}\\
\hline
\vspace{-.5em}
\SpecificWN
\\[.1cm]
\hline
%%%%%%%%%%%%%%%%%%%%
\rowcolor{Gray!30} \multicolumn{1}{|l|}{\color{Black} \ref{sec:Wrapper}. \patternname{Wrapper}}\\
\hline
\vspace{-.5em}
\WrapperWN
\\[.1cm]
\hline
%%%%%%%%%%%%%%%%%%%%
\rowcolor{Gray!30} \multicolumn{1}{|l|}{\color{Black} \ref{sec:Heartbeat}. \patternname{Heartbeat}}\\
\hline
\vspace{-.5em}
\HeartbeatWN
\\[.1cm]
\hline
%%%%%%%%%%%%%%%%%%%%
\rowcolor{Gray!30} \multicolumn{1}{|l|}{\color{Black} \ref{sec:Newcomer}. \patternname{Newcomer}}\\
\hline
\vspace{-.5em}
\NewcomerWN
\\[.1cm]
\hline
%%%%%%%%%%%%%%%%%%%%
\rowcolor{Gray!30} \multicolumn{1}{|l|}{\color{Black} \ref{sec:Scrapbook}. \patternname{Scrapbook}}\\
\hline
\vspace{-.5em}
\ScrapbookWN
\\[.1cm]
\hline

\end{tabular}
}
\caption{What's next for the Peeragogy project\label{tab:WhatsNextSummary}}
\end{table}

%% \subsubsection*{\hyperref[sec:Peeragogy]{Peeragogy}} 
%% \PeeragogyWN

%% \subsubsection*{\hyperref[sec:Roadmap]{Roadmap}} 
%% \RoadmapWN

%% \subsubsection*{\hyperref[sec:Reduce, reuse, recycle]{Reduce, reuse, recycle}}
%% \ReduceWN

%% \subsubsection*{\hyperref[sec:Carrying capacity]{Carrying capacity}} 
%% \CarryingWN

%% \subsubsection*{\hyperref[sec:A specific project]{A specific project}}
%% \SpecificWN

%% \subsubsection*{\hyperref[sec:Wrapper]{Wrapper}}
%% \WrapperWN

%% \subsubsection*{\hyperref[sec:Heartbeat]{Heartbeat}}
%% \HeartbeatWN

%% \subsubsection*{\hyperref[sec:Newcomer]{Newcomer}}
%% \NewcomerWN

%% \subsubsection*{\hyperref[sec:Scrapbook]{Scrapbook}} 
%% \ScrapbookWN

