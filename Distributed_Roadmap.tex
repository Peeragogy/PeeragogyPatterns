\section{Emergent Roadmap} \label{sec:Distributed_Roadmap}

Table \ref{tab:WhatsNextSummary} reprises the ``What's Next'' steps
from all of the previous patterns, offering another view on the
Peeragogy project's \patternname{Roadmap} in a concrete emergent form.
%
This table has been vetted by project participants, who suggested
revisions.  At least one new pattern was outlined in that discussion;
`\patternnameext{Onboarding}', as a process that would complement the
\patternnameext{Newcomer} pattern.

In our working list of tasks, we've tagged many of the items using the
names of patterns in the pattern
catalog.\footnote{\url{https://goo.gl/tcN3q6}} The techniques we
describe with these patterns reach far beyond our project.  In many
cases, they are ancient.  The notions underlying the patterns can be
pursued with or without high technology -- and with or without design
patterns.  The Peeragogy project is one of ``tens of thousands of
projects in the traditions of world improvement \'elan -- without any
central committee that would have to, or even could, tell the active
what their next operations should be''
\cite[p. 402]{sloterdijk2013change}.  When we talk about ``next
steps,'' we aim to clarify our own commitments, and show what can be
realistically expected from us.
%% ; one can compare, for instance, the
%% pre-Columbian communal working practice of \emph{minga} or
%% \emph{mink'a}.\footnote{\url{https://es.wikipedia.org/wiki/Minka} (in
%%   Spanish).}
And yet, the emergent \patternname{Roadmap} also goes beyond specific
individuals in the project; they will come and go.

Despite this tendency to longevity, distribution, and broad
applicability, the outlook for a peeragogical approach can also be
understood through some of its limits.

Celebrated mathematician Michael Atiyah points out that, benefits to
collaboration notwithstanding, ``many mathematicians do not like or
are incapable of collaborating with other mathematicians,'' and he
further remarks that ``when it comes to the crunch, there is no
substitute for really hard thinking on your own''
\cite{atiyah1974research}.  Don't use \patternname{Peeragogy}, or use
it in carefully limited forms, or when sharing your concerns with
others brings no advantage; for instance, when the cost of building
trust is prohibitive.  More broadly, when involving others in your
work would have more costs than benefits \cite{coase1937nature}, it is
better to work on your own.

Indeed, the very idea of a \patternname{Roadmap} in a peeragogy is
something of a paradox.  We can't dictate the behavior of other
participants, and we often can't even guess ourselves what's coming
up.  A peeragogical \patternname{Roadmap} should prepare people for
the \emph{absence} of clear step-by-step direction, the
\emph{presence} of different view points and priorities, and the
consequent \emph{requirement} to be relatively self-directed.  Many
features the peeragogical approach will be irrelevant to a project
that needs to be managed in a top-down fashion, and that can rely on
other coordination mechanisms (like contracts) to manage work.

Architectual maverick Christopher Alexander asked the following
questions to an audience of computer programmers:
\cite{alexander1999origins}:
\begin{quote}
``What is the Chartres of programming? What task is at a high enough level to inspire people writing programs, to reach for the stars?''
\end{quote}


%%%%%%%%%%%%%%%%%%%%%%%%%%%%%%%%%%%%%%%%%%%%%%%%%%%%%%%%%%%%%%%%%%%%%%%%%%%%%%%%%%%%%%%%%%%%%%%%%%%%

\begin{table}
{\footnotesize
\newcolumntype{C}{>{\centering\arraybackslash}X}
\begin{tabularx}{\textwidth}{|C|}
\hline
%\rule{\textwidth}{0mm}
\cellcolor{Gray!30} \color{Black} \ref{sec:Peeragogy}. \patternname{Peeragogy}\vspace{.25em}\\
\hline
\multicolumn{1}{|p{.99\textwidth}|}{
\vspace{.01em}
% \textbf{How can we solve problems together?}\\
% Get really concrete about what the problems are.
\PeeragogyWN
\vspace{.4em}}\\
\hline 
%%%%%%%%%%%%%%%%%%%%
\cellcolor{Gray!30} \color{Black} \ref{sec:Roadmap}. \patternname{Roadmap}\vspace{.4em}\\
\hline
\multicolumn{1}{|p{.99\textwidth}|}{
\vspace{.01em}
%\textbf{How can we keep track of what everyone is doing?}\\
%Build a plan that we keep updating as we go along.
\RoadmapWN
\vspace{.4em}}\\
\hline
%%%%%%%%%%%%%%%%%%%%
\cellcolor{Gray!30} \color{Black} \ref{sec:Reduce, reuse, recycle}. \patternname{Reduce, reuse, recycle}\vspace{.4em}\\
\hline
\multicolumn{1}{|p{.99\textwidth}|}{
\vspace{.01em}
%\textbf{How can we avoid undue isolation?}\\
%Use what's there and share what we make.
\ReduceWN
\vspace{.4em}}\\
\hline
%%%%%%%%%%%%%%%%%%%%
\cellcolor{Gray!30} \color{Black} \ref{sec:Carrying capacity}. \patternname{Carrying capacity}\vspace{.4em}\\
\hline
\multicolumn{1}{|p{.99\textwidth}|}{
\vspace{.01em}
%\textbf{How can we avoid becoming overwhelmed?}\\
%Clearly express when we're frustrated.
\CarryingWN
\vspace{.4em}}\\
\hline
%%%%%%%%%%%%%%%%%%%%
\cellcolor{Gray!30} \color{Black} \ref{sec:A specific project}. \patternname{A specific project}\vspace{.4em}\\
\hline
\multicolumn{1}{|p{.99\textwidth}|}{
\vspace{.01em}
%\textbf{How can we avoid becoming perplexed?}\\
%Focus on concrete, doable tasks.
\SpecificWN
\vspace{.4em}}\\
\hline
%%%%%%%%%%%%%%%%%%%%
\cellcolor{Gray!30} \color{Black} \ref{sec:Wrapper}. \patternname{Wrapper}\vspace{.4em}\\
\hline
\multicolumn{1}{|p{.99\textwidth}|}{
\vspace{.01em}
%\textbf{How can people stay in touch with the project?}\\
%Maintain a coherent public surface.
\WrapperWN
\vspace{.4em}}\\
\hline
%%%%%%%%%%%%%%%%%%%%
\cellcolor{Gray!30} \color{Black} \ref{sec:Heartbeat}. \patternname{Heartbeat}\vspace{.4em}\\
\hline
\multicolumn{1}{|p{.99\textwidth}|}{
\vspace{.01em}
%\textbf{How can we make the project ``real'' for participants?}\\
%Keep up a regular, sustaining rhythm.
\HeartbeatWN
\vspace{.4em}}\\
\hline
%%%%%%%%%%%%%%%%%%%%
\cellcolor{Gray!30} \color{Black} \ref{sec:Newcomer}. \patternname{Newcomer}\vspace{.4em}\\
\hline
\multicolumn{1}{|p{.99\textwidth}|}{
\vspace{.01em}
%\textbf{How can we make the project accessible to new people?}\\
%Let's learn from newcomers.
\NewcomerWN
\vspace{.4em}}\\
\hline
%%%%%%%%%%%%%%%%%%%%
\cellcolor{Gray!30} \color{Black} \ref{sec:Scrapbook}. \patternname{Scrapbook}\vspace{.4em}\\
\hline
\vspace{.01em}
%\textbf{How can we maintain focus as time goes by?}\\
%Move things that are not of immediate use out of focus.
\ScrapbookWN
\vspace{.4em}\\
\hline
\end{tabularx}
}
\caption{What's next for the Peeragogy project\label{tab:WhatsNextSummary}}
\end{table}

%% \begin{table}
%% {\footnotesize
%% \begin{tabular}{|p{\textwidth}|}
%% \hline
%% \rowcolor{Gray!30} \multicolumn{1}{|l|}{\color{Black} \ref{sec:Peeragogy}. \patternname{Peeragogy}}\\
%% \hline
%% \vspace{-.5em}
%% \PeeragogyWN\\
%% \hline 
%% %%%%%%%%%%%%%%%%%%%%
%% \rowcolor{Gray!30} \multicolumn{1}{|l|}{\color{Black} \ref{sec:Roadmap}. \patternname{Roadmap}}\\
%% \hline
%% \vspace{-.5em}
%% \RoadmapWN\\[.1cm]
%% \hline
%% %%%%%%%%%%%%%%%%%%%%
%% \rowcolor{Gray!30} \multicolumn{1}{|l|}{\color{Black} \ref{sec:Reduce, reuse, recycle}. \patternname{Reduce, reuse, recycle}}\\
%% \hline
%% \vspace{-.5em}
%% \ReduceWN
%% \\[.1cm]
%% \hline
%% %%%%%%%%%%%%%%%%%%%%
%% \rowcolor{Gray!30} \multicolumn{1}{|l|}{\color{Black} \ref{sec:Carrying capacity}. \patternname{Carrying capacity}}\\
%% \hline
%% \vspace{-.5em}
%% \CarryingWN
%% \\[.1cm]
%% \hline
%% %%%%%%%%%%%%%%%%%%%%
%% \rowcolor{Gray!30} \multicolumn{1}{|l|}{\color{Black} \ref{sec:A specific project}. \patternname{A specific project}}\\
%% \hline
%% \vspace{-.5em}
%% \SpecificWN
%% \\[.1cm]
%% \hline
%% %%%%%%%%%%%%%%%%%%%%
%% \rowcolor{Gray!30} \multicolumn{1}{|l|}{\color{Black} \ref{sec:Wrapper}. \patternname{Wrapper}}\\
%% \hline
%% \vspace{-.5em}
%% \WrapperWN
%% \\[.1cm]
%% \hline
%% %%%%%%%%%%%%%%%%%%%%
%% \rowcolor{Gray!30} \multicolumn{1}{|l|}{\color{Black} \ref{sec:Heartbeat}. \patternname{Heartbeat}}\\
%% \hline
%% \vspace{-.5em}
%% \HeartbeatWN
%% \\[.1cm]
%% \hline
%% %%%%%%%%%%%%%%%%%%%%
%% \rowcolor{Gray!30} \multicolumn{1}{|l|}{\color{Black} \ref{sec:Newcomer}. \patternname{Newcomer}}\\
%% \hline
%% \vspace{-.5em}
%% \NewcomerWN
%% \\[.1cm]
%% \hline
%% %%%%%%%%%%%%%%%%%%%%
%% \rowcolor{Gray!30} \multicolumn{1}{|l|}{\color{Black} \ref{sec:Scrapbook}. \patternname{Scrapbook}}\\
%% \hline
%% \vspace{-.5em}
%% \ScrapbookWN
%% \\[.1cm]
%% \hline

%% \end{tabular}
%% }
%% \caption{What's next for the Peeragogy project\label{tab:WhatsNextSummary}}
%% \end{table}

%% \subsubsection*{\hyperref[sec:Peeragogy]{Peeragogy}} 
%% \PeeragogyWN

%% \subsubsection*{\hyperref[sec:Roadmap]{Roadmap}} 
%% \RoadmapWN

%% \subsubsection*{\hyperref[sec:Reduce, reuse, recycle]{Reduce, reuse, recycle}}
%% \ReduceWN

%% \subsubsection*{\hyperref[sec:Carrying capacity]{Carrying capacity}} 
%% \CarryingWN

%% \subsubsection*{\hyperref[sec:A specific project]{A specific project}}
%% \SpecificWN

%% \subsubsection*{\hyperref[sec:Wrapper]{Wrapper}}
%% \WrapperWN

%% \subsubsection*{\hyperref[sec:Heartbeat]{Heartbeat}}
%% \HeartbeatWN

%% \subsubsection*{\hyperref[sec:Newcomer]{Newcomer}}
%% \NewcomerWN

%% \subsubsection*{\hyperref[sec:Scrapbook]{Scrapbook}} 
%% \ScrapbookWN

\newpage
