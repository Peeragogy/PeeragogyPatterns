% DK: My book is out of print now, but the chapter in Rhythm is available online, http://www.informit.com/articles/article.aspx?p=28785 You might find some ideas that lineup with this pattern there
\section{Heartbeat}\label{sec:Heartbeat}

\subsubsection*{Context}
% DK: I think you might be missing some pieces of context….people are busy, the FLOSS project is probably not their day job, etc…
A number of people have a shared interest, and have connected with each other about it.  However, they are not going to spend 24 hours a day, 7 days a week working together, either because they are busy with other things, or because working separately on some tasks is vastly more efficient.

\subsubsection*{Problem} How will the effort be sustained and coordinated sufficiently?  What makes the difference between an interest group and an active collaboration?  Is there a \emph{there, there?}

\subsubsection*{Solution} People seem to naturally gravitate to regularly scheduled activities.  \emph{Once a day} (standups), \emph{once a week} (meetings), or \emph{once a year} (conferences, festivals) are common variants.  When the project is bigger than more than just a few people, it's likely that there will be several \patternnameplural{Heartbeat}, building a sophisticated polyrhythm, and the project will feel ``like an improvisational jazz ensemble'' \cite{david2001software}.  Much as the band director may gesture to specific players to invite them to solo or sync up, a project facilitator may craft individual emails to ask someone to lead an activity or invite them to re-engage.  Two common rhythm components are weekly synchronous meetings with an open agenda, combined with \emph{ad hoc} meetings for focused work on \patternname{A specific project}.  The precise details will depend on the degree of integration required by the group.

\subsubsection*{Rationale}  This pattern might seem too obvious, since regularly scheduled meetings are so ubiquitous.  
But there's an important difference between a mere meeting and a \patternname{Heartbeat}.  The project's heartbeat is what sustains it. Just as the people matter more than the code, so does the life of the working group matter more than precision of the structure (see \patternname{Carrying capacity}).
% DK: Help the reader understand what this difference is...

\subsubsection*{Resolution} Nearly everyone is familiar with the power of a supportive routine.  Used mindfully, the \patternname{Heartbeat} can be a sophisticated tool.  Noticing when a new \patternname{Heartbeat} is beginning to emerge is a way to be aware of the shifting priorities in the group, and may be a good source of new patterns.  Similarly, if a specific \patternname{Heartbeat} has faded it may be a sign that one of our patterns should be moved to the \patternname{Scrapbook}.

\begin{framed}
\emph{What's Next.}
Identifying and fostering new \patternnameplural{Heartbeat} and new working groups is a task that can help make the community more robust.  This is the time dimension of spin off projects described in \patternname{Reduce, reuse, recycle}.
\end{framed}


