\section{Heartbeat}

\paragraph{Context:}
People have a shared interest, and have connected with each other about it.

\paragraph{Problem:} What's an easy way for these people feel like there's a ``there, there?''

\paragraph{Solution:} People seem to naturally gravitate to regularly scheduled
activities. Once a week (meeting) or once a year (conference) are two common variants.  Sometimes people need a little extra prompt to join in.\footnote{In the ``Collaborative Lesson Planning'' course led
by Charlie Danoff at P2PU, Charlie wrote individual emails to people who
were signed up for the course and who had disappeared, or lurked but
didn't participate. This kept a healthy number of the people in the
group to re-engage and make positive contributions. In more recent
months, Charlotte Pierce has been running weekly meetings by Google
Hangout to coordinate work on the Peeragogy Handbook. Not only have we
gotten a lot of hands-on editorial work done this way, we've generated a
tremendous amount of new material (both text and video footage) that is
likely to find its way into future versions of the \emph{Peeragogy Handbook}.}

\paragraph{Rationale:}  This pattern might seem too obvious, since regularly scheduled meetings are so ubiquitous.  But there's an important difference between a mere meeting and a \emph{Heartbeat}: in short, if the energy from your meetings isn't helping you or your group thrive, something needs to change.

\paragraph{Resolution:} This pattern is one of the easiest to explain to \emph{Newcomers} to the idea of design patterns, since nearly everyone is familiar with the pattern of regular routine.  But the pattern is also a sophisticated tool: noticing when a new \emph{Heartbeat} occurs is a way to be aware of the priorities in the group, and may be a good source of new patterns.

\paragraph{What's Next:} When the project is bigger than more than just a few people, it's likely to have several \emph{Heartbeats}\footnote{We've operated two weekly meetings in the Peeragogy project at several times, for members with slightly different interests and slightly different availability.  Often this relates to small special-purpose projects, like our work on this paper.}  Identifying and fostering new \emph{Heartbeats} and new working groups is a task that can help make the community more robust.
