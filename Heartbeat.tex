% DK: My book is out of print now, but the chapter in Rhythm is available online, http://www.informit.com/articles/article.aspx?p=28785 You might find some ideas that lineup with this pattern there
\section{Heartbeat}\label{sec:Heartbeat}

\subsubsection*{Context}
% DK: I think you might be missing some pieces of context….people are busy, the FLOSS project is probably not their day job, etc…
A number of people have a shared interest, and have connected with each other about it.  However, they are not going to spend 24 hours a day, 7 days a week working together, either because they are busy with other things, or because working separately on some tasks is vastly more efficient.
\textbf{Even if we do spend lots of time together, it isn't all equally meaningful.  Something needs to hold the project together, or it will fall apart.}

\subsubsection*{Problem} How will the effort be sustained and coordinated sufficiently?  How do we know this an active collaboration, and not just a bunch of people milling about?  Is there a \emph{there, there?}  

\subsubsection*{Solution} People seem to naturally gravitate to something with a pulse.  \emph{Once a day} (standups), \emph{once a week} (meetings), or \emph{once a year} (conferences, festivals) are common variants.  When the project is populated by more than just a few people, it's likely that there will be several \patternnameplural{Heartbeat}, building a sophisticated polyrhythm.  A well-running project will feel ``like an improvisational jazz ensemble'' \cite{david2001software}.  Much as the band director may gesture to specific players to invite them to solo or sync up, a project facilitator may craft individual emails to ask someone to lead an activity or invite them to re-engage.  Two common rhythm components are weekly synchronous meetings with an open agenda, combined with \emph{ad hoc} meetings for focused work on \patternname{A specific project}.  The precise details will depend on the degree of integration required by the group.

\subsubsection*{Rationale}  The project's heartbeat is what sustains it. Just as \emph{people matter more than code} \cite{torvalds-interview}, so does the life of the working group matter more than mechanics of the work structure.  Indeed, there is an quick way to do a reality check and find the project's strongest pulse: the activities that sustain a healthy project should sustain us, too (cf. \patternname{Carrying capacity}).
% DK: Help the reader understand what this difference is...

\subsubsection*{Resolution} Used mindfully, the \patternname{Heartbeat} can be a sophisticated tool.  Noticing when a new \patternname{Heartbeat} is beginning to emerge is a way to be aware of the shifting priorities in the group, and may be a good source of new patterns.  Like a \patternname{Heartbeat}, patterns recur.  On the other hand, if a specific activity is no longer sustaining the project, stop doing it, much as you would move an out-of-date pattern to the \patternname{Scrapbook} in order to make room for other concerns.
%
The power of the \patternname{Heartbeat} is that the project can be as focused and intensive as it needs to be.

\subsubsection*{Example 1} The yearly in-person gathering, Wikimania, is the most visible
example of a \patternname{Heartbeat} for the Wikimedia movement.\footnote{\url{https://meta.wikimedia.org/wiki/Wikimania}}
Local chapters and projects may run additional in-person get-togethers.\footnote{\url{http://wikiconferenceusa.org/}}
Also of note is the twice-yearly call for proposals for individual
engagement grants.\footnote{\url{https://meta.wikimedia.org/wiki/Grants:IEG}}

\subsubsection*{Example 2} Although it may sound quaint, working farms could help to physically
sustain peeragogues, while putting the project's \patternname{Heartbeat} in tune with that of the seasons.  In the
current distributed mode, we tend our windowboxes and allotment gardens.

\begin{framed}
\noindent 
\emph{What's Next.}
Identifying and fostering new \patternnameplural{Heartbeat} and new working groups is a task that can help make the community more robust.  This is the time dimension of spin off projects described in \patternname{Reduce, reuse, recycle}.
\end{framed}


