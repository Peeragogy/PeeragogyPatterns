\documentclass{acmlarge}

\usepackage{showframe}
% Support for CCSXML file
\RequirePackage{comment}
\excludecomment{CCSXML}

% New concepts scheme
%
% The first argument is the significance, the
% second is the concept(s)
%
% https://dl.acm.org/ccs.cfm?
\makeatletter
\let\@concepts\@empty

\def\category#1#2#3{\@ifnextchar
  [{\@category{#1}{#2}{#3}}{\@xcategory{#1}{#2}{#3}}}
\def\@category#1#2#3[#4]{\edef\@tempa{\ifx \@concepts\@empty 
    \else ; \fi}{\def\protect{\noexpand\protect
      \noexpand}\def\and{\noexpand\and}\xdef\@concepts{\@categories\@tempa #1
      [{\bf #2}]: 
      #3\kern\z@---\hskip\z@{\it #4}}}}
\def\@xcategory#1#2#3{\edef\@tempa{}{\def\protect{\noexpand\protect\noexpand}\def\and{\noexpand
      \and}\xdef\@categories{\@categories\@tempa #1}}}
\def\@categories{}

\newcommand\ccsdesc[2][100]{%
  \ccsdesc@parse#1~#2~}
%
% The parser of the expression Significance~General~Specific
%
\def\ccsdesc@parse#1~#2~#3~{%
  \expandafter\ifx\csname CCS@#2\endcsname\relax
    \expandafter\gdef\csname CCS@#2\endcsname{\textbullet\textbf{#2} $\to$ }%
  \g@addto@macro{\@concepts}{\csname CCS@#2\endcsname}\fi
  \expandafter\g@addto@macro\expandafter{\csname CCS@#2\endcsname}{%
    \ifnum#1>499\textbf{#3; }\else
    \ifnum#1>299\textit{#3; }\else
    #3 \fi\fi}}

\newcommand\printccsdesc{%
 {\footnotesize \@concepts}}

\def\maketitle{\newpage \thispagestyle{titlepage}\par
  \begingroup \lineskip = \z@\null \vskip -13.5pt\relax 
  \parindent\z@ {\hyphenpenalty\@M
    {\titlefont \@title \par
    \global\firstfoot
    \global\runningfoot
  }}
  \global\@firstpg\the\c@page
      {\vskip 13.5pt\relax \normalsize \authorfont %vskip 13.5pt between title and author
	\begingroup \addtolength{\baselineskip}{2pt}
	\linespread{0.5}\@author\par \vskip -2pt 
	\endgroup }
      {\ifx \@categories\@empty 
	\else 
	\baselineskip 17pt\relax
	\hbox{\vrule height .2pt width \@acmWidth}%to eliminate the lines for jacm
      }
      \vskip 8.5pt \footnotesize \box \@abstract \vskip 4pt\relax %vskip8.5 space above abstract
	     {\def\and{\unskip\/{\rm ; }}
	       Categories and Subject Descriptors: \@categories \fi}\par\vskip 4pt\relax
	     \box\@terms \vskip 4pt\relax
	     \box\@keywords \par
	     \ifx\@acmformat\@empty\else
             \footnotesize \hsize \@acmWidth \parindent 0pt \noindent
             \vskip 4\p@
             \noindent  {\bf ACM Reference Format:}\\[2pt]
             \@acmformat\vskip 0.5\p@
             \par\fi%
		 {\baselineskip 14pt\relax
		   \@abstractbottom
		 }
		 \vskip 23pt\relax
		 \endgroup
		 \let\maketitle\relax
		 \gdef\@categories{}}
\makeatother

\usepackage{tikz}
\usetikzlibrary{calc}
\usetikzlibrary{positioning,backgrounds,fit,arrows,arrows.meta,shapes,shadows}
\usetikzlibrary{shapes.multipart}
\usepackage{pgflibraryarrows}

% Metadata Information
\makeatletter
\def\@journalNameShort{$\mathit{jn}$}
\makeatother
\acmVolume{$V$}
\acmNumber{$N$}
\acmArticle{$n$}
\articleSeq{$m$}
\acmYear{2015}
\acmMonth{5}

\usepackage{latexsym}
\usepackage{amsfonts,amsmath,amssymb}
\usepackage{url}
\usepackage[utf8]{inputenc}
\PassOptionsToPackage{hyphens}{url}
\usepackage{hyperref}
\makeatletter
\g@addto@macro{\UrlBreaks}{\UrlOrds}
\makeatother
%\usepackage[hyphenbreaks]{breakurl}
%\usepackage[hyphens]{url}
\hypersetup{colorlinks=false,pdfborder={0 0 0}}
%\usepackage{textcomp}
%\usepackage{longtable}
\usepackage{multirow,booktabs}

\title{Patterns of Peeragogy}
\author{JOSEPH CORNELI \affil{Department of Computing, Goldsmiths College, University of London}\ \\
CHARLES JEFFREY DANOFF \affil{Mr Danoff's Teaching Laboratory}\ \\
CHARLOTTE PIERCE \affil{Pierce Press}\ \\
PAOLA RICAURTE \affil{Department of Cultural Studies, Tecnol\'ogico de Monterrey}\ \\  
LISA SNOW MACDONALD \affil{independent researcher}}

\begin{CCSXML}
<ccs2012>
<concept>
<concept_id>10010405.10010489.10010492</concept_id>
<concept_desc>Applied computing~Collaborative learning</concept_desc>
<concept_significance>500</concept_significance>
</concept>
<concept>
<concept_id>10003120.10003130.10003233</concept_id>
<concept_desc>Human-centered computing~Collaborative and social computing systems and tools</concept_desc>
<concept_significance>300</concept_significance>
</concept>
<concept>
<concept_id>10003456.10003457.10003490.10003491</concept_id>
<concept_desc>Social and professional topics~Project and people management</concept_desc>
<concept_significance>300</concept_significance>
</concept>
<concept>
<concept_id>10003456.10003462.10003463.10003470</concept_id>
<concept_desc>Social and professional topics~Licensing</concept_desc>
<concept_significance>100</concept_significance>
</concept>
</ccs2012>
\end{CCSXML}

\ccsdesc[500]{Applied computing~Collaborative learning}
\ccsdesc[300]{Human-centered computing~Collaborative and social computing systems and tools}
\ccsdesc[300]{Social and professional topics~Project and people management}
\ccsdesc[100]{Social and professional topics~Licensing}

% \category{K.3.1}{Computer Uses in Education}[Collaborative learning]
% \category{H.5.3}{Group and Organization Interfaces}[Collaborative computing]
% \category{K.6.1}{Project and People Management}[Systems analysis and design]

\begin{abstract}
We describe eleven design patterns that we have developed in our work on the Peeragogy project, in which we aim to design the future of education along the principles of free/libre/open source software.  We use these patterns to build a ``distributed roadmap'' for the project.
\end{abstract}

\category{\noexpand\printccsdesc}{}{}

\keywords{peer production, peer learning, design patterns}

\acmformat{Corneli, J., Danoff, C. J., Pierce, C., Ricaurte, P.,  Snow MacDonald, L. 2015.Patterns of Peeragogy.}

\copyr{PLoP'15, October 25-30, Pittsburgh, Pennsylvania, USA. Copyright 2015 is held by the author(s). ACM XXX-X-XXXX-XXXX-X}


\begin{document}
\begin{bottomstuff}
While preparing this paper Joseph Corneli was supported by the Future and Emerging
Technologies (FET) programme within the Seventh Framework Programme
for Research of the European Commission, under FET-Open Grant number
611553 (COINVENT).\\
Correspondence address: J. Corneli, Department of Computing, Goldsmiths, University of London, New Cross, London SE14 6NW; email: j.corneli@gold.ac.uk\\

Permission to make digital or hard copies of all or part of this work for personal or classroom use is granted without fee provided that copies are not made or distributed for profit or commercial advantage and that copies bear this notice and the full citation on the first page. To copy otherwise, to republish, to post on servers or to redistribute to lists, requires prior specific permission. A preliminary version of this paper was presented in a writers' workshop at the 22nd Conference on Pattern Languages of Programs (PLoP). 
\end{bottomstuff}

\maketitle


\section{Peeragogy Project}

\begin{quote}
This section introduces the \emph{Peeragogy Project} in the form of a \emph{design pattern}.
\end{quote}

\textbf{Context:}  The Peeragogy project began in 2011.  To date it has been run by a small and changing group of interested volunteers, collaboring via real-time meetings and shared documents that describe the patterns and processes of peer learning we've observed in this project and others.   We are not currently an official organization, but we have some lofty goals.  As time goes by we would like to become be a knowledge capital competitor/collaborator with Wikipedia, somewhat akin to StackExchange.

Architectual maverick Christopher Alexander asked the following question to an audience of computer programmers in 1999: 
\begin{quote}
``What is the Chartres of programming? What task is at a high enough level to inspire people writing programs, to reach for the stars?''
\end{quote}
We believe that the nexus of learning and computers -- exemplified today by Wikipedia, StackExchange, and the Peeragogy Project -- may be just the thing.

\textbf{Problem:} In a volunteer context, telling people what to do really doesn't work.  So we need another way to communicate.

\textbf{Solution:} Christopher Alexander introduced the idea of \emph{design patterns} -- using a simple template to describe and build the human life world.  In our context, design patterns allow us to bridge physical and virtual, move from fantastic to concrete and back.  The template we use is relatively traditional.  We've made a few minor alterations to Alexander's original model, in order to use patterns in a project that is always changing shape.  Specifically, we think of each pattern as something active: we write down the specific benefits of documenting the pattern as ``Resolution of forces'' and write down the ``What's next'' steps.  Like Alexander, we cross-reference our patterns to understand the links between them.  The \emph{Peeragogy Project} pattern is itself an up-to-date example of one of Alexander's patterns, \href{http://en.wikipedia.org/wiki/Networked_learning#1970s}{\emph{Network of Learning}}.

\textbf{Rationale:}
Patterns are intuitive to write and read.  Our specific interperation of the framework helps us document what we've learned, and even moreso, helps everyone involved keep learning.  We use our design pattern catalogue to build on what we've learned so far, and to scaffold our work on other parts of the project -- our technical platform, our Handbook, our meetings -- and to connect in fruitful ways with other projects.  

\textbf{Pattern:}
The idea of a shared roadmap has been with us since the beginning of the \emph{Peeragogy Project}.  Of course, the roadmap itself changes as time goes by.  More precisely, the \emph{Peeragogy Project} uses actively updated patterns to develop a current but never complete distributed \emph{Roadmap} for our shared efforts.  If we sense that something needs to  change about the project, that’s a clue that we might need to record or adapt one of our patterns. 

\textbf{Resolution of forces:}  
Writing down this pattern defines some of the key terms for \emph{Newcomers}, and illustrates the way our template works.

\textbf{What's next:} 
This project has proved to be interesting, social, and with time may offer a range of new business model for education -- less lucrative perhaps, but more rewarding.  We've seen that participating in the peeragogy project gives us technical social and theory-building skills we can use in our day jobs for collaborating with colleagues.  Feel free to join us!
\section{Peeragogy}\label{sec:Peeragogy} 
%% Possibly change to Peeragogy in Action.
%% Maybe change Wrapper to Wrap Up, or call the other patterns according to roles
% DK: I question whether the Peeragogy Project is a pattern, or an instance of a pattern. We generally think about patterns as something that can be instantiated. But your project seems like a specific case of the concept.
% JC: There could be many projects

\subsubsection*{Motivation} This pattern provides an entry point to the other patterns.  It is relevant to anyone who wants to do active learning together with others in a relatively non-hierarchical setting.

\subsubsection*{Context}  Architectual maverick Christopher Alexander asked the following questions to an audience of computer programmers \cite{alexander1999origins}: 
\begin{quote}
``What is the Chartres of programming? What task is at a high enough level to inspire people writing programs, to reach for the stars?''
\end{quote}
In order for humanity to pull itself up by its bootstraps, on this planet or any other, we need to continue to learn and adapt.  Collaborative projects like Wikipedia, StackExchange, and FLOSS represent an implicit challenge to the old ``industrial'' organization of work.  This new way of working appears to promise something more resilient, more exciting, and more humane.  In the context of these free, open, post-modern organizations, individual participants are learning and growing -- and adapting the methods and infrastructure as they go.
Because everyone in these projects primarily learns by putting in effort on a shared work-in-progress, participants are more in touch with an \emph{equality of intelligence} than an \emph{inequality of knowledge} \cite[pp.~38,119]{ranciere1991ignorant}.
At the same time, they invoke a form of friendly competition, in which \emph{the best craftmanship wins} \cite[p.~89]{raymond2001cathedral}.
%%% FORCES

\subsubsection*{Forces}

\raisebox{-.5\baselineskip}
{{\centering
\begin{tabular}{p{.85\textwidth}}
\textbf{Threshold}: there is a tension between inclusiveness and specificity.\\
\textbf{Trust}: is only built through sharing and reciprocity.
\end{tabular}

\par}}

\subsubsection*{Problem} Even a highly successful project like Wikipedia is a work in progress that can be improved to \emph{\emph{better} empower and engage people around the world, to develop \emph{richer and more useful} educational content, and to disseminate it \emph{more} effectively} -- and deploy it more creatively.\footnote{\url{https://wikimediafoundation.org/wiki/Mission_statement}}  How to go about this is a difficult question, and we don't know the answers in advance.  There are rigorous challenges facing smaller projects as well, and fewer resources to draw on.  Many successful free software projects are not particularly collaborative -- and the largest projects are edited only by a small minority of users \cite{free-software-better,who-writes-wikipedia}.  Can we work smarter together?

\subsubsection*{Solution} People who learn actively together talk to each other about material problems, share practical solutions, and constructively critique works-in-progress.  There are many different ways to go about this -- bug reports, mailing lists, writers workshops, Q\&A forums, watercoolers and skateparks are all places where peeragogy can happen.  In the Peeragogy project have found that the necessary ``reflection'' aspects of the process are particularly well-matched to Christopher Alexander's idea of a \emph{pattern language}, in which commonly occurring,  interconnected, elements of an optative design are refined until they can be described in terms of a simple template.  Indeed, thought of as a design pattern, \patternname{Peeragogy} can be understood as an up-to-date revision of Alexander's \patternnameext{Network of Learning} \cite[p. 99]{alexander1977pattern}.  It \emph{decentralizes the process of learning and enriches it through contact with many places and people} -- in interconnected networks that may reach all over the world.   Importantly, while people involved in a peeragogical process may be collaborating on \patternname{A specific project}, they don't have to be direct collaborators outside of the learning context or co-located in time or space.  Peeragogy often takes place in mostly-horizontal relationships between people who have different but compatible objectives.  The techniques we describe with these patterns are in many cases ancient; one can compare, for instance, the Quechua communal working practice of \emph{mink'a}.  The underlying practices can be pursued with or without high technology and with or without the technique of design patterns.  
% Can we clarify that we're describing Peeragogy as a pattern, and a pattern language, and that it is useful whether or not others decide to use patterns...
%OSS: We talk about patterns - does this mean learners are meant to organize their knowledge in new patterns? Would be great if learners made new patters, but we're not requiring that and we need to be more explicit about that? May also be useful for people into pattersns


\subsubsection*{Rationale}
% DK: I have written many, and shepherded many more, and I don’t think this is the case. However, the effort that goes into writing them makes them more intuitive to read :+)
The peeragogical approach particularly addresses the problems of small projects stuck in their individual silos, and large projects becoming overwhelmed by their own complexity.  It does this by going the opposite route: explicating \emph{what by definition is tacit} and employing \emph{a continuous design process} \cite[pp. 9--10]{schummer2014beyond}.  The very act of asking ``can we work smarter together?'' puts learning front and center.  \patternname{Peeragogy} takes that ``center'' and distributes it across a pool of heterogeneous relationships.  As pedagogy articulates the transmission of knowledge from teachers to students, peeragogy articulates the way peers produce and use knowledge together (Figure \ref{fig:connections}).  Active learning together with others brings social and emotional intelligence to bear on the things that matter most.

\subsubsection*{Resolution}

Peeragogy helps people in different projects describe and solve real problems. 
If you share the problems that you're experiencing in your project, someone may be able to help you solve them.  Bringing a problem across the \textbf{threshold} of someone else's awareness helps achieve clarity.  
This process can guide individual action in ways that we wouldn't have seen on our own, and may lead to new forms of collective action we would never have imagined possible.  People who gain experience comprehending problems together build \textbf{trust}.
%
Making room for multiple right answers helps to resolve the tension between generality and specificity.
The Peeragogy project is one of ``tens of thousands of projects in the traditions of world improvement \'elan -- without any central committee that would have to, or even could, tell the active what their next operations should be'' \cite[p. 402]{sloterdijk2013change}.  When we talk about ``next steps,'' we aim to clarify our own commitments, and show what can be realistically expected from us.

\subsubsection*{Example 1} Wikipedia and its sister sites rely on user generated content,
peer produced software, and are managed, by and large, by a pool of users who choose to
get involved with governance and other ``meta'' duties.
%
Wikimedia's pluralistic approach achieves something quite impressive: the
Wikimedia Foundation runs the 7\textsuperscript{th} most popular
website in the world, and has under 300 employees.  For comparison,
the 6\textsuperscript{th} (Amazon) and 8\textsuperscript{th} (QQ) most popular
websites are run by companies with over 200K and 28K employees,
respectively.\footnote{\url{https://en.wikipedia.org/wiki/Wikimedia_Foundation\#Employees}}\textsuperscript{,}\footnote{\url{http://phx.corporate-ir.net/phoenix.zhtml?c=97664&p=irol-newsArticle&ID=2100418}}\textsuperscript{,}\footnote{\url{https://www.google.com/finance?cid=695431}}\textsuperscript{,}\footnote{\url{http://www.alexa.com/topsites}}
%OSS: which ones are they?   

\subsubsection*{Example 2} Although one of the strengths of \patternname{Peeragogy} is to
distribute the workload, this does not mean that infrastructure is
irrelevant.  No less than their predecessors, the students and
researchers of the future university will need access to an
Observatory and other scientific apparatus if they are truly to reach \emph{ad astra, per aspera}.\footnote{``With difficulty, to the stars.''}

\begin{framed}
\noindent 
\emph{What's next.}
\definecollection{PeeragogyWN}
\begin{collectinmacro}{\PeeragogyWN}{}{}
We intend to revise and extend the patterns and methods of peeragogy to make it a workable model for learning, inside or outside of institutions.
\end{collectinmacro}
\PeeragogyWN
\end{framed}

% People won't get invested without a return, although this may not be the same for everyone.
% , and the idea of a specific goal or something concrete on offer


  

  
  
  
  
  
  
  

\section{Roadmap} \label{sec:Roadmap}

% DK: This is a bit self-referential. You have roadmap embedded throughout the description of the pattern. E.g. the context and problem should be able to describe the situation before the solution has been applied, so you should be able to describe them without the solution name in them.

% DK: How is this different from a Backlog? Who “owns” the roadmap? How are changes made to it? How precise is it? Does it have a time dimension to it? You refer to deadlines later on. Does the roadmap include them? (What do deadlines really mean in a project like this anyhow?) Priority?

\subsubsection*{Motivation} This pattern describes the main ``design object'' that is of interest in peeragogy: the group's communication about their work-in-progress to address their shared goals.  This is the central pattern in our pattern language. 


\subsubsection*{Context} \patternname{Peeragogy} has both distributed and centralized aspects. The discussants or contributors who collaborate on a project have different points of view and heterogeneous priorities, but they come together in conversations and joint activities.

\subsubsection*{Forces}~
\parbox[t]{.85\textwidth}{
\textbf{Variety}: people have different goals and interests in mind.\\
\textbf{Clarity}: some may be quite specific, and some rather vague.\\
\textbf{Coherence}: some of these goals will be well-aligned, others less so.
}

\subsubsection*{Problem} In order to collaborate, people need a way to share current, though incomplete, understanding of the space they are working in, and to nurture relationships with one another and the other elements of this space.  At the outset, there may not even be a coherent vision for a project -- but a only loose collection of motivations and sentiments.  Once the project is up and running, people are likely to pull in different directions.   

\subsubsection*{Solution} Building a guide to the goals, activities, experiments and working methods can help \patternnameplural{Newcomer} and old-timers alike understand how the nature of their relationship with the project.  This guide may be a research question or an outline, an organizational mission statement, or a business plan.  It may be a design pattern or a pattern language \cite{kohls2010structure}.  It may combine features of a manifesto, a syllabus, and an issue tracker.  The distinguishing qualities of a project \patternname{Roadmap} are that it should be adaptive to circumstances, and that it should ultimately get us from \emph{here} to \emph{there}.  By this same token, any given version of the roadmap is seen as fallible.  Everyone with an interest in the project should have the right to update it, although in practice not everyone will choose to do so.  In lieu of widespread participation, the project's \patternname{Wrapper} should attempt to synthesize an accurate roadmap that is informed by participants' behavior, and should help moderate in case of conflict.  Nevertheless, full consensus is not necessary: different goals, with different \emph{heres} and \emph{theres}, can be pursued separately, while maintaining communication.

\subsubsection*{Rationale} 
% DK: This seems more like advice about how to implement the solution than it is an explanation of how the solution addresses the forces from the context/problem [jc: fixed]
An adaptive roadmap that incorporates multiple simultaneous solution
paths can achieve integration around core values without
over-determining or over-constraining participation.  The structure of
the roadmap needs to be able shift along with its contents: it is an
antidote to \patternnameext{Tunnel Vision}
\cite[pp. 121--124]{david2001software}.  In the Peeragogy project our
initial roadmap was an outline of the first draft of the
\emph{Peeragogy Handbook}.  Later, it took the form of a schedule of
meetings following a regular ``\patternname{Heartbeat}'' supplemented
by a list of upcoming submission deadlines.  Most recently, it is
expressed in the emergent objectives listed in Section
\ref{sec:Distributed_Roadmap} of the current paper.  In each case the
method used to maintain the roadmap, was ``distributed'', and the
generated artifact itself, ``emergent''.  By contrast, we've seen that
a list of nice-to-have features created in a top-down fashion is
comparatively unlikely to \emph{go} anywhere.  A backlog of tasks and
a realistic plan for accomplishing them are vastly different things.

\subsubsection*{Resolution}
% DK: This seems like Rationale to me [jc: fixed]
An emergent roadmap is rooted in real problems and justifiable
solutions-in-progress in all their \textbf{variety} and communicates
both resolution and followthrough.  The process of meshing varied
issues with one another requires thought and discussion, and this
encourages \textbf{clarity}.  The test of \textbf{coherence} is that
contributed goals and ideas should be actionable.
%
One quality-control test for the roadmap as a whole is that it should
give a \patternnameplural{Newcomer} a reasonable idea of what it would
mean to participate in the project, and help them decide whether,
where, and how to get involved.

\subsubsection*{Inversion}
The \patternname{Roadmap} is something of a paradox.  We can't dictate
the behavior of other participants, and we often can't even guess
ourselves what's coming up.  A peeragogical \patternname{Roadmap}
should prepare people for the \emph{absence} of clear step-by-step
direction, the \emph{presence} of different view points and
priorities, and the need to be relatively self-directed.  This pattern
isn't particularly suitable for a project that needs to be managed in
a top-down fashion, and that can rely on other coordination tools
(like contracts) to manage work.

\subsubsection*{Example 1}  The \emph{Help} link present on every Wikipedia page could be seen as a
localized \patternname{Roadmap} for individual user
engagement.\footnote{\url{https://en.wikipedia.org/wiki/Help:Contents}}
It tells users what they can do on the site, and gives instructions
about how to do it:

\begin{quotation}
\noindent 
I want to \emph{read} or \emph{find} an article \ldots; 
I want to \emph{edit} an article \ldots;
I want to \emph{report a problem} with an article \ldots;
I want to \emph{create a new article} or \emph{upload media} \ldots;
I have a \emph{factual question}\ldots
~[Etc.]
\end{quotation}

For someone who is prepared to jump in and get to work, there are
around 30 pages listing articles with various kinds of problems, for
example articles tagged with style issues, or ``orphaned'' articles
(i.e., articles with no links from other pages in the
encyclopedia).\footnote{\url{https://en.wikipedia.org/wiki/Category:Wikipedia_article_cleanup}}\textsuperscript{,}\footnote{\url{https://en.wikipedia.org/wiki/Category:Wikipedia_articles_with_style_issues}}\textsuperscript{,}\footnote{\url{https://en.wikipedia.org/wiki/Category:All_orphaned_articles}}

Community-organized WikiProjects and official Wikimedia projects also announce their objectives  and invite outside involvement (cf.~\patternname{A
  specific project}).  Wikimedia previously developed
a detailed strategic plan drawing on community input
\cite{wikimedia2011plan}.  The current description of the State of
the Wikimedia Foundation includes a pointer to a two-week 2015
Strategy Community Consultation (now closed).\footnote{\url{https://meta.wikimedia.org/wiki/Communications/State_of_the_Wikimedia_Foundation}}\textsuperscript{,}\footnote{\url{https://blog.wikimedia.org/2015/02/23/strategy-consultation/}}\textsuperscript{,}\footnote{\url{https://meta.wikimedia.org/wiki/2015_Strategy/Community_consultation}}  The resulting synthesis appears in a blog post by Terence Gilbey, the WMF's Interim Chief Operating Officer.\footnote{\url{https://blog.wikimedia.org/2015/08/27/strategy-potential-mobile-multimedia-translation/}}  He notes: ``Unlike in past years, we are approaching strategy not as a set of goals or objectives, but rather as a direction that will guide the decisions for the organization.''

\subsubsection*{Example 2}
In the future university run in a peer produced manner, a fancy
President's Residence would be not be needed.  However it may be
appropriate for project facilitators to gather at a University Hall
for the primary purpose of working together on the university's
\patternname{Roadmap}.  For now, we mostly meet online, and in person
less frequently: at cafes, when passing through town, or at
conferences.  In New York alone, there are a million members of
meetup.com with similar habits, although they most likely have never
heard of
peeragogy.\footnote{\url{http://blog.meetup.com/thanks-a-million-ny/}}
There is strength in numbers -- but there is leverage in organization.
Whatever we balance we strike between ``global'' and ``local''
operations, the purpose of our roadmap is to help us get organized.

\begin{framed}
\noindent
\emph{What's Next in the Peeragogy Project}
\definecollection{RoadmapWN}
\begin{collectinmacro}{\RoadmapWN}{}{}
If we sense that something needs to change about the project, that is a clue that we might need to record a new pattern, or revise our existing patterns.
\end{collectinmacro}
\RoadmapWN
\end{framed}

% Produce versus ???
% lingo is somewhat obscure: "Use or Make" - clarify?
\section{Use or make}\label{sec:Use_or_make}
\subsubsection*{Context}
% learning?
Peer production, as the name indicates, is about producing, in other words --
``making.'' But it also involves building on (``using'') the work of others.

\subsubsection*{Problem}
People are often very attached to their own projects and don't have a sense of how their own initiatives can benefit from connecting with others.

\subsubsection*{Solution} Learning often involves recycling and remixing others' ideas and techniques. Be mindful of the value of remixing!  And make it possible for other people to remix and adapt your work too.\footnote{As a first key step, we've released the \emph{Peeragogy Handbook} using the Creative Commons Public Domain Dedication (CC0).  This legal instrument grants the greatest possible leeway to downstream users; see \url{https://creativecommons.org/publicdomain/zero/1.0/}.  Contributors need to agree to the following terms: ``\emph{I hereby waive all copyright and related or neighboring rights together with all associated claims and causes of action with respect to this work to the extent possible under the law.}''  An email to one of the handbook editors or a comment to this effect on \url{http://peeragogy.org/resources/license/} suffices.}  Show appreciation when they do.  In the case of shared content, make backups so that you don't have to worry about losing the record of idea that the other person might not have noticed was important.

\subsubsection*{Rationale} 
Many projects die because the cost of \patternname{\href{http://c2.com/cgi/wiki?ReinventingTheWheel}{Reinventing the Wheel}} [c2] is too high.  \patternname{Creating a guide} can help people avoid reinventing the wheel.\footnote{Clearly we are not the first people to notice these things!  Consider the following quote from the Wikimedia Foundation: ``Unfortunately, many of our most valuable resources for learning and evaluation are scattered across wikis, buried in archived reports, incomplete, out of date, or are only available in a single language. As a result, we sometimes find ourselves re-inventing the wheel: missing opportunities, repeating common mistakes, and working harder than we need to because we are not aware of related projects done by others who came before us.'' via \url{https://blog.wikimedia.org/2013/11/19/learning-patterns-new-way-share-important-lessons/}.}  There are lots of tools out there -- \emph{use them}, at least on a trial basis; return to the ones that work.

\subsubsection*{Resolution} Noticing how difficult it is to remake things every time, and encapsulating what we observed with the \patternname{Use or Make} pattern reminds us to always consider re-purposing the work of others, to think about how others can leapfrog ahead, building on our experiences by incorporating our work.

\subsubsection*{What's Next} We've spun off the pattern catalog from the \emph{Peeragogy Handbook} into this paper, sharing it with a new community and gaining new perspectives.  Let's look for other parts of the handbook we can spin off!


\section{Carrying capacity}\label{sec:Carrying capacity}

\subsubsection*{Motivation} This pattern can help project participants recognise and communicate their stresses to make themselves and the project  more resilient.


%% \begin{center}
%% \begin{tabular}{l}
%% \textbf{$\leftarrow$\patternname{Reduce, reuse, recycle}: All available perspectives can give the project more to work with.}\\
%% \textbf{$\leftarrow$\patternname{A specific project}: We may need help to create or activate a plan.}\\
%% \textbf{$\leftarrow$\patternname{Wrapper}: Share skills and be transparent about limitations and bottlenecks.}\\
%% \textbf{$\leftarrow$\patternname{Heartbeat}: Project activites should give us rewards, not drain our energy.}\\
%% \end{tabular}
%% \end{center}

\subsubsection*{Context}

One of the important maxims from the world of FLOSS is:
``Given enough eyeballs, all bugs are shallow'' \cite[p.~30]{raymond2001cathedral}.
A partial converse is also true.

\subsubsection*{Forces}~
\parbox[t]{.85\textwidth}{
\textbf{Boundedness}: there's only so much any one person can do.\\
\textbf{Independence}: in a peeragogy context, it is often impossible to delegate work to others.\\
\textbf{Antifragility}: potential can only be realised if people take on enough but not too much.
}

\subsubsection*{Problem}

How can we help prevent those people who are involved with the project from overpromising or overcommitting, and subsequently crashing and burning?  First, let's be clear that are lots of ways things can go wrong.  Simplistic expectations -- like \emph{assuming that others will do the work for you} \cite{torvalds-interview} -- can undermine your ability to correctly gauge your own strengths, weaknesses, and commitments.  Without careful, critical engagement, you might not even notice when there's a problem.  Where one person has trouble letting go, others may have trouble speaking up.  Pressure builds when communication isn't going well.  
% At the same time, we all seem to have a lot to learn about how to pay attention and make constructive contributions in situations that are always changing.

\subsubsection*{Solution}

Symptoms of burnout are a sign that it's time to revisit the group's \patternname{Roadmap} and your own individual plan.  Are these realistic?  Frustration with other people is a good time to ask questions and let others answer.  Do they see things the same way you do?  Your goals may be aligned, even if your methods and motivations differ. If you have a ``buddy'' they can provide a reality check.   Maybe things are not \emph{that hard} after all -- and maybe they don't need to be done \emph{right now}.  Generalizing from this: the project can promote an open dialog by creating opportunities for people to share their worries and generate an emergent plan for addressing them \cite{seikkula2006dialogical}.  Use the project \patternname{Scrapbook} to make note of obstacles.  For example, if you'd like to pass a baton, you'll need someone there who can take it.  Maybe you can't find that person right away, but you can bring up the concern and get it onto the project's \patternname{Roadmap}.  The situation is always changing, but if we continue to create suitable checkpoints and benchmarks, then we can take steps to take care of an issue that's getting bogged down.    

\subsubsection*{Rationale}

Think of the project as an ecosystem populated by acts of participation.  As we get to know more about ourselves and each other, we know what sorts of things we can expect, and we are able to work together more sustainably \cite{ostrom2010revising}.
%
We can regulate our individual stress levels and improve collective outcomes by discussing concerns openly.

\subsubsection*{Resolution}

Guiding and rebalancing behaviour in a social context can begin with speaking up about a concern.  When we acknowledge our concerns and those of others, we take into account our \textbf{boundedness}.  Being aware of the problems and limitations that others face, we have the opportunity to help out, without impinging on others' \textbf{independence}.  For example, one person who listens to another's concerns may discover a concern of his or her own, and help create an opportunity for both people to learn.  This process is  consistent with inclusivity \cite{garrison2013toward}, but it is also caring as participants are invited to be candid about what works well for them and what does not.  This doesn't mean including all possible stresses: we work to stay within the realm of \textbf{antifragility} \cite{taleb2012antifragile}, where stress improves the system, rather than degrading them. 
%
As we share concerns and are met with care and practical support, our actions begin to align better with expectations (often as a result of forming more realistic expectations). 

\subsubsection*{Example 1}
Wikipedia aims to emphasize a neutral point of view, but its users are
not neutral.\footnote{\url{https://en.wikipedia.org/wiki/Wikipedia:Neutral_point_of_view}}
Wikipedia is relevant to things that matter to us.  It
helps inform us regarding our necessary purposes -- and we are invited
to ``speak up'' by making edits on pages that matter to us.  However,
coverage and participation are not neutral in another sense.
More information on Wikipedia deals with Europe than
all of the locations outside of Europe \citep{graham2014uneven}.
A recent solicitation for donations to the Wikimedia Foundation
says ``Wikipedia has over 450 million readers.  Less than 1\% give.''
%
As we remarked in the \patternname{Peeragogy} pattern, most of the
actual work is contibuted by a small percentage of users as well.
%
Furthermore, the technology limits what can be said; 
\cite{graham2014uneven} remark on
``the structural inability of the platform itself to incorporate fundamental epistemological diversity.''
%
Finally, the overall population of editors is an important concern for
the Wikimedia Foundation: the total number of active editors has been
falling since
2007.\footnote{\url{https://strategy.wikimedia.org/wiki/Editor_Trends_Study/Results}}

\subsubsection*{Example 2}
A separate Ladies Hall seems entirely archaic.  Progressive thinkers have for
some time subscribed to the view that ``there shall be no women in
case there be not men, nor men in case there be not women''
\cite[Chapter 1.LII]{rabelais1894gargantua}.  However, in light of the
extreme gender imbalance in free software, and still striking
imbalance at Wikipedia \cite{gender,FM4291}, it will be important to
do whatever it takes to make women and girls welcome, not least
because this is a significant factor in boosting our
\patternname{Carrying capacity}.

%\subsubsection*{Summary}

\begin{framed}
\noindent 
\emph{What's Next in the Peeragogy Project}
\definecollection{CarryingWN}
\begin{collectinmacro}{\CarryingWN}{}{}
Making it easy and fruitful for others to get involved is one of the best ways to redistribute the load.  This often requires skill development among those involved; compare the \patternname{Newcomer} pattern.
\end{collectinmacro}
\CarryingWN
\end{framed}



  

\section{A specific project}\label{sec:A_specific_project}
\subsubsection*{Context}
You find yourself interested in or concerned about something.

\subsubsection*{Problem}
It's easy to think about issues that matter: there are many of them. The problem is figuring out what you're going to do about it.

\subsubsection*{Solution} 
Being concrete about what you'd like to do, learn, and achieve, takes you from thinking about a topic to becoming a practitioner.  You may realize that your ``specific project'' is too large to tackle directly. In this case, you will have to become even more specific.  Maintaining a project \patternname{Roadmap} can help keep track of the smaller pieces and the bigger picture.

\subsubsection*{Rationale} 
Being specific is important for bringing about to change.\footnote{In the January, 2013, plenary
session, \href{http://ipne.org}{Independent Publishers of New England}
(IPNE) President Tordis Isselhardt quietly listened to a presentation
about how we created the \emph{Peeragogy Handbook}. During the Q\&A, she
spoke up, wondering if peer-learning effort in IPNE might be more likely
to succeed if the organization's members ``focused around a specific
project.'' As this lightbulb illuminated the room, those of us attending
the plenary session suggested that IPNE could focus the project by
creating an ``Independent Publishing Handbook.'' (Applause!) In the
course of creating the IPNE Handbook, peer learners would assemble
resource repositories, exchange expertise, and collaboratively edit
documents. To provide motivation and incentive to participate in
``PeerPubU'', members of the association will earn authorship credit for
contributing articles, editor credit for working on the manuscript, and
can spin off their own chapters as stand-alone, profit-making
publications.} But while actions speak louder than words, it's important
to act in a coherent way if you want to be understood by others.  However, in
general it would be a mistake to try to seek consensus before acting: it's much better to combine action with dialog.

\subsubsection*{What's Next}  Each project connected with the \patternname{Peeragogy Project} should be described with one or more patterns, each with specific, tangible ``what's next'' steps.\footnote{We've found that writing papers for conferences is one activity that can help us focus and make improvements to our body of work that would not come about by simply meandering through revisions to the \emph{Peeragogy Handbook}.  Remixing these efforts into the handbook is a good source of improvements; see \patternname{Use or Make}.}  The \patternname{Pattern Audit Routine} can help make these ``what's next'' steps concrete. 


\section{Wrapper}\label{sec:Wrapper}

\subsubsection*{Context} You are part of an active, long-running, and possibly quite complex project with more than a handful of participants.  

\subsubsection*{Problem} In an active project, it can be effectively impossible to stay up to date with all of the details.  Not everyone will be able to attend every meeting (see \patternname{Heartbeat}) or read every email.  Project participants can easily get lost and drift away.  The experience can be more difficult for \patternname{Newcomers}: joining an existing project can feel like trying to get aboard a rapidly moving vehicle.  If you've taken time off, you may feel like things have moved on so far that you cannot catch up.  Information overload is not the only concern: there is also problem with missing information.  If they aren't shared, key skills can quickly become bottlenecks (see \patternname{Carrying capacity}).

\subsubsection*{Solution}
% DK: Be more direct.  Don’t say what “can” be done…just say what to do. [also, typo -jc]
Someone involved with the project should regularly create a wrap-up summary, distinct from other project communications, that makes current activities comprehensible to people who may not have been following all of the details.  In addition, project members should keep other informative resources like the landing page, \patternname{Roadmap}, and documentation up to date.  Ensure that these resources accurately represent the facts on the ground, and that they really show interested parties how they can get involved.

\subsubsection*{Rationale}
According to the theory proposed by Yochai Benkler, for free/open ``commons-based'' projects to work, it is important for participants to the ability to contribute small pieces, and the project needs a way to stitch those pieces together \cite{coases-penguin}.  The wrapper helps perform this integrative stitching function.

\subsubsection*{Resolution} 
% DK: This sounds like Rationale
Regularly circulated summaries can help to engage or re-engage members of a project, and can give an emotional boost to peeragogues who see their contributions and concerns mentioned.  Well-maintained records chronicle the project's history, and up-to-date documentation makes the project more transparent, accessible, and robust.

\begin{framed}
\emph{What's Next.}
We have prototyped a visual ``dashboard'' that people can access to immediately get an idea of what work is ongoing in the project with links to ways to get further engaged (Figure \ref{dashboard}).  Let's deploy it.
\end{framed}    


\begin{figure}
\begin{tikzpicture}[every node/.style={anchor=south west,inner sep=0pt},x=1mm, y=1mm,]
     \node (fig1) at (0,0)
       {\includegraphics[width=\textwidth,trim=0mm 135mm 0mm 0mm,clip=true]{figures/peeragogy_dashboard_draft1/peeragogy_dashboard_draft1.jpg}};
     \node (fig2) at (55, 14)
       {\includegraphics[width=.45\textwidth,trim=0mm 0mm 0mm 0mm,clip=true]{figures/dashboard/dash-trans.png}};  
\end{tikzpicture}
\caption{Design for a Peeragogy project dashboard (design sketch by Amanda Lyons, prototype by Fabrizio Terzi; images used with permission).\label{dashboard}}
\end{figure}


\section{Heartbeat}

\paragraph{Context:}
People have a shared interest, and have connected with each other about it.

\paragraph{Problem:} What's an easy way for these people feel like there's a ``there, there?''

\paragraph{Solution:} People seem to naturally gravitate to regularly scheduled
activities. Once a week (meeting) or once a year (conference) are two common variants.  Sometimes people need a little extra prompt to join in.\footnote{In the ``Collaborative Lesson Planning'' course led
by Charlie Danoff at P2PU, Charlie wrote individual emails to people who
were signed up for the course and who had disappeared, or lurked but
didn't participate. This kept a healthy number of the people in the
group to re-engage and make positive contributions. In more recent
months, Charlotte Pierce has been running weekly meetings by Google
Hangout to coordinate work on the Peeragogy Handbook. Not only have we
gotten a lot of hands-on editorial work done this way, we've generated a
tremendous amount of new material (both text and video footage) that is
likely to find its way into future versions of the \emph{Peeragogy Handbook}.}

\paragraph{Rationale:}  This pattern might seem too obvious, since regularly scheduled meetings are so ubiquitous.  But there's an important difference between a mere meeting and a \emph{Heartbeat}: in short, if the energy from your meetings isn't helping you or your group thrive, something needs to change.

\paragraph{Resolution:} This pattern is one of the easiest to explain to \emph{Newcomers} to the idea of design patterns, since nearly everyone is familiar with the pattern of regular routine.  But the pattern is also a sophisticated tool: noticing when a new \emph{Heartbeat} occurs is a way to be aware of the priorities in the group, and may be a good source of new patterns.

\paragraph{What's Next:} When the project is bigger than more than just a few people, it's likely to have several \emph{Heartbeats}\footnote{We've operated two weekly meetings in the Peeragogy project at several times, for members with slightly different interests and slightly different availability.  Often this relates to small special-purpose projects, like our work on this paper.}  Identifying and fostering new \emph{Heartbeats} and new working groups is a task that can help make the community more robust.

\section{Creating a guide}\label{sec:Creating_a_guide}
\subsubsection*{Definition} Meaning-carrying tools, like handbooks or maps, can help collect content and stories as well as assist others who want to adopt the idea, .

\subsubsection*{Problem} 
Established ideas have knowledge cartography challenges for newcomers, consider trying to decipher a subway map in a foreign city. When the idea or system is only ``newly discovered'', the associated meanings may not be well understood, and indeed they may not have been created. Even if a topic is only ``personally new'', it can be hard to find one's way around.

\subsubsection*{Solution}
The process of creating the guide can go hand-in-hand with figuring out how the system works. Thus, techniques of \href{http://knowledgecartography.org/}{knowledge cartography} and \href{http://www.hitl.washington.edu/publications/r-97-47/two.html}{meaning making} are useful for would-be guide creators.\footnote{We started the Peeragogy project by collaboratively making an outline for the Peeragogy Handbook. We recommended this
handbook-making practice to others, as a way to learn collaboratively and build a strong group.}

\subsubsection*{Challenges} 
It is important to keep in mind how ``the map is not the territory,'' and map-making is only one facet of shared human activity. For instance, a pattern description can be thought of as a ``micro-map'' of a specific activity. These maps are not useful if they are divorced from practice.

\subsubsection*{What's Next} 
Working with our shepherd at PLoP to improve this paper!



\section{Newcomer}\label{sec:Newcomer}
\subsubsection*{Context}
A lot of ``education'' assumes we are speaking to a new generation. 
In learning more broadly, the ``audience'' is often new to the topic.
Sometimes we are the \patternname{Newcomers}, sometimes we're the oldtimers.

\subsubsection*{Problem} \patternname{Newcomers} can feel overwhelmed by the amount of things to learn.  They
don't know where to start.\footnote{Peeragogy Project participant
R\'egis Barondeau: ``I joined this handbook project late, making me
a `newcomer'. When I started to catch up, I rapidly faced doubts:
Where do I start? How can I help? How will I make it, having to read
more than 700 posts to catch up? What tools are we using ? How do I use
them? Etc. Although this project is amazingly interesting, catching the
train while it already reached high speed can be an extreme sport. By
taking care of newcomers, we might avoid losing valuable contributors
because they don't know how and where to start, and keep our own project
on track.''}  They may have a bunch of ideas that the oldtimers have
never considered -- or they may think they have new ideas, which are actually
a different take on old ideas; see \patternname{Use or Make}.

\subsubsection*{Solution} It is good to try to become aware of what a \patternname{Newcomer}
needs, and what their motivations are.\footnote{Peeragogy Project participant
Charlotte Pierce: ``Joe was working a lot on the book, and I thought
`this is interesting hard work, and he shouldn't have to do
this alone.' As a Peeragogy newcomer, I was kindly welcomed and
mentored by Joe, Howard, Fabrizio, and others. I asked naive questions
and was met with patient answers, guiding questions, and resource links.
Concurrently, I bootstrapped myself into a position to contribute to the
workflow by editing the live manuscript for consistency, style, and continuity.''}
\patternname{Newcomers} themselves may have only a general idea about what their goals are, so it can be
helpful to add concreteness with \patternname{A Specific Project}.

\subsubsection*{Rationale} \patternname{Newcomers} in the Peeragogy project have often complained
about feeling confused about what the project is about, suggesting that our \patternname{Roadmap}
has not been sufficiently clear.  Some feel it is too theoretical, which suggests
we need to do more work on \patternname{Creating a guide} on ways to get involved, while also
making it clear that we do not have an exhaustive list in mind.  New ideas can prompt us to consider how we may have been limiting ourselves.\footnote{Dilrukshi Gamage, Julia Echeverria, and Federico Monaco first joined a Peeragogy hangout in February, 2015.  They were all interested in the idea of designing and running a course on peer learning.  Although we had done some work on a syllabus, and considered the notion of using peeragogical models in formal education, we hadn't tried running a course on the topic of peeragogy.  There were a number of earlier experiments that \emph{used} ideas from peeragogy inside of a formal course, and the difference between these two approaches prompted interesting discussions.}

\subsubsection*{Resolution}
The frustration and confusion felt by a \patternname{Newcomer} familiar to anyone who is starting something new.  An awareness of how to help \patternname{Newcomers} can help us be more compassionate to ourselves and others.

\subsubsection*{What's Next} A more detailed (but non-limiting) ``How to Get Involved'' walk-through in text or video form would be good to develop. We can start by listing some of the things we're learning about.\footnote{Business issues relevant to the Peeragogy project, how to run a MOOC, hot-syncing our website from Git, etc.}

\section{Pattern Audit Routine}\label{sec:Pattern_Audit_Routine}

\subsubsection*{Context} As a collection of patterns grows it is important to "prune" them to make sure they are up to date and do not lose relevance.

\subsubsection*{Problem} This becomes confusing. Not all of the patterns are equally relevant and some will become completely irrelevant as a project evolves.

\subsubsection*{Solution} Periodically run a pattern audit. Bring in a real-time aspect by using the following five-part ``\href{http://metameso.org/~joe/docs/The-Paragogical-Action-Review.pdf}{Paragogical Action Review}'' \cite[Chapter 28]{peeragogy-handbook}:

\begin{enumerate}
\item Review what was supposed to happen.
\item Establish what is happening/happened.
\item Determine what’s right and wrong with what we are doing/have done.
\item What did we learn or change?
\item What else should we change going forward?
\end{enumerate}

After asking these five questions with respect to progress made with any pattern, the pattern will likely become clearer and/or show its irrelevance.  After a thorough review, any patterns that cannot be revised to become relevant for our current interests can be moved to the \patternname{Scrapbook}.

\subsubsection*{Rationale} We want to keep the attention focused on the most relevant issues.

\subsubsection*{Resolution} Regular evaluation helps us improve the pattern catalog and describe our effort to focus.

\subsubsection*{What's Next} Regularly go through the \patternname{Pattern Audit Routine} to check our patterns (and especially their next steps) in our future meetings.

\section{Scrapbook} \label{sec:Scrapbook}

\subsubsection*{Motivation} This pattern describes a way to make the project meaningful.  

\subsubsection*{Context} We have been working together for a while now.
We have maintained and revised our pattern catalog, and we are
achieving some of the ``What's Next'' steps associated with some of
the patterns.

\subsubsection*{Forces}~
\parbox[t]{.85\textwidth}{
\textbf{Attention}: due to limited energy, we need to ask: where should we set the focus?\\
\textbf{Interest}: new ideas catch our attention\\
\textbf{Meaning}: a history of working on things makes them meaningful.
}

\subsubsection*{Problem} Not all of the ideas we've come up with have proved workable.
Not all of the patterns we've noticed remain equally relevant.
In particular, some patterns no longer lead to concrete next steps.

\subsubsection*{Solution}
In order to maintain focus, is important to ``tune'' and ``prune'' the
things we give our attention to.  We can connect this understanding to
any actions undertaken in the project by asking questions like these:
%%%%%%%%%%%%%%%%%%%%%%%%%%%%%%%%%%%%%%%%%%%%%%%%%%%%%%%%%%%%%%%%%%%%%%%%%%%%%%%%%%%%%%%%%%%%%%%%%%%%
\begin{quote}
(1) Review what was supposed to happen.
(2) Establish what is happening/happened.
(3) Determine what’s right and wrong with what we are doing/have done.
(4) What did we learn or change? 
(5) What else should we change going forward?  \cite[Chapter 28]{peeragogy-handbook}.
\end{quote}
%
%OSS: Who maintains the scrapbook? ... People say when you're learning, you should retain a learning log. Maybe scrapbook is like a shared notebook? All the process should be shared together, even if people take different paths, its all open. Journal of activities?
As current priorities become clearer, we decide where to focus.
Anything that isn't receiving active attention should be moved to a
\patternname{Scrapbook}.  This may encompass:
\begin{itemize}
\item \emph{Retired patterns} whose ``next steps'' have either been
  achieved, set aside, or refactored;
\item \emph{Proto-patterns} made of problems, issues, and concerns
  that have yet to generate concrete next steps;
\item \emph{A back-catalog} of publications, reports, or other
  artifacts that create an institutional memory.
\end{itemize}
In the Peeragogy project, we were initially maintaining a collection
of antipatterns (like `\patternnameext{Magical thinking}') but the
next steps coming from these seemed particularly convoluted and
abstract.  Accordingly, we archived the antipatterns for the time
being.\footnote{\url{http://paragogy.net/Scrapbook}.}  We present a
list of outstanding problems -- without known solutions -- right up
front in the Introduction to the \emph{Peeragogy Handbook}
\cite[Chapter 1]{peeragogy-handbook}.  Our back-catalog includes
academic papers
\cite{building-peeragogy-accelerator,corneli2013inaction,corneli2012paragogical,paragogy-okcon}
and a thesis \cite{corneli-thesis}, earlier editions of the handbook,
as well as informal reports.
%
You don't need to limit
yourself to \emph{your own} creativity: you can include interesting
patterns and ideas from other sources (see \patternname{Reduce, reuse,
recycle}). Ideally many people will contribute by describing their
ideas and concerns, but in some cases a designated
\patternname{Wrapper} may have to do further work to elicit and
organize that material.

\subsubsection*{Rationale} 
We want our pattern catalog to be concretely useful and actively used,
and to keep attention focused on the most relevant issues.
If a pattern is not specifically useful or actionable at the
moment, sufficient time for reflection may offer a better
understanding, or it may prove useful in a different context.

\subsubsection*{Resolution} 
Judicious use of the \patternname{Scrapbook} can help focus project participants' \textbf{attention} on current concerns, without losing grasp of items of \textbf{interest}.  The currently active pattern catalog is leaner and more action-oriented as a result. If the \patternname{Roadmap} shows where we're going, it is the \patternname{Scrapbook} that shows most clearly where we've been, and collects the observations that are most \textbf{meaningful} to us.

\begin{wrapfigure}{r}{.52\textwidth}
\vspace{-1.4cm}
\begin{center}
\includegraphics[width=.5\textwidth,trim=0 200 0 0, clip=true]{ChristsPieces}
\end{center}
\vspace{-.5cm}
\caption{Christ's Pieces, Cambridge, UK.
% Public domain.
\label{christs-pieces}}
\vspace{-1.7cm}
\end{wrapfigure}

\subsubsection*{Example 1} The history of the Wikimedia Foundation,
and of Wiki\-pedia, are maintained as wiki
pages.\footnote{\url{https://wikimediafoundation.org/wiki/History_of_the_Wikimedia_Foundation}}\textsuperscript{,}\footnote{\url{https://en.wikipedia.org/wiki/Wikipedia}}
There is also a page on Wikipedia detailing
critiques.\footnote{\url{https://en.wikipedia.org/wiki/Criticism_of_Wikipedia}}
Five years on, the previous ``five year plan'' somewhat resembles a \patternname{Scrapbook} \cite{wikimedia2011plan}.  

\subsubsection*{Example 2} 
In the future university, the patterns described here will continue to
shape the landscape, but considerable activity will be focused on new
problems and new patterns -- just as a university campus grows and
changes in an emergent fashion over time.


\begin{framed}
\noindent 
\emph{What's Next in the Peeragogy Project}
\definecollection{ScrapbookWN}
\begin{collectinmacro}{\ScrapbookWN}{}{}
After pruning back our pattern catalog, we want it to grow again: new patterns are needed.
One strategy would be to ``patternize'' the rest of the \emph{Peeragogy Handbook.}
\end{collectinmacro}
\ScrapbookWN
\end{framed}


\newpage

\section{Emergent Roadmap} \label{sec:Distributed_Roadmap}

Table \ref{tab:WhatsNextSummary} reprises the ``What's Next'' steps from all of the previous
patterns, offering another view on the Peeragogy project's
\patternname{Roadmap} in a concrete emergent form.

%% \subsubsection*{\hyperref[sec:Peeragogy]{Peeragogy}} 
%% \PeeragogyWN

%% \subsubsection*{\hyperref[sec:Roadmap]{Roadmap}} 
%% \RoadmapWN

%% \subsubsection*{\hyperref[sec:Reduce, reuse, recycle]{Reduce, reuse, recycle}}
%% \ReduceWN

%% \subsubsection*{\hyperref[sec:Carrying capacity]{Carrying capacity}} 
%% \CarryingWN

%% \subsubsection*{\hyperref[sec:A specific project]{A specific project}}
%% \SpecificWN

%% \subsubsection*{\hyperref[sec:Wrapper]{Wrapper}}
%% \WrapperWN

%% \subsubsection*{\hyperref[sec:Heartbeat]{Heartbeat}}
%% \HeartbeatWN

%% \subsubsection*{\hyperref[sec:Newcomer]{Newcomer}}
%% \NewcomerWN

%% \subsubsection*{\hyperref[sec:Scrapbook]{Scrapbook}} 
%% \ScrapbookWN

\begin{table}
{\footnotesize
\begin{tabular}{|p{\textwidth}|}
\hline
\rowcolor{Gray!30} \multicolumn{1}{|l|}{\color{Black} \ref{sec:Peeragogy}. \patternname{Peeragogy}}\\
\hline
\vspace{-.5em}
\PeeragogyWN\\
\hline 
%%%%%%%%%%%%%%%%%%%%
\rowcolor{Gray!30} \multicolumn{1}{|l|}{\color{Black} \ref{sec:Roadmap}. \patternname{Roadmap}}\\
\hline
\vspace{-.5em}
\RoadmapWN
\\[.1cm]
\hline
%%%%%%%%%%%%%%%%%%%%
\rowcolor{Gray!30} \multicolumn{1}{|l|}{\color{Black} \ref{sec:Reduce, reuse, recycle}. \patternname{Reduce, reuse, recycle}}\\
\hline
\vspace{-.5em}
\ReduceWN
\\[.1cm]
\hline
%%%%%%%%%%%%%%%%%%%%
\rowcolor{Gray!30} \multicolumn{1}{|l|}{\color{Black} \ref{sec:Carrying capacity}. \patternname{Carrying capacity}}\\
\hline
\vspace{-.5em}
\CarryingWN
\\[.1cm]
\hline
%%%%%%%%%%%%%%%%%%%%
\rowcolor{Gray!30} \multicolumn{1}{|l|}{\color{Black} \ref{sec:A specific project}. \patternname{A specific project}}\\
\hline
\vspace{-.5em}
\SpecificWN
\\[.1cm]
\hline
%%%%%%%%%%%%%%%%%%%%
\rowcolor{Gray!30} \multicolumn{1}{|l|}{\color{Black} \ref{sec:Wrapper}. \patternname{Wrapper}}\\
\hline
\vspace{-.5em}
\WrapperWN
\\[.1cm]
\hline
%%%%%%%%%%%%%%%%%%%%
\rowcolor{Gray!30} \multicolumn{1}{|l|}{\color{Black} \ref{sec:Heartbeat}. \patternname{Heartbeat}}\\
\hline
\vspace{-.5em}
\HeartbeatWN
\\[.1cm]
\hline
%%%%%%%%%%%%%%%%%%%%
\rowcolor{Gray!30} \multicolumn{1}{|l|}{\color{Black} \ref{sec:Newcomer}. \patternname{Newcomer}}\\
\hline
\vspace{-.5em}
\NewcomerWN
\\[.1cm]
\hline
%%%%%%%%%%%%%%%%%%%%
\rowcolor{Gray!30} \multicolumn{1}{|l|}{\color{Black} \ref{sec:Scrapbook}. \patternname{Scrapbook}}\\
\hline
\vspace{-.5em}
\ScrapbookWN
\\[.1cm]
\hline

\end{tabular}
}
\caption{What's next for the Peeragogy project\label{tab:WhatsNextSummary}}
\end{table}



\bibliographystyle{acmlarge}
\bibliography{peeragogy-bib}

\end{document}
