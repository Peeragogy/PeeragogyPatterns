\begingroup \color{OliveGreen}

\section{A specific project}\label{sec:A_specific_project}
\subsubsection*{Context}
%DK This seems like a problem…I would think that the context would be something like "there is an existing shared project with a lot of parts/scope, etc."
We often find ourselves confronted with what seems to be a difficult, complex, or even insurmountable problem.  It won't go away, but a workable solution does not present itself either.  Perhaps there is \emph{some} candidate solution, but there are nowhere near enough resources to make it feasible.  

\subsubsection*{Problem}
We are often blinded by our own prejudices and preferences.  Considerable energy goes into pondering, discussing, exploring and feeling stuck.  Meanwhile there may be a strong urge to make more concrete progress, and time is passing by.  In a group setting, when the forward-movers try to act, those who are more wrapped up the experience may attempt to shut them down as they feel that they are being left behind.  However, if moves are being made at random, relative inaction may be the only safe choice.

\subsubsection*{Solution} 
One of the best ways to start to make concrete progress on a hard problem is to ask for help.   Formulating a question helps your thinking become more specific and concrete.  Sometimes you'll see that a solution was within your grasp all along, and you don't actually need to ask the question anymore.  In the case of a really difficult problem, one question won't be enough, but you can repeat the process: turning something that is too large or too ephemeral to tackle directly into a collection of smaller, specific, manageable projects that you can learn something from.  Maintain an overall project \patternname{Roadmap} to keep track of how the smaller pieces relate to the bigger picture.  If you have a fairly specific idea about what you want to do, but you're finding it difficult to get it done, it could be because the project's \patternname{Carrying Capacity} is larger than you thought: you're alone in a rich landscape.  Don't just ask for advice in this case: recruit material help.
% DK: This seems a bit speculative, “If you are able…” What steps does this solution entail? How do you know that a project is sufficiently specific? The front part of the pattern makes it sound like the point of view of the pattern is from someone coming into a project looking for something to do, but is there a piece of this pattern for people on a project with ideas about what to do but without the bandwidth to do it? I.e. someone populating the Roadmap with Specific Projects Is there some risk about being too specific about a project?

\subsubsection*{Rationale} 
In our culture, asking for help is often seen as an act of weakness rather than an act of intelligence.
But what we've seen time and again is that asking for help is a recipe for getting specific.
And getting specific is necessary for bringing about change.  Asking for help is one
of the best ways to gain coherence: mastering the challenge of making yourself understood can often go a long way
toward resolving deeper difficulties.  

\subsection{Resolution}
Where you may have felt stuck or realized you were going in circles, getting specific allows forward progress.  The struggle between consensus and action is resolved in a tangible project that combines action with dialog.  Learning something new is a strong sign that this is working.

\begin{framed}
\emph{What's Next.}
We need to build specific, tangible ``what's next'' steps. The \patternname{Pattern Audit Routine} can help make these ``what's next'' steps concrete. 
\end{framed}

\endgroup
    
    
  
  
  