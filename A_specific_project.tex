\section{A specific project}\label{sec:A specific project}
\subsubsection*{Context}
%DK This seems like a problem…I would think that the context would be something like "there is an existing shared project with a lot of parts/scope, etc."
We often find ourselves confronted with what seems to be a difficult, complex, or even insurmountable problem.  It won't go away, but a workable solution doesn't present itself, either.  If there is a candidate solution, it's also clear there are not enough resources for it to be feasible.
\textbf{In the face of serious difficulties we often find ourselves wringing our hands. ``New'' projects may just be preaching to the choir.  The real challenge is to make actionable plans leading to concrete change.}

\subsubsection*{Problem}
We are often blinded by our own prejudices and preferences.  Considerable energy goes into pondering, discussing, exploring and feeling stuck.  Meanwhile there may be a strong urge to make more concrete progress, and time is passing by.  In a group setting, when the forward-movers ultimately try to act, those who are more wrapped up in the experience of pondering and exploring may attempt to shut them down, if they feel that they are being left behind.  Inaction may seem like the only safe choice, but it has risks too.

\subsubsection*{Solution} 
One of the best ways to start to make concrete progress on a hard problem is to ask a specific question.   Formulating a question helps your thinking become more concrete.  Sometimes you'll see that a solution was within your grasp all along, and you don't actually need to ask the question to anyone anymore.  In the case of a truly difficult problem, one question won't be enough, but you can repeat the process: turning something that is too large or too ephemeral to tackle directly into a collection of smaller, specific, manageable tasks that you can learn something from. Maintain an overall project \patternname{Roadmap} to keep track of how the smaller pieces relate to the bigger picture.  If you have a fairly specific idea about what you want to do, but you're finding it difficult to get it done, don't just ask for advice: recruit material help (cf.~\patternname{Carrying capacity}).
% DK: This seems a bit speculative, “If you are able…” What steps does this solution entail? How do you know that a project is sufficiently specific? The front part of the pattern makes it sound like the point of view of the pattern is from someone coming into a project looking for something to do, but is there a piece of this pattern for people on a project with ideas about what to do but without the bandwidth to do it? I.e. someone populating the Roadmap with Specific Projects Is there some risk about being too specific about a project?

\subsubsection*{Rationale} 
We've seen time and again that asking specific questions is a recipe
for getting concrete, and that getting concrete is necessary for
bringing about change.  Asking for help (which is what happens
when you vocalize a question) is one of the best ways to
gain coherence.  Making yourself understood can go a long way
toward resolving deeper difficulties.

\subsubsection*{Resolution}
Where you may have felt stuck or realized you were going in circles, getting specific allows forward progress.  The struggle between consensus and action is resolved in a tangible project that combines action with dialog.  Learning something new is a strong sign that things are working.
%
Real change starts out ``bite-sized.'' If you want something to happen, get specific.

\subsubsection*{Example 1}
One of the best ways to jump in, get to know other Wikipedia users,
and start working on a focused todo list is to join (or start)
\patternname{A specific project}.  Within Wikipedia, these are known
as
``WikiProjects.''\footnote{\url{https://en.wikipedia.org/wiki/Wikipedia:WikiProject_Council/Directory}}\textsuperscript{,}\footnote{\url{https://en.wikipedia.org/wiki/Wikipedia:WikiProject_Council/Guide}}
There are many other public projects, for instance, the Wikipedia
Education
Program.\footnote{\url{https://outreach.wikimedia.org/wiki/Education/Wikipedia_Education_Collaborative/Tasks}}

\subsubsection*{Example 2}
Dormitories could be seen as an ``optional extra,'' since studying
from where you live is often an option already.  However,
cooperatively-owned living/working spaces may frequently be an asset
for \patternname{A specific project}.

% \subsubsection*{Summary}
% ``''

\begin{framed}
\noindent 
\emph{What's Next.}
We need to build specific, tangible ``what's next'' steps and connect them with concrete action. Use the \patternname{Scrapbook} to organize that process. 
\end{framed}

