\begingroup \color{BurntOrange}

\section{A specific project}\label{sec:A_specific_project}
\subsubsection*{Context}
You find yourself interested in or concerned about something, but you
only have a vague idea about how it works or how you fit in.

\subsubsection*{Problem}
It's easy to think about issues that matter: there are many of
them. The problem is figuring out what you're going to do about it.
As a further problem, getting concrete can be scary, because you risk
failure.\footnote{In the Peeragogy project and more broadly, we've
  observed that some people are happy with a sense of experience or
  process, while others want to see results. Some others are in the
  middle.  All of these variations are OK!  However, we are often
  blinded by our own preferences, and in the worst case this can
  undermine or destroy group dynamics.  At the very least it will add
  tension, as some want to continue to discuss and engage generally
  while others want to move forward.  When the forward-movers try to
  act, those enjoying the experience may attempt to shut them down or
  may feel that they are being left out/behind.}

\subsubsection*{Solution} 
If you \emph{are} able to get concrete about something to do, learn, and achieve, you move from thinking about a topic to becoming a practitioner.  You may realize that your ``specific project'' is too large to tackle directly. In this case, you will have to become even more specific.  Maintaining a project \patternname{Roadmap} can help keep track of the smaller pieces and the bigger picture.

\subsubsection*{Rationale} 
Being specific is important for bringing about to change.\footnote{In the January, 2013, plenary
session, \href{http://ipne.org}{Independent Publishers of New England}
(IPNE) President Tordis Isselhardt quietly listened to a presentation
about how we created the \emph{Peeragogy Handbook}. During the Q\&A, she
spoke up, wondering if peer-learning effort in IPNE might be more likely
to succeed if the organization's members ``focused around a specific
project.'' As this lightbulb illuminated the room, those of us attending
the plenary session suggested that IPNE could focus the project by
creating an ``Independent Publishing Handbook.'' (Applause!) In the
course of creating the IPNE Handbook, peer learners would assemble
resource repositories, exchange expertise, and collaboratively edit
documents. To provide motivation and incentive to participate in
``PeerPubU'', members of the association will earn authorship credit for
contributing articles, editor credit for working on the manuscript, and
can spin off their own chapters as stand-alone, profit-making
publications.} But while actions speak louder than words, it's important
to act in a coherent way if you want to be understood by others.  However, in
general it would be a mistake to try to seek consensus before acting: it's much better to combine action with dialog.

\subsubsection*{What's Next}  Each project connected with the Peeragogy Project should be described with one or more patterns, each with specific, tangible ``what's next'' steps. The \patternname{Pattern Audit Routine} can help make these ``what's next'' steps concrete. 

\endgroup