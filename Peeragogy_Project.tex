\section{Peeragogy}\label{sec:Peeragogy_Project}
% DK: I question whether the Peeragogy Project is a pattern, or an instance of a pattern. We generally think about patterns as something that can be instantiated. But your project seems like a specific case of the concept.

\subsubsection*{Context}  Architectual maverick Christopher Alexander asked the following questions to an audience of computer programmers \cite{alexander1999origins}: 
\begin{quote}
``What is the Chartres of programming? What task is at a high enough level to inspire people writing programs, to reach for the stars?''
\end{quote}
In order for humanity to pull itself up by its bootstraps, on this planet or any other, we need to continue to learn and adapt.  Computers can help accelerate this process.  In particular, collaborative projects like Wikipedia, StackExchange, and Free/Libre/Open Source Software (FLOSS) represent an implicit challenge to the old ``industrial'' organization of education, in which graduates are the product.  In the context of these free, open, post-modern organizations, not only are individual participants learning and growing, but they can contribute to adapting the methods and infrastructure as they go.

\subsubsection*{Problem} What is the real potential of this way of organizing the human life-world, and how can we best fulfill it?  Simply creating a free/open piece of software does not mean that other people will get involved: in fact, the majority of free software projects fail to attract users or developers.  Even a successful project like Wikipedia is a work in progress that can be adapted to \emph{better} empower and engage people around the world, to develop richer and more useful educational content, and to deploy it in more contexts.  These are difficult questions.  Pedagogy won't be of much use if we don't know the answers.

\subsubsection*{Solution} \emph{Peeragogy} is our word for the process whereby different people involved with different projects can describe material problems and share solutions and work in progress.  There are different ways to go about this -- bug reports, mailing lists, writers workshops -- but we have found that the process is particularly well-matched to Christopher Alexander's idea of \emph{design patterns}, in which a pattern can be described using a simple template, and refined until it becomes perfectly clear.  Indeed, thought of as a design pattern, \patternname{Peeragogy} is an up-to-date revision of Alexander's \patternname{Network of Learning} \cite[p. 99]{alexander1977pattern}.  It \emph{decentralizes the process of learning and enriches it through contact with many places and people} potentially all over the world.   The essence of peeragogy is to combine a free/open approach with reflection on what works.

\subsubsection*{Rationale}
% DK: I have written many, and shepherded many more, and I don’t think this is the case. However, the effort that goes into writing them makes them more intuitive to read :+)
Well-written design patterns is intuitive to read.  The process of finding and writing good design patterns is more involved, but we can be guided by an intuitive sense of what works.
% DK. I am not quite sure what you mean by this: We can use our design pattern catalog to scaffold our work on other parts of the project -- our technical platform, our Handbook, our meetings -- and to connect with others.  
Indeed, ``patterns'' can be shared in different ways, and finding better ways to share them is the important thing (\cite{meszaros1998pattern} describe common patterns of good design patterns).

\subsubsection*{Resolution}
% DK: I think this misses the point of the Resolution. How did Peeragogy resolve the forces that made the problem challenging? Are the participants better off for having used Peeragogy?
%% Writing down this pattern defines some of the key terms for \patternname{Newcomers}, and shows how to use the pattern template introduced schematically above.  Together with Figure \ref{fig:connections}, this pattern begins to show how we are affected by using patterns to build emergent structure.
Peeragogy helps people in different projects describe and solve real problems.  This can guide action in ways that we wouldn't have seen on our own.

\begin{mdframed}
\textbf{What's next}    
Feel free to join us and suggest new patterns and projects, or adapt our patterns to help shape your own \patternname{Peeragogy Project}.
% \footnote{In the present document, the term Peeragogy Project, written in a standard font, refers to a current historical, real-world, example of the \patternname{Peeragogy Project} pattern. This project has been active since it was convened by Howard Rheingold \cite{howard-rheingold-lecture}. In order to enhance the readability of our patterns for a general audience, \emph{examples} drawn from our experiences in the Peeragogy Project and related projects appear in footnotes, rather than the pattern template as such.}
\end{mdframed}
