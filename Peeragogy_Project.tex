\section{Peeragogy}\label{sec:Peeragogy}
% DK: I question whether the Peeragogy Project is a pattern, or an instance of a pattern. We generally think about patterns as something that can be instantiated. But your project seems like a specific case of the concept.

\subsubsection*{Context}  Architectual maverick Christopher Alexander asked the following questions to an audience of computer programmers \cite{alexander1999origins}: 
\begin{quote}
``What is the Chartres of programming? What task is at a high enough level to inspire people writing programs, to reach for the stars?''
\end{quote}
In order for humanity to pull itself up by its bootstraps, on this planet or any other, we need to continue to learn and adapt.  Collaborative projects like Wikipedia, StackExchange, and FLOSS represent an implicit challenge to the old ``industrial'' organization of work.  We see a more resilient, enjoyable, and exciting way to work.  In the context of these free, open, post-modern organizations, individual participants are learning and growing -- and adapting the methods and infrastructure as they go.  Collectively, they manifest a ``growth mindset'' \cite{dweck,ranciere}.
\textbf{There is a tension between the generality/inclusiveness of an ``open'' work and the specificity/concretion required in order to develop something really useful.  Trust is built through sharing and reciprocity.}

\subsubsection*{Problem} Even a highly successful project like Wikipedia is a work in progress that can be improved to \emph{\emph{better} empower and engage people around the world, to develop \emph{richer and more useful} educational content, and to disseminate it \emph{more} effectively} -- and deploy it more creatively.\footnote{\url{https://wikimediafoundation.org/wiki/Mission_statement}}  How to go about this is a difficult question, and we don't know the answers in advance.  There are rigorous challenges facing smaller projects as well, and fewer resources to draw on.  Many successful free software projects are not particularly collaborative -- and the largest projects are edited only a small minority of users \cite{free-software-better,who-writes-wikipedia}.  Can we work smarter together?

\subsubsection*{Solution} Peeragogy is our word for the process whereby different people involved with different projects can describe material problems, share practical solutions, and constructively critique works-in-progress.   There are many different ways to go about this -- bug reports, mailing lists, writers workshops, Q\&A forums, watercoolers and skateparks are all places where peeragogy can happen.  We have found that the ``reflection'' part of the process is particularly well-matched to Christopher Alexander's idea of a \emph{pattern language}, in which commonly occurring,  interconnected, elements of an optative design are refined until they can be described in terms of a simple template.  Indeed, thought of as a design pattern, \patternname{Peeragogy} can be understood as an up-to-date revision of Alexander's \patternnameext{Network of Learning} \cite[p. 99]{alexander1977pattern}.  It \emph{decentralizes the process of learning and enriches it through contact with many places and people} -- in interconnected networks that may reach all over the world.   Importantly, while people involved in a peeragogical process may be collaborating on \patternname{A specific project}, they don't have to be direct collaborators outside of the learning context or co-located in time or space.  Peeragogy often takes place in mostly-horizontal relationships between people who have different but compatible objectives. 

\subsubsection*{Rationale}
% DK: I have written many, and shepherded many more, and I don’t think this is the case. However, the effort that goes into writing them makes them more intuitive to read :+)
The peeragogical approach  particularly addresses the problems of small projects stuck in their individual silos, and large projects becoming overwhelmed by their own complexity.  It does this by going the opposite route: explicating \emph{what by definition is tacit} and employing \emph{a continuous design process} \cite[pp. 9--10]{schummer2014beyond}.  The very act of asking ``can we work smarter together?'' puts learning front and center.  \patternname{Peeragogy} takes that ``center'' and distributes it across a pool of heterogeneous relationships.  As pedagogy articulates the transmission of knowledege from teachers to students, peeragogy articulates the way peers produce and use knowledge together (Figure \ref{fig:connections}).

% \emph{It is the tacit performance that cannot be described but has to be experienced, which leads to quality aspects that cannot be named} \cite{schummer2014beyond}.
%  While we would be thrilled if readers got involved with the \emph{Peeragogy Handbook} (Figure \ref{fig:connections}), if we're honest the benefits come from putting peeragogy into action.

% Peeragogy weds Alexander's vision of a programmer's cathedral to Eric Raymond's vision of the corresponding open source bazaar \cite{raymond2001cathedral} -- and opens the doors to non-programmers.
% Our earlier draft: Well-written design patterns is intuitive to read.  The process of finding and writing good design patterns is more involved, but we can be guided by an intuitive sense of what works.
% DK. I am not quite sure what you mean by this: We can use our design pattern catalog to scaffold our work on other parts of the project -- our technical platform, our Handbook, our meetings -- and to connect with others.  

\subsubsection*{Resolution}
%% Our earlier draft: Writing down this pattern defines some of the key terms for \patternname{Newcomers}, and shows how to use the pattern template introduced schematically above.  Together with Figure \ref{fig:connections}, this pattern begins to show how we are affected by using patterns to build emergent structure.
% DK: I think this misses the point of the Resolution. How did Peeragogy resolve the forces that made the problem challenging? Are the participants better off for having used Peeragogy?
Peeragogy helps people in different projects describe and solve real problems. 
If you share the problems that you're experiencing in your project, someone may be able to help you solve them.
This process can guide individual action in ways that we wouldn't have seen on our own, and may lead to new forms of collective action we would never have imagined possible.
%
Our use of ``both/and'' thinking is one of the key methods for resolving the tension between generality and specificity.
Peeragogy is one of ``tens of thousands of projects in the traditions of world improvement \'elan -- without any central committee that would have to, or even could, tell the active what their next operations should be'' \cite[p. 402]{sloterdijk2013change}.  When we talk about ``next steps,'' we aim to clarify our own commitments.

\subsubsection*{Example 1} Wikimedia users are invited to contribute content, extensions to
software, and to get involved with governance and other ``meta''
duties.  We claim that this pluralistic approach is an example of
\patternname{Peeragogy}.  It achieves something impressive: the
Wikimedia Foundation runs the 7\textsuperscript{th} most popular
website in the world, and has around 230 employees.  For comparison,
the 6\textsuperscript{th} and 8\textsuperscript{th} most popular
websites are run by companies with 150K and 30K employees,
respectively.

\subsubsection*{Example 2} Although one of the strengths of \patternname{Peeragogy} is to
distribute the workload, this does not mean that infrastructure is
irrelevant.  No less than their predecessors, the students and
researchers of the future university will need access to an
observatory and other scientific apparatus if they are truly to reach \emph{ad astra, per aspera}.

\begin{framed}
\noindent 
\emph{What's next.} We intend to revise and extend the patterns and methods of peeragogy to make it a workable model for education.
\end{framed}

% People won't get invested without a return, although this may not be the same for everyone.
% , and the idea of a specific goal or something concrete on offer


  

  
  
  
  
  