\section{Roadmap} \label{sec:Roadmap}

\subsubsection*{Context} It is very useful to have an up-to-date public roadmap for the project in a place where it can be discussed and maintained. The roadmap exists as an artifact with which to share current, but never complete, understanding of the space.

\subsubsection*{Problem} Without a roadmap, there will not be a shared sense of the project's goals or working methods. It will be much harder for people to volunteer to help out, or to assess the project's progress.  As everyone has time limitations, the project has to offer enough of what the participant seeks to keep it on their calendars.  Because everyone comes at it with different interests, expertise  and motivations, the process has to be adaptive to circumstances.

\subsubsection*{Solution} Keeping a list of current and upcoming activities, as well as goals and working methods can help \patternname{\href{http://peeragogy.org/practice/heuristics/newcomer/}{Newcomers}} and old-timers alike see where they can jump in. As we cross things off the list, this gives a sense of the accomplishments to date, and any major challenges that lie ahead.  At the same time, this is a project witht emergen structure, and so the \patternname{Roadmap} needs to be emergent, not simply prescriptive.  (See Section \ref{sec:Distributed_Roadmap} for our ``distributed roadmap.'')

\subsubsection*{Rationale} Unless the roadmap is easy for people to see and to update, they are not likely to use it.  If they manage to get involved, they will fly off in their own direction; see \patternname{Use or Make}.\footnote{In the Peeragogy project, once the handbook's outline became fairly mature, we used it as a roadmap, by marking the sections that are ``finished'', marking the sections where editing is currently taking place, and marking the stubs (possible starting points for future contributors). After this outline matured into a real \href{http://peeragogy.org/table-of-contents/}{table of contents}, we started to look in other directions for things to work on, and created a \href{http://peeragogy.org/peeragogy-org-roadmap/}{roadmap for further development of the website and peeragogy project as a whole}.  Unlike the earlier outline, this one was relatively top down and did not have strong buy-in from contributors.  Our thinking about roadmaps continued to evolve \cite{corneli2013roadmaps}, and in the Peeragogy Accelerator phase of the project \cite{building-peeragogy-accelerator}, we included a roadmap in the ``behind the scenes'' version of our landing page, we used it as a way to link to other documents we were working on.  This shared ``\href{https://docs.google.com/document/d/1RZEsqFDwF-jPiCvgWzJgi6n6faTRTDuPQS1CMEeXxRE/edit\#heading=h.p197njr3jsn8}{bulletin board}'' listing upcoming opportunities and completed tasks was much more active than the top-down roadmap on peeragogy.org.}  But frequently it's impossible to know in advance what will happen! A \patternname{Roadmap} that's not quite right will feel burdening. Sometimes it's better to be more open to the unknown.

\subsubsection*{Resolution}
A project can survive without a regularly updated \patternname{Roadmap}, but progress will be harder to come by and deadlines will be more likely to be forgotten.  Over the past years of the project our various roadmaps have helped guide and organize our work across continents, time zones and different day job work schedules.  But only now do we have a robust mechanism in place for building and maintaining a ``distributed roadmap.''

\subsubsection*{What's Next} 
Adding ``What's Next'' steps to our patterns gives us a ``distributed roadmap.''  And this works both ways: If we sense that something needs to change about the project, that is a clue that we might need to record a new pattern.
