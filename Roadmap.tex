\section{Roadmap} \label{sec:Roadmap}

% DK: This is a bit self-referential. You have roadmap embedded throughout the description of the pattern. E.g. the context and problem should be able to describe the situation before the solution has been applied, so you should be able to describe them without the solution name in them.

% DK: How is this different from a Backlog? Who “owns” the roadmap? How are changes made to it? How precise is it? Does it have a time dimension to it? You refer to deadlines later on. Does the roadmap include them? (What do deadlines really mean in a project like this anyhow?) Priority?

\subsubsection*{Motivation} This pattern describes the main ``design object'' that is of interest in peeragogy: the group's communication about their work-in-progress to address their shared goals.  This is the central pattern in our pattern language. 


\subsubsection*{Context} \patternname{Peeragogy} has both distributed and centralized aspects. The discussants or contributors who collaborate on a project have different points of view and heterogeneous priorities, but they come together in conversations and joint activities.

\subsubsection*{Forces}~
\begin{tabular}[t]{p{.8\textwidth}@{\hspace{.03\textwidth}}c}
\textbf{Variety}: people have different goals and interests in mind. & {\icon \symbol{"002127}}\\
\textbf{Clarity}: some may be quite specific, and some rather vague. & {\icon \symbol{"0021A6}} \\
\textbf{Coherence}: some of these goals will be well-aligned, others less so. & {\icon \symbol{"0021A9}} 
\\
\end{tabular}

\subsubsection*{Problem} In order to collaborate, people need a way to share current, though incomplete, understanding of the space they are working in, and to nurture relationships with one another and the other elements of this space.  At the outset, there may not even be a coherent vision for a project -- but a only loose collection of motivations and sentiments.  Once the project is up and running, people are likely to pull in different directions.   

\subsubsection*{Solution}  Building a guide to the goals, activities, experiments and working methods can help \patternnameplural{Newcomer} and old-timers alike understand how the nature of their relationship with the project.  %% This guide may be a research question or an outline, an organizational mission statement, or a business plan.
It may combine features of a manifesto, a syllabus, and an issue tracker.  It may be a design pattern or a pattern language \cite{kohls2010structure}.  The distinguishing qualities of a project \patternname{Roadmap} are that it should be adaptive to circumstances, and that it should ultimately get us from \emph{here} to \emph{there}.  By this same token, any given version of the roadmap is seen as fallible.  %% Everyone with an interest in the project should have the right to update it, although in practice not everyone will choose to do so.
In lieu of widespread participation, the project's \patternname{Wrapper} should attempt to synthesize an accurate roadmap that is informed by participants' behavior, and should help moderate in case of conflict.  Nevertheless, full consensus is not necessary: different goals, with different \emph{heres} and \emph{theres}, can be pursued separately, while maintaining communication.

\subsubsection*{Rationale} 
% DK: This seems more like advice about how to implement the solution than it is an explanation of how the solution addresses the forces from the context/problem [jc: fixed]
In the Peeragogy project our initial roadmap was an outline of the
first draft of the \emph{Peeragogy Handbook}.  Later, it took the form
of a schedule of meetings following a regular
``\patternname{Heartbeat}'' supplemented by a list of upcoming
submission deadlines.  Most recently, it is expressed in the emergent
objectives listed in Section \ref{sec:Distributed_Roadmap} of the
current paper.  We have seen that a list of nice-to-have features created
in a top-down fashion is comparatively unlikely to \emph{go} anywhere.
A backlog of tasks and a realistic plan for accomplishing them are
vastly different things.  An adaptive roadmap is an
antidote to \patternnameext{Tunnel Vision}
\cite[pp. 121--124]{david2001software}. 

\subsubsection*{Resolution}
% DK: This seems like Rationale to me [jc: fixed]
An emergent roadmap is rooted in real problems and justifiable
solutions-in-progress in all their \textbf{variety} and communicates
both resolution and followthrough.  The process of meshing varied
issues with one another requires thought and discussion, and this
encourages \textbf{clarity}.  The test of \textbf{coherence} is that
contributed goals and ideas should be actionable.
%
One quality-control test for the roadmap as a whole is that it should
give a \patternnameplural{Newcomer} a reasonable idea of what it would
mean to participate in the project, and help them decide whether,
where, and how to get involved.

%% \subsubsection*{Inversion}
%% The \patternname{Roadmap} is something of a paradox.  We can't dictate
%% the behavior of other participants, and we often can't even guess
%% ourselves what's coming up.  A peeragogical \patternname{Roadmap}
%% should prepare people for the \emph{absence} of clear step-by-step
%% direction, the \emph{presence} of different view points and
%% priorities, and the consequent need to be relatively self-directed.  This pattern
%% isn't particularly suitable for a project that needs to be managed in
%% a top-down fashion, and that can rely on other coordination tools
%% (like contracts) to manage work.

\subsubsection*{Example 1}  The \emph{Help} link present on every Wikipedia page could be seen as a
localized \patternname{Roadmap} for individual user
engagement.\footnote{\url{https://en.wikipedia.org/wiki/Help:Contents}}
%% It tells users what they can do on the site, and gives instructions
%% about how to do it:
%% \begin{quotation}
%% \noindent 
%% I want to \emph{read} or \emph{find} an article \ldots; 
%% I want to \emph{edit} an article \ldots;
%% I want to \emph{report a problem} with an article \ldots;
%% I want to \emph{create a new article} or \emph{upload media} \ldots;
%% I have a \emph{factual question}\ldots
%% ~[Etc.]
%% \end{quotation}
For someone who is prepared to jump in and get to work, there are
around 30 pages listing articles with various kinds of problems, for
example articles tagged with style issues, or ``orphaned'' articles
(i.e., articles with no links from other pages in the
encyclopedia).\footnote{\url{https://en.wikipedia.org/wiki/Category:Wikipedia_article_cleanup}}\textsuperscript{,}\footnote{\url{https://en.wikipedia.org/wiki/Category:Wikipedia_articles_with_style_issues}}\textsuperscript{,}\footnote{\url{https://en.wikipedia.org/wiki/Category:All_orphaned_articles}}
%
%
Wikimedia previously developed
a detailed strategic plan drawing on community input
\cite{wikimedia2011plan}.  In 2015, a two-week 
Community Consultation was carried out and
synthesized, resulting in ``a direction that will guide the decisions for the organization.''\footnote{\url{https://blog.wikimedia.org/2015/02/23/strategy-consultation/}}\textsuperscript{,}\footnote{\url{https://blog.wikimedia.org/2015/08/27/strategy-potential-mobile-multimedia-translation/}}
%
%
 Community-organized WikiProjects often invite outside involvement on \patternname{A
  specific project}.

\begin{wrapfigure}{r}{.48\textwidth}
\vspace{-2.05cm}
\begin{center}
\includegraphics[width=.46\textwidth,trim=140 30 30 30, clip=true]{alabama-big}
\end{center}
\vspace{-.5cm}
\caption{President's Home, University of Alabama.
%Public domain.
\label{presidents-home}}
\vspace{-1.9cm}
\end{wrapfigure}

\subsubsection*{Example 2}
In a future university run in a peer produced manner, a fancy
President's Residence isn't likely be needed.  Leadership would be
carried out in a more collaborative and distributed manner.  It may be
appropriate for project facilitators to meet together at a University
Hall for the primary purpose of working together on the university's
\patternname{Roadmap}.


\smallskip
\bigskip

% \noindent \begin{minipage}{.45\textwidth}
\begin{framed}
\noindent
\emph{What's Next in the Peeragogy Project}
\definecollection{RoadmapWN}
\begin{collectinmacro}{\RoadmapWN}{}{}
If we sense that something needs to change about the project, that is a clue that we might need to record a new pattern, or revise our existing patterns.
\end{collectinmacro}
\RoadmapWN
\end{framed}
%\end{minipage}

\newpage
