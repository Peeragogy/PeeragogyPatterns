\section{Roadmap} \label{sec:Roadmap}

% DK: This is a bit self-referential. You have roadmap embedded throughout the description of the pattern. E.g. the context and problem should be able to describe the situation before the solution has been applied, so you should be able to describe them without the solution name in them.

% DK: How is this different from a Backlog? Who “owns” the roadmap? How are changes made to it? How precise is it? Does it have a time dimension to it? You refer to deadlines later on. Does the roadmap include them? (What do deadlines really mean in a project like this anyhow?) Priority?

\subsubsection*{Context} \patternname{Peeragogy} has both distributed and centralized aspects. The different discussants or contributors who collaborate on a project have different points of view and heterogeneous priorities which they pursue in conversations and joint activities.

\subsubsection*{Problem} In order to collaborate, people need a way to share current, though incomplete, understanding of the space they are working in, and to nurture relationships with one another and the other elements of this space.  Without a sense of our individual goals or how they fit together in the context of addressing outstanding problems, it is difficult for people to help out, or to assess the 
project's progress.  Particularly if the project has big ambitions, lack of clarity is a serious problem.

\subsubsection*{Solution} Building a guide to current and upcoming activities, as well as goals and working methods can help \patternname{Newcomers} and old-timers alike see where they can jump in.  This guide can take various forms, and can have different levels of detail: it may be a research question or an outline for a draft document, an organizational mission statement or a business plan, a todo list or a backlog of issues in an issue tracker, a course syllabus, a manifesto, a calendar of upcoming events, or some combination of these.  The distinguishing features of a project \patternname{Roadmap} are that it should be adaptive to circumstances and that it should ultimately get us from \emph{here} to \emph{there}.  The roadmap should be accessible to everyone with an interest in the project, though in practice not everyone will choose to update it.  In lieu of widespread participation, the project's \patternname{Wrapper} should attempt to synthesize an accurate roadmap, and should help moderate in case of conflict.

\subsubsection*{Rationale} Unless the project's plan is easy for people to see and to update, they are not likely to use it, and are less likely to get involved.  The key point of the roadmap is to help support involvement by those who \emph{are} involved.   The level of detail in the roadmap (and the existence of a roadmap at all) should correspond to the felt need for sharing information and to the tolerance of uncertainty among participants.
% DK: This seems more like advice about how to implement the solution than it is an explanation of how the solution addresses the forces from the context/problem [jc: fixed]
The structure of the roadmap can shift along with its contents: it is an antidote to \patternname{Tunnel Vision} \cite[pp. 121--124]{david2001software}. 
In the Peeragogy project our roadmap evolved from an outline of the first draft of the
\emph{Peeragogy Handbook}, to a schedule of meetings with a regular
``\patternname{Heartbeat}'' supplemented by a list of upcoming submission deadlines, to the emergent list of objectives in Section \ref{sec:Distributed_Roadmap}.
By contrast, we've seen that a list of nice-to-have features is comparatively
unlikely to \emph{go} anywhere.  A backlog of tasks and a realistic
plan for accomplishing them can be vastly different things.

\subsubsection*{Resolution}
% DK: This seems like Rationale to me [jc: fixed]
Using the pattern catalog as an
organizational tool gives us a robust mechanism for
building and maintaining a ``distributed'', and ultimately
``emergent'' roadmap -- whose components are rooted in real problems
and justifiable solutions, with a concrete resolution and
followthrough.  When these components are put together, we get a
reasonably coherent and actionable idea of where the project is going.

\begin{framed}
\emph{What's Next.}
If we sense that something needs to change about the project, that is a clue that we might need to record a new pattern.
\end{framed}


    
    
    
    
    
    
    
    
    
