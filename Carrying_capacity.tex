

\section{Carrying capacity}\label{sec:Carrying capacity}
% DK: There is how much you can do….and there is how much you can do at once :+)
\subsubsection*{Context} There's only so much any one person can do with
limited resources and a limited amount of time.  Moreover, in a
peeragogy context, participants are likely to be unpaid, working on
the project in very limited spare time. A ``successful'' project may have many
users or consumers who do not contribute much (or at least not very much that is immediately visible).

%DK: To some degree this is a red herring. For virtually all FLOSS projects, there are a small number of contributors and a wide group of users. If that makes you uncomfortable as a contributor, you will never be comfortable.
\subsubsection*{Problem} These universal considerations can
be approached more or less gracefully.  There are two mistakes that
Linus Torvalds highlights in relation to FLOSS: assuming that others
will do the work for you, and assuming that the code matters more than
the people who use it \cite{torvalds-interview}.  Even if both of these mistakes are avoided,
there is a slushy middle ground in which a given contributor may feel
that others are under-contributing, that whatever he or she is
contributing isn't producing the desired result, or that someone else
is railroading an idea or dominating the discussion.

\subsubsection*{Solution} If something like this comes up (which is almost inevitable),
take a step back and observe the dynamics of involvement.  Symptoms of
burnout are a sign that it's time to revisit the group's
\patternname{Roadmap}.  See if you can clarify to others the concrete
goal that you're working towards -- and remember, other participants
don't have to share the same goal.  This is also a good time to ask
questions and let others answer.  Do they see things the same way you do?
Think of the project as an ecosystem populated by acts of participation.
If you're not thriving, it might be time to migrate some of your energy
elsewhere -- or shift more of it into system renovation.

% DK: This seems to be argument by negation. Can you make a positive argument for the rationale?
\subsubsection*{Rationale}  One of the things we're aiming to learn in
peeragogy is what it means to ``work smart.''  Putting effort into creating
a thriving ecosystem rather than a comparatively arid monoculture is likely to
aid mutual understanding of much deeper problems.

\subsubsection*{Resolution}
This pattern serves as a gentle reminder to anyone who believes him- or
herself to be caring more about project outcomes than other
participants, that, in fact, outcomes do not affect everyone the same way.
Cultivating a healthy co-working environment will lead to positive relationships
with others who have a different outlook, and different, but still compatible goals.

\begin{framed}
\emph{What's Next.}
Making it easy
and fruitful for others to get involved is one of the best ways to
redistribute the load (compare the \patternname{Newcomer} pattern).
\end{framed}
