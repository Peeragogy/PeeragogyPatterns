\begingroup \color{OliveGreen}

\section{Carrying capacity}\label{sec:Carrying_capacity}
% DK: There is how much you can do….and there is how much you can do at once :+)
\subsubsection*{Context} There's only so much any one person can do with
limited resources and a limited amount of time.  Moreover, in a
peeragogy context, participants are likely to be unpaid, working on
the project in very limited spare time; there may be many
``consumers'' who do not actively contribute at all.


%DK: To some degree this is a red herring. For virtually all FLOSS projects, there are a small number of contributors and a wide group of users. If that makes you uncomfortable as a contributor, you will never be comfortable.
\subsubsection*{Problem} These are universal considerations can
be approached more or less gracefully.  There are two mistakes that
Linus Torvalds highlights in relation to FLOSS: assuming that others
will do the work for you, and assuming that the code matters more than
the people who use it \cite{torvalds-interview}.  Even if both of these mistakes are avoided,
there is a slushy middle ground in which a given contributor may feel
that others are under-contributing, that whatever he or she is
contributing isn't producing the desired result, or that someone else
is railroading an idea or dominating the discussion.

\subsubsection*{Solution} If something like this comes up (and it inevitably will),
take a step back and observe the dynamics of involvement.  Symptoms of
burnout are a sign that it's time to revisit the group's
\patternname{Roadmap}.  See if you can clarify to others the concrete
goal that you're working towards -- and remember, other participants
don't have to share the same goal.  This is also a good time to ask
questions and let others answer.  Think of the project as an
ecosystem: acts of participation belong to the various species
residing there.  Is there a way to make your participation more
sustainable and fruitful?  If you feel you're not thriving, it might
be time to migrate elsewhere -- which presumably wouldn't be the end
ef the world.  Alternatively, with some work, you may be able to
actively cultivate the kind of environment you need to thrive here.

% DK: This seems to be argument by negation. Can you make a positive argument for the rationale?
\subsubsection*{Rationale} Just as explaining your point more and more vigorously
isn't always the best way to help someone understand what you're
saying, you need to be a bit subtle when thinking about how to bring
things into balance.  One of the things we're learning through
peeragogy is what it means to ``work smart.''  Although frustrations
are inevitable, thinking of the project as ecosystem rather than a
machine will help avoid problems of the ``square peg, round hole''
variety.

\subsubsection*{Resolution}
Awareness that everyone, and everything, has a \patternname{Carrying
  Capacity} serves as a good reminder for anyone who believes him- or
herself to be caring more about project outcomes than other
participants, that, in fact, outcomes do not affect everyone the same way.
Cultivating a healthy working environment will lead to positive relationships
with others who have a different outlook, and different, but still compatible goals.

\begin{framed}
\emph{What's Next.}
Making it easy
and fruitful for others to get involved is one of the best ways to
redistribute the load (compare the
\patternname{\href{http://peeragogy.org/practice/heuristics/newcomer/}{Newcomer}}
pattern).
\end{framed}
\endgroup

    
    
    

    