\section{Carrying capacity}\label{sec:Carrying capacity}

\subsubsection*{Context}

There's only so much any one person can do with limited resources and a limited amount of time. In a peeragogy context, it is often impossible to delegate work to others.  Lines of responsibility are not always clear, and people can easily get burnt out. 
\textbf{Our concern is not simply ``inclusion'' but rather to help people to fulfil their potential.}

\subsubsection*{Problem}

How can we help prevent those people who are involved with the project from overpromising or overcommitting, and subsequently crashing and burning?  First, let's be clear that are lots of ways things can go wrong.  Simplistic expectations -- like \emph{assuming that others will do the work for you} \cite{torvalds-interview} -- can undermine your ability to correctly gauge your own strengths, weaknesses, and commitments.  Without careful, critical engagement, you might not even notice when there's a problem.  Where one person has trouble letting go, others may have trouble speaking up.  Pressure builds when communication isn't going well.  
% At the same time, we all seem to have a lot to learn about how to pay attention and make constructive contributions in situations that are always changing.

\subsubsection*{Solution}

Symptoms of burnout are a sign that it's time to revisit the group's \patternname{Roadmap} and your own individual plan.  Are these realistic?  Frustration with other people is a good time to ask questions and let others answer.  Do they see things the same way you do?   If you have a ``buddy'' they can provide a reality check.   Maybe things are not \emph{that hard} after all -- and maybe they don't need to be done \emph{right now}.  Generalizing from this: the project can promote an open dialog by asking questions, and creating opportunities for people to share their worries \cite{seikkula2006dialogical}.  Use the project \patternname{Scrapbook} to make note of obstacles.  For example, if you'd like to pass a baton, you'll need someone there who can take it.  Maybe you can't find that person right away, but you can bring up the concern and get it onto the project's \patternname{Roadmap}.  The situation is always changing, but if we continue to create suitable checkpoints and benchmarks, then we can take steps to take care of an issue that's getting bogged down.    

\subsubsection*{Rationale}

Think of the project as an ecosystem populated by acts of participation.  As we get to know more about ourselves and each other, we know what sorts of things we can expect, and we are able to work together more sustainably \cite{ostrom2010revising}.
%
We can regulate our individual stress levels and improve collective outcomes by discussing concerns openly.

\subsubsection*{Resolution}

Guiding and rebalancing behaviour in a social context may begin by simply speaking up about a concern.  What we learn in this process is  consistent with inclusivity \cite{garrison2013toward}, but goes further, as participants are invited to be candid about what works well for them and what does not.
%
As we share concerns and are met with care and practical support, our actions begin to align better with expectations (often as a result of forming more realistic expectations).  When we have the opportunity to express and rethink our concerns, we can become more clear about the commitments we're prepared to make.  As we become aware of the problems others are facing, we often find places where we ourselves have something to learn.

\subsubsection*{Example 1}
Wikipedia aims to emphasize a neutral point of view, but its users are
not neutral.\footnote{\url{https://en.wikipedia.org/wiki/Wikipedia:Neutral_point_of_view}}
Wikipedia is relevant to things that matter to us.  It
helps inform us regarding our necessary purposes -- and we are invited
to ``speak up'' by making edits on pages that matter to us.  However,
this, in combination with factors of accessibility, leads to the
situation in which coverage and participation are not neutral in
another sense: more information on Wikipedia deals with Europe than
all of the locations outside of Europe \cite{graham2014uneven}.
Wikipedia's \patternname{Carrying capacity} is unevenly distributed
in other ways.  A recent solicitation for donations to the Wikimedia
Foundation says ``Wikipedia has over 450 million readers.  Less than
1\% give.''
%
Editor engagement is a fundamental concern for the
Wikimedia Foundation.  The number of active editors has been falling since 2007.\footnote{\url{https://strategy.wikimedia.org/wiki/Editor_Trends_Study/Results}}

\subsubsection*{Example 2}
A separate Ladies Hall seems entirely archaic.  Progressive thinkers have for
some time subscribed to the view that ``there shall be no women in
case there be not men, nor men in case there be not women''
\cite[Chapter 1.LII]{rabelais1894gargantua}.  However, in light of the
extreme gender imbalance in free software, and still striking
imbalance at Wikipedia \cite{gender,FM4291}, it will be important to
do whatever it takes to make women and girls welcome, not least
because this is a significant factor in boosting our
\patternname{Carrying capacity}.

%\subsubsection*{Summary}

\begin{framed}
\noindent 
\emph{What's Next.}  Making it easy and fruitful for others to get involved is one of the best ways to redistribute the load (compare the
\patternname{Newcomer} pattern).
\end{framed}



  
