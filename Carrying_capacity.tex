\section{Carrying capacity}\label{sec:Carrying capacity}

\subsubsection*{Motivation} This pattern can help project participants recognise and communicate their stresses to make themselves and the project  more resilient.


%% \begin{center}
%% \begin{tabular}{l}
%% \textbf{$\leftarrow$\patternname{Reduce, reuse, recycle}: All available perspectives can give the project more to work with.}\\
%% \textbf{$\leftarrow$\patternname{A specific project}: We may need help to create or activate a plan.}\\
%% \textbf{$\leftarrow$\patternname{Wrapper}: Share skills and be transparent about limitations and bottlenecks.}\\
%% \textbf{$\leftarrow$\patternname{Heartbeat}: Project activites should give us rewards, not drain our energy.}\\
%% \end{tabular}
%% \end{center}

\subsubsection*{Context}

One of the important maxims from the world of FLOSS is:
``Given enough eyeballs, all bugs are shallow'' \cite[p.~30]{raymond2001cathedral}.
A partial converse is also true.

\subsubsection*{Forces}
\raisebox{-2\baselineskip}
{{\centering
\begin{tabular}{p{.85\textwidth}}
\textbf{Boundedness}: there's only so much any one person can do.\\
\textbf{Independence}: in a peeragogy context, it is often impossible to delegate work to others.\\
\textbf{Antifragility}: potential can only be realised if people take on enough but not too much.
\end{tabular}
}}

\subsubsection*{Problem}

How can we help prevent those people who are involved with the project from overpromising or overcommitting, and subsequently crashing and burning?  First, let's be clear that are lots of ways things can go wrong.  Simplistic expectations -- like \emph{assuming that others will do the work for you} \cite{torvalds-interview} -- can undermine your ability to correctly gauge your own strengths, weaknesses, and commitments.  Without careful, critical engagement, you might not even notice when there's a problem.  Where one person has trouble letting go, others may have trouble speaking up.  Pressure builds when communication isn't going well.  
% At the same time, we all seem to have a lot to learn about how to pay attention and make constructive contributions in situations that are always changing.

\subsubsection*{Solution}

Symptoms of burnout are a sign that it's time to revisit the group's \patternname{Roadmap} and your own individual plan.  Are these realistic?  Frustration with other people is a good time to ask questions and let others answer.  Do they see things the same way you do?  Your goals may be aligned, even if your methods and motivations differ. If you have a ``buddy'' they can provide a reality check.   Maybe things are not \emph{that hard} after all -- and maybe they don't need to be done \emph{right now}.  Generalizing from this: the project can promote an open dialog by creating opportunities for people to share their worries and generate an emergent plan for addressing them \cite{seikkula2006dialogical}.  Use the project \patternname{Scrapbook} to make note of obstacles.  For example, if you'd like to pass a baton, you'll need someone there who can take it.  Maybe you can't find that person right away, but you can bring up the concern and get it onto the project's \patternname{Roadmap}.  The situation is always changing, but if we continue to create suitable checkpoints and benchmarks, then we can take steps to take care of an issue that's getting bogged down.    

\subsubsection*{Rationale}

Think of the project as an ecosystem populated by acts of participation.  As we get to know more about ourselves and each other, we know what sorts of things we can expect, and we are able to work together more sustainably \cite{ostrom2010revising}.
%
We can regulate our individual stress levels and improve collective outcomes by discussing concerns openly.

\subsubsection*{Resolution}

Guiding and rebalancing behaviour in a social context can begin with speaking up about a concern.  When we acknowledge our concerns and those of others, we take into account our \textbf{boundedness}.  Being aware of the problems and limitations that others face, we have the opportunity to help out, without impinging on others' \textbf{independence}.  For example, one person who listens to another's concerns may discover a concern of his or her own, and help create an opportunity for both people to learn.  This process is  consistent with inclusivity \cite{garrison2013toward}, but it is also caring as participants are invited to be candid about what works well for them and what does not.  This doesn't mean including all possible stresses: we work to stay within the realm of \textbf{antifragility} \cite{taleb2012antifragile}, where stress improves the system, rather than degrading them. 
%
As we share concerns and are met with care and practical support, our actions begin to align better with expectations (often as a result of forming more realistic expectations). 

\subsubsection*{Example 1}
Wikipedia aims to emphasize a neutral point of view, but its users are
not neutral.\footnote{\url{https://en.wikipedia.org/wiki/Wikipedia:Neutral_point_of_view}}
Wikipedia is relevant to things that matter to us.  It
helps inform us regarding our necessary purposes -- and we are invited
to ``speak up'' by making edits on pages that matter to us.  However,
coverage and participation are not neutral in another sense.
More information on Wikipedia deals with Europe than
all of the locations outside of Europe \cite{graham2014uneven}.
A recent solicitation for donations to the Wikimedia Foundation
says ``Wikipedia has over 450 million readers.  Less than 1\% give.''
%
As we remarked in the \patternname{Peeragogy} pattern, most of the
actual work is contibuted by a small percentage of users as well.
%
Furthermore, the technology limits what can be said; 
\cite{graham2014uneven} remark on
``the structural inability of the platform itself to incorporate fundamental epistemological diversity.''
%
Finally, the overall population of editors is an important concern for
the Wikimedia Foundation: the total number of active editors has been
falling since
2007.\footnote{\url{https://strategy.wikimedia.org/wiki/Editor_Trends_Study/Results}}

\subsubsection*{Example 2}
A separate Ladies Hall seems entirely archaic.  Progressive thinkers have for
some time subscribed to the view that ``there shall be no women in
case there be not men, nor men in case there be not women''
\cite[Chapter 1.LII]{rabelais1894gargantua}.  However, in light of the
extreme gender imbalance in free software, and still striking
imbalance at Wikipedia \cite{gender,FM4291}, it will be important to
do whatever it takes to make women and girls welcome, not least
because this is a significant factor in boosting our
\patternname{Carrying capacity}.

%\subsubsection*{Summary}

\begin{framed}
\noindent 
\emph{What's Next.}  Making it easy and fruitful for others to get involved is one of the best ways to redistribute the load.  This often requires skill development among those involved; compare the \patternname{Newcomer} pattern.
\end{framed}



  
