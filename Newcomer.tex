

% DK: I am not sure that this is a Pattern. It is a role for sure, but is it a solution to a problem? Maybe the name of the pattern is wrong. Something like Welcome Wagon? Something that evokes the response to the Newcomer.
\section{Newcomer}\label{sec:Newcomer}

\subsubsection*{Motivation} This pattern can help project participants be aware of the issues faced by newcomers, and cultivate a ``beginner's mind'' themselves.

\begin{center}
\begin{tabular}{l}
\textbf{$\leftarrow$\patternname{Roadmap}: A transparent plan can show outsiders what it would be like to get involved.}\\
\textbf{$\leftarrow$\patternname{Carrying capacity}: Boosting the project's capacity may require training in new participants.}\\
\textbf{$\leftarrow$\patternname{Wrapper}: Ideally we would actively welcome all contributions with love.}\\
\end{tabular}
\end{center}

\subsubsection*{Context}
% DK: This is an interesting point. I am not sure it is Context. Maybe Rationale?
When there's learning happening, it's because there is someone who is new to a topic, or to something about the topic.

\subsubsection*{Forces}
\raisebox{-.5\baselineskip}
{{\centering
\begin{tabular}{l}
\textbf{Individuation: each person learning optimally is what's best for the community.}\\
\textbf{Mutuality: our individuality does not isolate us from one another, but draws us together}
\end{tabular}
}}

\subsubsection*{Problem} Newcomers can feel overwhelmed by the amount of things to learn.  They
don't know where to start.  They may have a bunch of ideas that the
oldtimers have never considered -- or they may think they have new
ideas, which are actually a different take on an old idea; see
\patternname{Reduce, reuse, recycle}. People who are new to the project can tell you what makes their participation difficult.  Since you're learning as you go as well, you can ask yourself the same question: what aspects of this are difficult for me?  

% DK: This seems a little spare
\subsubsection*{Solution}

In an active learning context, we render assistance to others more effectively when we do
so as peers, rather than doing it as experts operating in a \emph{provisionist} mode
\cite{boud2005peer}.  Instead of thinking of newcomers as
``them'', and trying to provide solutions, we focus on newcomers as
``us'' -- which makes the search for solutions that much more urgent.
We permit ourselves to ask naive questions.  We entertain vague ideas.
We add concreteness by trying \patternname{A specific project}.  We may then genuinely turn
to others for help. 
% OSS: What about action research?
% similar... not handholding but interactive
Best practice for newcomers is to immediately engage in formal action research on their peer learning about the project. Systematically taking notes and gathering data to analyze and reflect upon later would leave robust artifacts for future newcomers to use and build upon in own action research. This is a lot to ask for a someone just joining a group, so certainly it is not required, but action research cycles of reflect, plan, act, and observe can be a very effective way to improve the project and their own understanding.\footnote{\url{https://valenciacollege.edu/faculty/development/tla/actionResearch/ARP_softchalk/mobile_pages/index.html}}

% DK: I can empathize :+) I wonder if there is a pattern that is missing. Vision? A clear articulation about what the effort is for
%
\subsubsection*{Rationale} 
%
Sharing vulnerability and confusion gives us a chance to learn
together.  A newcomer's confusion about how best to get involved or
what the point of all this actually is may be due to a lack of
structure in the project \patternname{Roadmap}, and it points to
places where others in the project probably have something to learn too.
%
%% In the words of Antoine de Saint-Exup\'ery:
%% ``If you want to build a boat, do not instruct the men to saw wood,
%% stitch the sails, prepare the tools and organize the work,
%% but make them long for setting sail and travel to distant lands.''

% DK: The resolution should describe the circumstance after applying the solution. This seems to be describe the circumstance without the solution.
\subsubsection*{Resolution}
An awareness of the difficulties that newcomers face can
help us be more compassionate to ourselves and others.  We strengthen the community
by supporting all participants' \textbf{individuation}.  We will have a better chance of making
the project useful for others if we're clear about how it is useful to \emph{us}; by welcoming newcomers, we enhance the sense of \textbf{mutuality} with people who have never encountered the project before, and learn together with them.

\subsubsection*{Example 1} Wikipedia \patternnameplural{Newcomer} can make use of resources that
include a ``Teahouse'' where questions are welcomed, a platform extension that changes the user
interface for new editors, and lots of documentation.\footnote{\url{https://en.wikipedia.org/wiki/Wikipedia:Teahouse}}%
\textsuperscript{,}\footnote{\url{https://en.wikipedia.org/wiki/Wikipedia:GettingStarted}}%
\textsuperscript{,}\footnote{\url{https://en.wikipedia.org/wiki/Help:Editing}}
The efforts of exceptional newcomers may be given special
recognition.\footnote{\url{https://en.wikipedia.org/wiki/Template:The_New_Editor\%27s_Barnstar}}
Newcomer ``survival'' is of interest to the Wikimedia
Foundation.\footnote{\url{https://meta.wikimedia.org/wiki/Research:Newcomer_survival_models}}
The degree to which Wikimedia projects emphasize continuous upskilling
(\`a la the \patternname{Newcomer} pattern) is somewhat less clear.

\subsubsection*{Example 2} It will often be pragmatic to connect
\patternnameplural{Newcomer} with employment, so that the future
university may see a closer coupling of science and industry than is
held in the former model.  Inspiration can be drawn the London-based freelancing cooperative Founders\&Coders, which is
able to offer intensive training in web development at no cost to
successful applicants, on the basis that some trainees will choose to
join the cooperative as paying members later
on.\footnote{\url{http://www.foundersandcoders.com/academy/}}


\begin{framed}
\noindent 
\emph{What's Next.}
A more detailed (but non-limiting) ``How to Get Involved'' walk-through or ``DIY Toolkit'' would be good to develop. We can start by listing some of the things we're currently learning about.
\end{framed}
