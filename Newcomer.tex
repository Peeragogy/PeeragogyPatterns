

% DK: I am not sure that this is a Pattern. It is a role for sure, but is it a solution to a problem? Maybe the name of the pattern is wrong. Something like Welcome Wagon? Something that evokes the response to the Newcomer.
\section{Newcomer}\label{sec:Newcomer}

\subsubsection*{Context}
% DK: This is an interesting point. I am not sure it is Context. Maybe Rationale?
Education assumes we are speaking to a new generation. 
In learning more broadly, the ``audience'' is typically new to the topic or to some aspect of the topic.

\subsubsection*{Problem} Newcomers can feel overwhelmed by the amount of things to learn.  They
don't know where to start.  They may have a bunch of ideas that the
oldtimers have never considered -- or they may think they have new
ideas, which are actually a different take on an old idea; see
\patternname{Reduce, reuse, recycle}.

% DK: This seems a little spare
\subsubsection*{Solution}
The primary feature of our solution to the problems faced by newcomers
is to shift the focus to our own experience as newcomers.
It is is good to try to become aware of what a newcomer needs, and what their
motivations are -- but it is even more effective when we can do this in a first
person mode rather than a ``provisionist'' mode \cite{boud2005peer}.  Instead of
thinking of newcomers as ``them'', and trying to provide solutions, we focus
on newcomers as ``us'' -- which makes the search for solutions that much more urgent. 
As newcomers, we find ourselves asking naive questions.
We begin with a relatively vague idea of our goals are, 
and it can be helpful to add concreteness by trying \patternname{A Specific Project}.

% DK: I can empathize :+) I wonder if there is a pattern that is missing. Vision? A clear articulation about what the effort is for
%
\subsubsection*{Rationale} 
When we're open about being newcomers ourselves, we become much more
ready to join other newcomers as peers.  After all, we're new to each other.  A
\patternnameposesssive{Newcomer} confusion about how best to get involved
or what the point of all this actually is may not be a question of
insufficient detail, but lack of structure in the project
\patternname{Roadmap}.  Sharing vulnerability gives us a chance to learn together.
%
%% In the words of Antoine de Saint-Exup\'ery:
%% ``If you want to build a boat, do not instruct the men to saw wood,
%% stitch the sails, prepare the tools and organize the work,
%% but make them long for setting sail and travel to distant lands.''

% DK: The resolution should describe the circumstance after applying the solution. This seems to be describe the circumstance without the solution.
\subsubsection*{Resolution}
An awareness of the difficulties that newcomers face can
help us be more compassionate to ourselves and others.  We
become open to new ideas, which can show how we have
been limiting ourselves.

\begin{framed}
\emph{What's Next.}
A more detailed (but non-limiting) ``How to Get Involved'' walk-through in text or video form would be good to develop. We can start by listing some of the things we're currently learning about, including: business issues relevant to the Peeragogy project, how to run a MOOC, and hot-syncing our website from Git.
\end{framed}


