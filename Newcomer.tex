

% DK: I am not sure that this is a Pattern. It is a role for sure, but is it a solution to a problem? Maybe the name of the pattern is wrong. Something like Welcome Wagon? Something that evokes the response to the Newcomer.
\section{Newcomer}\label{sec:Newcomer}

\subsubsection*{Context}
% DK: This is an interesting point. I am not sure it is Context. Maybe Rationale?
A lot of ``education'' assumes we are speaking to a new generation. 
In learning more broadly, the ``audience'' is typically new to the topic or some aspect of the topic.
Sometimes we are the \patternname{Newcomers}, sometimes we are the oldtimers.

\subsubsection*{Problem} \patternname{Newcomers} can feel overwhelmed by the amount of things to learn.  They
don't know where to start.  They may have a bunch of ideas that the
oldtimers have never considered -- or they may think they have new
ideas, which are actually a different take on old ideas; see
\patternname{Reduce, Reuse, Recycle}.

% DK: This seems a little spare
\subsubsection*{Solution} It is good to try to become aware of what a
\patternname{Newcomer} needs, and what their motivations are.  They may ask naive questions, which
should be met with patient answers, guiding questions, and links to resources.
Moreover, a naive question can be quite sophisticated -- pointing out unrealistic
assumptions on the part of project organisers.  \patternname{Newcomers} themselves
may have only a general idea about what their goals are, so it can be helpful to add
concreteness by nudging them towards work on \patternname{A Specific Project}.

% DK: I can empathize :+) I wonder if there is a pattern that is missing. Vision? A clear articulation about what the effort is for
\subsubsection*{Rationale} 
New ideas can prompt us to consider how we may have been limiting ourselves.
If \patternname{Newcomers} complain that they are confused, this is a sign that
the \patternname{Roadmap} has not been made sufficiently clear.  This may not
be a question of detail, but of a clear articulation of the high-level vision.
%% In the words of Antoine de Saint-Exup\'ery:
%% ``If you want to build a boat, do not instruct the men to saw wood,
%% stitch the sails, prepare the tools and organize the work,
%% but make them long for setting sail and travel to distant lands.''

% DK: The resolution should describe the circumstance after applying the solution. This seems to be describe the circumstance without the solution.
\subsubsection*{Resolution}
An awareness of the difficulties that \patternname{Newcomers} face can
help us be more compassionate to ourselves and others.

\begin{framed}
\emph{What's Next.}
A more detailed (but non-limiting) ``How to Get Involved'' walk-through in text or video form would be good to develop. We can start by listing some of the things we're currently learning about, including: business issues relevant to the Peeragogy project, how to run a MOOC, and hot-syncing our website from Git.
\end{framed}


