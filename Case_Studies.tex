\section{Case Studies}\label{sec:Case Studies}

\subsection{Wikimedia projects.}

\patternname{Newcomer} There are lots of resources newcomers can use.
\footnote{\url{https://commons.wikimedia.org/wiki/File:Exceptional_newcomer.jpg}}
\footnote{\url{https://en.wikipedia.org/wiki/Help:Editing}}
\footnote{\url{https://en.wikipedia.org/wiki/Wikipedia:Teahouse}}
\footnote{\url{https://m.wikimediafoundation.org/wiki/Staff_and_contractors}}
\footnote{\url{https://en.m.wikipedia.org/wiki/Wikipedia:List_of_Wikipedians_by_number_of_edits}}

\patternname{Reduce, reuse, recycle} DBPedia is super huge in semantic
web.  There have been some interesting experimental projects e.g. for
building learning paths through Wikipedia content, or showing heatmaps
of editing activity -- these aren't made that accessible to day-to-day
users.

\patternname{Wrapper} There are lots of blogs.  However, given the
issue with finding the latest experiments, Wikimedia as a community
seems to be good at building resources but not always giving exposure
to contributors.

\patternname{Roadmap} The \emph{Help} tab is useful for newcomers, but
also provides a sort of roadmap for the user's
engagement.\footnote{\url{https://en.wikipedia.org/wiki/Help:Contents}}

-- but many people will bypass help functions (in Charlotte's
experience).  Companies can save a lot of time \& money by making the
help function really good.  If someone rates them highly that is even
better.

\begin{quotation}
\noindent 
I want to read or find an article;
I want to edit an article;
I want to report a problem with an article;
I want to create a new article or upload media;
I have a factual question.
etc.
\end{quotation}

A recent post from Jonathan Morgan of WMF talks about \emph{outlines}: 

\begin{quotation}
\noindent Cross-posting this request to the {\tt wiki-research-l} mailing
list. Anyone have data on frequently used section titles in articles
(any language), or know of datasets/publications that examined this?
\end{quotation}

Quoting from \patternname{A specific project}: ``we are often blinded by our ... preferences'' -- In everything that's written here, there are certain phrases that I especially respond to.

Buzzwords or things that are relevant, things that feel useful --
through these statements, I feel a connection, can start to plug in.
If you had a paper like this, and each person could log in, and
highlight the lines they liked, they might find out how much they
resonated with other people with similar interests and concerns--
there are many issues there and many ways to plug in.

re the five pillars of wikipedia: number 2 neutrality. I'm not
neutral. the phrase "objective understanding" doesn't mean you don't
care, it just means that you know what you care about!  CD:
Independent news sources can get this really twisted.  MH: "I'm not
neutral about children's lives."  "Because if we don't they are in
trouble."

``Objective'' means to push off of -- we know what we want and what
we don't want.  If you want to guard the kids, then you, as a person,
can see 10000 things that might go wrong -- although you don't have
to think about each of the 10000 things We're not bullshitting, we're
talking about things that matter to us!  Wikipedia is important
because it is relevant (to things that matter to us).  helping to
inform us re our necessary purposes...

re: number 4:ethics. yes. ethics means civil solutions. As in
peaceably, agreeably as humanly possible, not violently.
 
\patternname{Heartbeat} \emph{Wikimania} provides the clearest
heartbeat for the wikimedia projects.

\patternname{Scrapbook} - we could reference the previous ``five year
plan'' as a sort of scrapbook.

\patternname{Carrying capacity} - ?

\patternname{Peeragogy} - ?

% Ladies Hall, South Dormitory, University Hall, Assembly Halls \& Library, North Dormitory, Science Hall, President's Residence, University Farm, and Washburn Observatory.
\subsection{The future university.}
It is useful to compare and contrast the patterns we have introduced
with the typical or stereotypical image of a university from Figure
\ref{madison-map}.  Most likely we would have little use for a
separate Ladies Hall, for ``there shall be no women in case there be
not men, nor men in case there be not women'' \cite[Chapter
  1.LII]{rabelais1894gargantua}.  However, in light of the current
extreme gender imbalance in free software, it is important to do
whatever it takes to make women and girls welcome, not least because
this is a significant factor in boosting our \patternname{Carrying
  capacity}.  Dormitories may be seen as as an optional extra, since
studying from where you live is an option.  However,
cooperatively-owned living/working spaces may frequently be an asset
for \patternname{A specific project}.  In-person meetings for
special-purpose assemblies (e.g., for conventions, dances, and
commencement ceremonies) could comprise an important part of the
project's \patternname{Wrapper}.  One of the most noticeable things
about ``now'' is that we have a vast array of resources available
online.  Indeed, this may be the most significant factor in making the
future university conceivable.  However, these resources are not
always as organized as they would need to be for serious educative
purposes, so putting effort into \patternname{Reduce, reuse, and
  recycle}'ing them is one place where peeragogues could usefully
apply effort.  It will often make sense to connect learning as a
research \patternname{Newcomer} to part-time or future employment, so
that the future university may see a closer coupling of science and
industry.  We have no need to maintain a special president's
residence, although it would be nice to have a place to throw parties
and work collaboratively on the project's \patternname{Roadmap} and
\patternname{Scrapbook}.  One or more working farms could be useful to
help physically sustain peeragogues, while putting the project's
\patternname{Heartbeat} in tune with that of the seasons.  And as for
an observatory, that is also important, if \patternname{Peeragogy} is
truly to reach \emph{ad astra, per aspera}.


  