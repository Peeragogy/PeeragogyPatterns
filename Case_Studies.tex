\section{Case Studies}\label{sec:Case Studies}

In order to show some examples of these patterns ``in action'' we will
briefly look at examples from the Wikimedia project.  We then turn to
a hypothetical case study in design.

\subsection{Wikimedia projects.}

Wiki users are invited to contribute content as well as software, and
can get involved with governance and other ``meta'' duties.  We would
claim that this pluralistic approach is an example of
\patternname{Peeragogy} in action.  These numbers show that the
approach achieves something important: Wikimedia Foundation runs one
the 7\textsuperscript{th} most popular website, and has around 230
employees.  For comparison, the 6\textsuperscript{th} most popular
website is Amazon, with more than 150K employees, and the
8\textsuperscript{th} most popular website is run by Tencent Holdings
Limited, with around 30K employees.

There are lots of resources a Wikipedia \patternnameplural{Newcomer}
can use, including the ``Teahouse'', and lots of
documentation.\footnote{\url{https://en.wikipedia.org/wiki/Wikipedia:Teahouse}}\textsuperscript{,}\footnote{\url{https://en.wikipedia.org/wiki/Help:Editing}}
They may be rewarded for their efforts with a special
``barnstar''.\footnote{\url{https://commons.wikimedia.org/wiki/File:Exceptional_newcomer.jpg}}
Newcomer ``survival'' is of interest to the Wikimedia
foundation.\footnote{\url{https://meta.wikimedia.org/wiki/Research:Newcomer_survival_models}}
%
The \emph{Help} tab (on every Wikipedia page) could provides a sort of
\patternname{Roadmap} for the user's
engagement:\footnote{\url{https://en.wikipedia.org/wiki/Help:Contents}}

\begin{quotation}
\noindent 
I want to read or find an article;
I want to edit an article;
I want to report a problem with an article;
I want to create a new article or upload media;
I have a factual question.
(etc.)
\end{quotation}
%
%% However, many people will bypass help functions.A recent post from
%% Jonathan Morgan of WMF talks about \emph{outlines} at the article
%% level, which would give some insight into an article-level
%% ``roadmap.''
%
%% \begin{quotation}
%% \noindent Cross-posting this request to the {\tt wiki-research-l} mailing
%% list. Anyone have data on frequently used section titles in articles
%% (any language), or know of datasets/publications that examined this?
%% \end{quotation}
%
%% One may have to turn to the research literature to give an idea of
%% current global coverage of Wikipedia \cite{holloway2007analyzing}.
%
Although articles and other items are filled in opportunistically,
plans for the future exist at the level of individual projects and for
the Wikimedia Foundation as a whole.  For example, we could reference
the Wikimedia Foundation's previous ``five year plan'' as a sort of
\patternname{Scrapbook}.


% \footnote{\url{https://m.wikimediafoundation.org/wiki/Staff_and_contractors}}
% \footnote{\url{https://en.m.wikipedia.org/wiki/Wikipedia:List_of_Wikipedians_by_number_of_edits}}

One of the best ways to connect with other users on Wikipedia is to
join \patternname{A specific project} (known as a
``WikiProject'').\footnote{\url{https://en.wikipedia.org/wiki/Wikipedia:WikiProject_Council/Directory}}\textsuperscript{,}\footnote{\url{https://en.wikipedia.org/wiki/Wikipedia:WikiProject_Council/Guide}}
Individual projects typically maintain their own lists of objectives,
e.g., articles in progress.

Users are encouraged to recycle existing works that are compatible
with the CC-By-SA license and the mission of the respective sites (and
some WikiProjects exist purely to help re-purpose existing works in
this way).  On the downstream side, DBPedia is huge in semantic web,
built from the data in Wikipedia's ``infoboxes''.  Researchers have
been able to \patternname{Reduce, reuse, recycle} Wikipedia material
in other ways, e.g.~developing tools for building learning paths
through Wikipedia content, or showing heatmaps of editing activity.
However, these research interventions do not always result in
something accessible to day-to-day users.  Dissemination is something
that can always be improved.  There are currently lots of blogs and
feeds that can be used to stay informed, comprising a
\patternname{Wrapper} function.
%%   Wikimedia as a community
%% seems to be good at building resources, but not always giving exposure
%% to contributors.

 
The yearly in-person gathering, Wikimania, provides the clearest
example of a \patternname{Heartbeat} for the Wikimedia movement.  This
is one way in which Wikipedia functions as a social network.  Also of
note is the twice-yearly call for for proposals for individual
engagement
grants.\footnote{\url{https://meta.wikimedia.org/wiki/Grants:IEG}}

%``we are often blinded
%by our ... preferences'' -- In everything that's written here, there
%are certain phrases that I especially respond to.
%
%% Buzzwords or things that are relevant, things that feel useful --
%% through these statements, I feel a connection, can start to plug in.
%% If you had a paper like this, and each person could log in, and
%% highlight the lines they liked, they might find out how much they
%% resonated with other people with similar interests and concerns--
%% there are many issues there and many ways to plug in.
%
%re the five pillars of wikipedia: number 2 neutrality.
Wikipedia may be neutral, but we're not neutral.
%
%% the phrase "objective understanding" doesn't mean you don't care,
%% it just means that you know what you care about!  CD: Independent
%% news sources can get this really twisted.  MH: "I'm not neutral
%% about children's lives."  "Because if we don't they are in
%% trouble."
%
%% ``Objective'' means to push off of -- we know what we want and what
%% we don't want.  If you want to guard the kids, then you, as a person,
%% can see 10000 things that might go wrong -- although you don't have
%% to think about each of the 10000 things We're not bullshitting, we're
%% talking about things that matter to us!  
%
Wikipedia is important because it is relevant to things that matter to
us.  It helps to inform us regarding our necessary purposes -- and we
are invited to ``speak up'' by making edits to pages that matter to
us.  In theory, this builds Wikipedia's \patternname{Carrying
  capacity}, although editor engagement is a continual concern (and
has been seen to be falling in recent years).
%
%% re: number 4:ethics. yes. ethics means civil solutions. As in
%% peaceably, agreeably as humanly possible, not violently.



% Ladies Hall, South Dormitory, University Hall, Assembly Halls \& Library, North Dormitory, Science Hall, President's Residence, University Farm, and Washburn Observatory.
\subsection{The future university.}
As a design exercise, it is useful to compare and contrast the
patterns we have introduced with the typical or stereotypical image of
a university from Figure \ref{madison-map}.  Most likely we would have
little use for a separate Ladies Hall, for ``there shall be no women
in case there be not men, nor men in case there be not women''
\cite[Chapter 1.LII]{rabelais1894gargantua}.  However, in light of the
current extreme gender imbalance in free software, it is important to
do whatever it takes to make women and girls welcome, not least
because this is a significant factor in boosting our
\patternname{Carrying capacity}.  Dormitories may be seen as as an
optional extra, since studying from where you live is an option.
However, cooperatively-owned living/working spaces may frequently be
an asset for \patternname{A specific project}.  In-person meetings for
special-purpose assemblies (e.g., for conventions, dances, and
commencement ceremonies) could comprise an important part of the
project's \patternname{Wrapper}.  One of the most noticeable things
about ``now'' is that we have a vast array of resources available
online.  Indeed, this may be the most significant factor in making the
future university conceivable.  However, these resources are not
always as organized as they would need to be for serious educative
purposes, so putting effort into \patternname{Reduce, reuse, and
  recycle}'ing them is one place where peeragogues could usefully
apply effort.  It will often make sense to connect learning as a
research \patternname{Newcomer} to part-time or future employment, so
that the future university may see a closer coupling of science and
industry.  We have no need to maintain a special president's
residence, although it would be nice to have a place to throw parties
and work collaboratively on the project's \patternname{Roadmap} and
\patternname{Scrapbook}.  One or more working farms could be useful to
help physically sustain peeragogues, while putting the project's
\patternname{Heartbeat} in tune with that of the seasons.  And as for
an observatory, that is also important, if \patternname{Peeragogy} is
truly to reach \emph{ad astra, per aspera}.


  
