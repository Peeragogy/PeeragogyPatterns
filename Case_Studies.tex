\section{Case Studies}\label{sec:Case Studies}

In order to show some examples of these patterns ``in action'' we will
briefly look at the way the patterns manifest in the Wikimedia
projects (that is, Wikipedia and its sister sites).  We then turn to a
case study in design, in which we sketch some key features of the
future university.

\subsection{Wikimedia projects.}

Wiki users are invited to contribute content, extensions to software,
and to get involved with governance and other ``meta'' duties.  We
claim that this pluralistic approach is an example of
\patternname{Peeragogy}.  It achieves something
impressive: the Wikimedia Foundation runs the 7\textsuperscript{th}
most popular website in the world, and has around 230 employees.  For
comparison, the 6\textsuperscript{th} and 8\textsuperscript{th} most
popular websites are run by companies with 150K and 30K employees,
respectively.
% most popular website is Amazon, with more than 150K employees, 
% is run by Tencent Holdings Limited, with around 30K employees.

Wikipedia \patternnameplural{Newcomer} can make use of resources that
include a ``Teahouse'' where questions are welcomed, a recently
developed ``Getting Started'' extension that changes the user
interface for new editors, and lots of preexisting
documentation.\footnote{\url{https://en.wikipedia.org/wiki/Wikipedia:Teahouse}}\textsuperscript{,}\footnote{\url{https://en.wikipedia.org/wiki/Wikipedia:GettingStarted}}\textsuperscript{,}\footnote{\url{https://en.wikipedia.org/wiki/Help:Editing}}
The efforts of exceptional newcomers may be given special
recognition.\footnote{\url{https://en.wikipedia.org/wiki/Template:The_New_Editor\%27s_Barnstar}}
Newcomer ``survival'' is of interest to the Wikimedia
foundation.\footnote{\url{https://meta.wikimedia.org/wiki/Research:Newcomer_survival_models}}
The degree to which Wikimedia projects emphasize continuous upskilling
(\`a la our \patternname{Newcomer} pattern) is somewhat less clear.

One important piece of documentation available for new and old users
alike is the \emph{Help} link, which occupies prime real estate --
it's on every Wikipedia page.  This could be seen as a localized
\patternname{Roadmap} for individual user
engagement:\footnote{\url{https://en.wikipedia.org/wiki/Help:Contents}}

\begin{quotation}
\noindent 
I want to read or find an article;
I want to edit an article;
I want to report a problem with an article;
I want to create a new article or upload media;
I have a factual question\ldots
[Etc.]
\end{quotation}
%
%% However, many people will bypass help functions.A recent post from
%% Jonathan Morgan of WMF talks about \emph{outlines} at the article
%% level, which would give some insight into an article-level
%% ``roadmap.''
%
%% \begin{quotation}
%% \noindent Cross-posting this request to the {\tt wiki-research-l} mailing
%% list. Anyone have data on frequently used section titles in articles
%% (any language), or know of datasets/publications that examined this?
%% \end{quotation}
%
%% One may have to turn to the research literature to give an idea of
%% current global coverage of Wikipedia \cite{holloway2007analyzing}.
%
Volunteers create articles and other changes opportunistically --
however plans for the future exist at the level of individual projects
and for Wikimedia as a whole.  We could reference the Wikimedia
Foundation's previous ``five year plan'' as an example of a
\patternname{Scrapbook} \cite{wikimedia2011plan}.  More recently this
has been supplanted by an ``ongoing'' strategy discussion, which might
be thought of as the current \patternname{Roadmap}, although one is
not sure what to make of the fact that this global plan has not been
edited for nearly a
year.\footnote{\url{https://meta.wikimedia.org/wiki/Strategy_project}}
%
% \footnote{\url{https://m.wikimediafoundation.org/wiki/Staff_and_contractors}}
% \footnote{\url{https://en.m.wikipedia.org/wiki/Wikipedia:List_of_Wikipedians_by_number_of_edits}}

One of the best ways to jump in, get to know other Wikipedia users,
and start working on a focused todo list is to join (or start)
\patternname{A specific project}, namely a
``WikiProject''.\footnote{\url{https://en.wikipedia.org/wiki/Wikipedia:WikiProject_Council/Directory}}\textsuperscript{,}\footnote{\url{https://en.wikipedia.org/wiki/Wikipedia:WikiProject_Council/Guide}}
Individual projects typically maintain their own lists of objectives
and articles in progress.
%
Users are encouraged to recycle existing works that are compatible
with the Wikimedia-wide CC-By-SA license, and the mission of the
respective sites (e.g.~books on Wikibooks or Wikisource, dictionary
entries on Wiktionary, encyclopedic writing on Wikipedia, etc.).  Some
WikiProjects exist purely to help re-purpose existing works in this
way.  On the downstream side, DBPedia is an important resource for the
semantic web, built by collating data from Wikipedia's
``infoboxes''.\footnote{\url{http://wiki.dbpedia.org/}} Researchers
have been able to \patternname{Reduce, reuse, recycle} in other ways,
e.g.~by developing tools for building learning paths through Wikipedia
content, or to show heatmaps of editing activity.  However, these
research interventions do not always result in something accessible to
day-to-day users. There are lots of blogs and feeds around the
Wikimedia project that comprise an elaborate \patternname{Wrapper}
function, but inevitably this could be improved.
%%   Wikimedia as a community
%% seems to be good at building resources, but not always giving exposure
%% to contributors.
% 
The yearly in-person gathering, Wikimania, provides the clearest
example of a \patternname{Heartbeat} for the Wikimedia movement. Also
of note is the twice-yearly call for proposals for individual
engagement
grants.\footnote{\url{https://meta.wikimedia.org/wiki/Grants:IEG}}

%``we are often blinded
%by our ... preferences'' -- In everything that's written here, there
%are certain phrases that I especially respond to.
%
%% Buzzwords or things that are relevant, things that feel useful --
%% through these statements, I feel a connection, can start to plug in.
%% If you had a paper like this, and each person could log in, and
%% highlight the lines they liked, they might find out how much they
%% resonated with other people with similar interests and concerns--
%% there are many issues there and many ways to plug in.
%
%re the five pillars of wikipedia: number 2 neutrality.
Wikipedia may emphasize a neutral point of view, but its users are not
neutral.
%
%% the phrase "objective understanding" doesn't mean you don't care,
%% it just means that you know what you care about!  CD: Independent
%% news sources can get this really twisted.  MH: "I'm not neutral
%% about children's lives."  "Because if we don't they are in
%% trouble."
%
%% ``Objective'' means to push off of -- we know what we want and what
%% we don't want.  If you want to guard the kids, then you, as a person,
%% can see 10000 things that might go wrong -- although you don't have
%% to think about each of the 10000 things We're not bullshitting, we're
%% talking about things that matter to us!  
%
Wikipedia is relevant to things that matter to us.  It helps inform us
regarding our necessary purposes -- and we are invited to ``speak up''
by making edits to pages that matter to us.  In theory, this builds
Wikipedia's \patternname{Carrying capacity} -- however, editor
engagement has been a significant concern in recent years, with
falling numbers of new editors and lower retention
rates.\footnote{\url{https://strategy.wikimedia.org/wiki/Editor_Trends_Study/Results}}
%
%% re: number 4:ethics. yes. ethics means civil solutions. As in
%% peaceably, agreeably as humanly possible, not violently.



% Ladies Hall, South Dormitory, University Hall, Assembly Halls \& Library, North Dormitory, Science Hall, President's Residence, University Farm, and Washburn Observatory.
\subsection{The future university.}
As a design exercise, it is useful to compare and contrast the
patterns we have introduced with the typical or stereotypical image of
a university from Figure \ref{madison-map}.  To be clear, we are not
suggesting that our immediate next steps would realise the design we
sketch here, but it gives us something to aim for in the long term.
%
Existing projects like Wikimedia's Wikiversity and the Peer-2-Peer
University (P2PU) have created ``a model for lifelong learning
alongside traditional formal higher
education''\footnote{\url{https://www.p2pu.org/en/}} but stop well
short of offering accredited degrees.  What would an \emph{accredited}
free/libre/open university look like?

Most likely it would not have a separate Ladies Hall, since we
subscribe to the view that ``there shall be no women in case there be
not men, nor men in case there be not women'' \cite[Chapter
  1.LII]{rabelais1894gargantua}.  However, in light of the extreme
gender imbalance in free software, and still striking imbalance at
Wikipedia \cite{gender,FM4291}, it will be important to do whatever it
takes to make women and girls welcome, not least because this is a
significant factor in boosting our \patternname{Carrying capacity}.
Dormitories may be seen as an ``optional extra,'' since studying from
where you live is often an option already.  However,
cooperatively-owned living/working spaces may frequently be an asset
for \patternname{A specific project}.  In-person meetings for
special-purpose assemblies (such as conventions, dances, and
commencement ceremonies) could comprise an important part of the
project's \patternname{Wrapper}.  While there is no immediate need for
fixed hard infrastructure, that could come in time.  A special
president’s residence seems like an undue oppulence, but at some point
it could be practical to have a University Hall in which to gather for
the purpose of working on the project's \patternname{Roadmap} and
\patternname{Scrapbook}.

For now, there are enough knowledge resources and collaboration tools
available online to keep us busy.  Indeed, these factors are what make
a low-cost, high-quality, formally-accredited future university
conceivable.  However, the available resources are not always as
organized as they would need to be for educative purposes, so putting
effort into \patternname{Reduce, reuse, recycle}'ing them into more
coherent forms is one place where peeragogues could usefully apply
effort.  It will be pragmatic to connect ``research
\patternnameplural{Newcomer}'' with employment, so that the future
university may see a closer coupling of science and industry than is
held in the former model.  We are inspired by the London-based
freelancing cooperative Founders\&Coders, which is able to offer
intensive training in web development at no cost, following a
competitive application process, on the basis that some trainees will
choose to join the cooperative as paying members later
on.\footnote{\url{http://www.foundersandcoders.com/academy/}}

Although it may sound quaint, working farms could help to physically
sustain peeragogues, while putting the project's
\patternname{Heartbeat} in tune with that of the seasons.  In the
current distributed mode, we can settle for windowboxes and
allotments.  As for an observatory, that will prove to be important,
if \patternname{Peeragogy} is truly to reach \emph{ad astra, per
  aspera}.

