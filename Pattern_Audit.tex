\section{Pattern Audit Routine}\label{sec:Pattern_Audit_Routine}

\subsubsection*{Context} As a collection of patterns grows it is important to "prune" them to make sure they are up to date and do not lose relevance.

\subsubsection*{Problem} This becomes confusing. Not all of the patterns are equally relevant and some will become completely irrelevant as a project evolves.

\subsubsection*{Solution} Periodically run a pattern audit. Bring in a real-time aspect using the \html{http://metameso.org/~joe/docs/The-Paragogical-Action-Review.pdf}{Paragogical Action Review}. 

\begin{enumerate}
\item Review what was supposed to happen
\item Establish what is happening/happened
\item Determine what’s right and wrong with what we are doing/have done
\item What did we learn or change?
\item What else should we change going forward?
\end{enumerate}

After asking these five questions of any pattern, it will likely become clearer and/or show its irrelevance. For each pattern go throuh these Any patterns that cannot be made relevant for our current interests can be moved to the \emph{Scrapbook}

\subsubsection*{Rationale} We want to keep the attention focused on the most relevant issues.

\subsubsection*{Resolution} Regular evaluation helps us improve the pattern catalog and describe our effort to focus.

\subsubsection*{What's Next} Regularly revise these \emph{Audit} pattern in our meetings.