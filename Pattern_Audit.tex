\section{Pattern Audit Routine}\label{sec:Pattern_Audit_Routine}

\subsubsection*{Context} As a collection of patterns grows, some of them may lose relevance.

\subsubsection*{Problem} Not all of the patterns are equally relevant and some will become completely irrelevant as a project evolves.

\subsubsection*{Solution} It is important to ``tune'' and ``prune'' the collection of patterns.  Bring a real-time aspect to debugging individual patterns using the following five-part ``\href{http://metameso.org/~joe/docs/The-Paragogical-Action-Review.pdf}{Paragogical Action Review}'' \cite[Chapter 28]{peeragogy-handbook}:

\begin{enumerate}
\item Review what was supposed to happen.
\item Establish what is happening/happened.
\item Determine what’s right and wrong with what we are doing/have done.
\item What did we learn or change?
\item What else should we change going forward?
\end{enumerate}

After asking these five questions with respect to progress made with any pattern, the pattern will likely become clearer and/or show its irrelevance.  Periodically run a full audit of the pattern catalog.   After a thorough review, any patterns that cannot be revised to become relevant for our current interests can be moved to the \patternname{Scrapbook}.

\subsubsection*{Rationale} We want to keep the attention focused on the most relevant issues.

\subsubsection*{Resolution} This pattern reminds us to improve our patterns, and presents a method for refocusing.

\subsubsection*{What's Next} Regularly go through the \patternname{Pattern Audit Routine} in our future meetings.
