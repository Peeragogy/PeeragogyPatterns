\section{Conclusion}\label{sec:Conclusion}

In this paper we have drawn on our shared experiences in the Peeragogy
project and our individual experiences in other collaborations, and
have taken whatever we can from the collected wisdom shared by expert
collaborators.  Our aim has been to outline a new approach to
education drawing on the principles of free/open/libre software and
``open culture.''  In order to do this, we have have attempted to
offer a clear discussion of these principles, rooted in an actionable,
hands-on way of thinking about things.

The potential societal benefits of a free/open/libre approach to
learning and education seem huge.  This approach has certainly served
us well over the course of our three-and-a-half year long
collaboration.  However, serious challenges remain.  We envision the
Peeragogy project as mutual aid society for other collaborative
projects.  And yet, many educators are not eager to collaborate, and
many enthusiastic collaborators are not particularly eager to educate.
Nevertheless, there are numerous inspiring examples of peeragogy (by
any other name) -- ranging from institutional forms like the co-op
format undergraduate training at Northeastern University and the
graduate-level entrepreneurship program at UC Berkeley, to more
informal collaborations in hacker-maker spaces and critique groups
around the world.
%, the NSF's Research Experiences for Undergraduates program

We hope that the current paper will be of use to those people who work
in and are interested in nurturing the overlap between ``libre'' and
``learning.''  If we have adopted a somewhat provocative stance toward
those working in and around this intersection, it is not because we
believe we have the right answer, but because we believe debate and
discussion are necessary for progress.


