\section{Conclusion}\label{sec:Conclusion}

% organization, motivation, and quality

Our aim in this paper has been to outline a new approach to education
drawing on the principles of free/open/libre software and ``open
culture.''  In order to do this, we have have attempted to discuss
these principles in an actionable, hands-on way.  Mako Hill suggests
that one recipe for success in peer production projects is to take a
familiar idea (like an encyclopedia or university) and make it easy
for people to participate in building it \cite{almost-wikipedia}.
Here, we have focused on the tacitly-familiar idea of peeragogy, deferring a
more detailed elaboration of next steps in the educational arena to
future work.  We can briefly evaluate the contributions of the current
paper under the headings \emph{organization}, \emph{motivation}, and
\emph{quality} borrowed from \cite{benkler2015peer}.

\vspace{-.25\baselineskip}

\subsubsection*{Organization} 
%% Peer production \emph{decentralizes both goal setting and execution to
%%   networks of individuals or more structured communities} although
%% \emph{governance and hierarchies tend to become more pronounced as
%%   peer production projects mature} (ibid.).  
Our use of design patterns \emph{decentralizes both goal setting and
  execution} \cite{benkler2015peer}, reintegrating structure in the
form of an emergent \patternname{Roadmap} for the Peeragogy project.  We have
aimed to make our discussion general enough to work at varying levels
of scale, degrees of formality, and degrees of integration.

\vspace{-.25\baselineskip}

\subsubsection*{Motivation}  Previous research in peer production does not always consider learning as
a central motivator, although as Philipp Schmidt indicates, learning can generally be found at the core of peer production communities \cite{schmidt+commons-based+2009}.  We hope that our paper will help to nurture the overlap between ``libre'' and ``learning.''

\vspace{-.25\baselineskip}

\subsubsection*{Quality} 
``By intervening in real communities, these efforts achieve a level of external validity that lab-based experiments cannot'' \cite{benkler2015peer}.  Using the ``What's next'' annotation piloted here may be helpful to other design pattern authors who want to use their patterns as a research intervention.

\vspace{-.25\baselineskip}

%% The potential societal benefits of a free/open/libre approach to
%% learning and education seem huge.  This approach has certainly served
%% us well over the course of our three-and-a-half year long
%% collaboration.  However, serious challenges remain.  We envision the
%% Peeragogy project as mutual aid society for other collaborative
%% projects.  And yet, many educators are not eager to collaborate, and
%% many enthusiastic collaborators are not particularly eager to educate.
%% Nonetheless, there are numerous inspiring examples of peeragogy (by
%% any other name) -- ranging from institutional forms like co-op format
%% undergraduate programs to graduate-level or accelerator-sponsored
%% training in entrepreneurship, to more informal collaborations in
%% hacker-maker spaces and other highly diverse coauthoring, critique,
%% and creative groups around the world.
%, the NSF's Research Experiences for Undergraduates program


% from a backyard treehouse to Wikipedia's
%Teahouse.\footnote{\url{https://en.wikipedia.org/wiki/Wikipedia:Teahouse}}

% outperform traditional organizational forms under conditions of widespread access to networked communications technologies, a multitude of motivations driving contributions, and non-rival information capable of being broken down into granular, modular, and easy-to-integrate pieces

% structuring design changes as experiments

% governance and hierarchies tend to become more pronounced as peer production projects mature

% With habituation and practice, internalized prosocial behaviors may lead people to adopt a more, or less, cooperative stance in specific contexts based on their interpretation of the appropriate social practice and its coherence with their self-understanding of how to live well

% Observational work that estimates the effects of motivational drivers across real communities offers an especially promising avenue for future work.

% By focusing only on the projects that successfully mobilize contributors, researchers interested in when peer production occurs or the reasons why it succeeds at producing high quality outputs have systematically selected on their dependent variables. An important direction for peer production research will be to study these failures.

% Benkler (2006) speculated that peer production may be better at producing functional works like operating system software and encyclopedias than creative works like code or art.

% For example, while experienced editors may perceive low quality contributions from inexperienced editors as a nuisance, “newbie” edits also attract the attention of experienced community members to improve popular pages they might otherwise have ignored

% - Atomised learner as individual unit of production?workable model for education.
