\section{Conclusion}\label{sec:Conclusion}

We introduced nine patterns of peeragogy and connected them to
concrete next steps for the Peeragogy project.  In order to
demonstrate the generality of these patterns, we included examples
showing how they manifest in current Wikimedia projects, and 
how the patterns could inform the design of a future university rooted in the
values and methods of peer production.
%
% organization, motivation, and quality
We will close by evaluating these using
three dimensions for analyzing research on peer production, borrowed
from \cite{benkler2015peer}.

\vspace{-.25\baselineskip}

\subsubsection*{Organization} 
Our project management strategy \emph{decentralizes both goal setting
  and execution} \cite{benkler2015peer}, reintegrating structure in
the form of an emergent \patternname{Roadmap}.  We have aimed to make
our discussion general and our methods extensible enough to work at
varied levels of scale and degrees of formality, inside or outside of
institutional frameworks.

\vspace{-.25\baselineskip}

\subsubsection*{Motivation}  The future university may be
the Chartes of programming, but it will have plenty in common with the
bazaar \cite{raymond2001cathedral}.  As P2PU cofounder Philipp Schmidt
indicates, \emph{learning is at the core of peer production
  communities} \cite{schmidt+commons-based+2009}.  Our patterns help
to explicate the way these communities work, but more importantly,
we hope they will foment a culture of learning.

\vspace{-.25\baselineskip}

\subsubsection*{Quality} 
``By intervening in real communities, these efforts achieve a level of
external validity that lab-based experiments cannot''
\cite{benkler2015peer}.  The ``What's next'' annotation piloted here
will be helpful to other design pattern authors who aim to use
patterns as part of a research intervention.  Peer production is not guaranteed to
  out-compete proprietary solutions
\cite{benkler2015peer,free-software-better}.  The potential for
success will depend on the way the problem, or problems, are framed.
This where the strategy for use of design patterns demonstrated in the
current paper holds the greatest promise.

% \vspace{-.25\baselineskip}
%\vspace{.25\baselineskip}


%% The potential societal benefits of a free/open/libre approach to
%% learning and education seem huge.  This approach has certainly served
%% us well over the course of our three-and-a-half year long
%% collaboration.  However, serious challenges remain.  We envision the
%% Peeragogy project as mutual aid society for other collaborative
%% projects.  And yet, many educators are not eager to collaborate, and
%% many enthusiastic collaborators are not particularly eager to educate.
%% Nonetheless, there are numerous inspiring examples of peeragogy (by
%% any other name) -- ranging from institutional forms like co-op format
%% undergraduate programs to graduate-level or accelerator-sponsored
%% training in entrepreneurship, to more informal collaborations in
%% hacker-maker spaces and other highly diverse coauthoring, critique,
%% and creative groups around the world.
%, the NSF's Research Experiences for Undergraduates program


% from a backyard treehouse to Wikipedia's
%Teahouse.\footnote{\url{https://en.wikipedia.org/wiki/Wikipedia:Teahouse}}

% outperform traditional organizational forms under conditions of widespread access to networked communications technologies, a multitude of motivations driving contributions, and non-rival information capable of being broken down into granular, modular, and easy-to-integrate pieces

% structuring design changes as experiments

% governance and hierarchies tend to become more pronounced as peer production projects mature

% With habituation and practice, internalized prosocial behaviors may lead people to adopt a more, or less, cooperative stance in specific contexts based on their interpretation of the appropriate social practice and its coherence with their self-understanding of how to live well

% Observational work that estimates the effects of motivational drivers across real communities offers an especially promising avenue for future work.

% By focusing only on the projects that successfully mobilize contributors, researchers interested in when peer production occurs or the reasons why it succeeds at producing high quality outputs have systematically selected on their dependent variables. An important direction for peer production research will be to study these failures.

% Benkler (2006) speculated that peer production may be better at producing functional works like operating system software and encyclopedias than creative works like code or art.

% For example, while experienced editors may perceive low quality contributions from inexperienced editors as a nuisance, “newbie” edits also attract the attention of experienced community members to improve popular pages they might otherwise have ignored

% - Atomised learner as individual unit of production?workable model for education.
