

%% Architectual maverick Christopher Alexander asked the following
%% questions to an audience of computer programmers:
%% \begin{quote}
%% ``What is the Chartres of programming? What task is at a high enough level to inspire people writing programs, to reach for the stars?'' \cite{alexander1999origins}
%% \end{quote}

%% The future of learning seems to be a sufficient challenge to engage
%% the minds of programmers, designers, educators and the public at
%% large.  However it is as far from a \emph{``set up'' ready-made problem}
%% \cite[p.~15]{deleuze1991bergsonism} handed down from society as one could hope for.

\begingroup
\setlength{\columnsep}{5pt}%

\begin{wrapfigure}{r}{.38\textwidth}
\vspace{-2cm}
\hspace{.15cm}
\resizebox{.37\textwidth}{!}{
\begin{tikzpicture}[dot/.style={circle,inner sep=1pt,fill,name=#1}]
\draw[thick] (0,.5) rectangle (10,10);
\node (assess) at (5, 9.55) {{\Large {\sc Assess}}};
\node (organize) at (5, 1.2) {{\Large {\sc Organize}}};
\node (convene)[text width=2cm,align=center,rotate=270] at (8.8, 5.4) {{\Large {\sc Convene}}};
\node (cooperate)[text width=15cm,align=center,rotate=90] at (1, 5.2) {{\Large {\sc Cooperate}}};

%% \node(legend)[draw,rectangle,text width=3cm] at (9.25,.75) {\begin{tabular}{p{1.1in}}
%% \textbf{Legend}\\ \hline\vspace{-2mm} \textbf{A}\hspace{.41in}\textbf{B}\\
%% if pattern \textbf{A} refers to pattern \textbf{B}.
%%   \end{tabular}};

%% \draw[draw=none,dashed] ([xshift=5mm,yshift=1.75mm]legend.west) -- ([xshift=-18mm,yshift=1.75mm]legend.east);
%%%%%%%%%%%%%%%%%%%%%%%%%%%%%%%%%%%%%%%%%%%%%%%%%%%%%%%%%%%%%%%%%%%%%%%%%%%%%%%%%%%%%%%%%%%%%%%%%%%%%
\node[below = 3cm of assess] (roadmap) {\hyperref[sec:Roadmap]{\raisebox{.3mm}{{\icon \symbol{"0021D4}}} \hspace{-.2mm}{\icon \symbol{"0021A6}} {\icon \symbol{"0021C2}}}};
\node (reduce) at (5, 8.8) {\hyperref[sec:Reduce, reuse, recycle]{{\icon \symbol{"002159}} {\icon \symbol{"00219B}} {\icon \symbol{"00219E}}}};
\node (carryingcapacity) at (1.75, 7.15) {\hyperref[sec:Carrying capacity]{{\icon \symbol{"002194}} {\icon \symbol{"0021D7}}}};
\node[below = 3.2cm of carryingcapacity] (heartbeat) {\hyperref[sec:Heartbeat]{{\icon \symbol{"002185}} {\icon \symbol{"0021A8}}}};
\node (aspecificproject) at (8, 6.5) {\hyperref[sec:A specific project]{{\icon \symbol{"0021A2}} {\icon \symbol{"00213C}}}};
\node[below = 1.2cm of roadmap] (wrapper) {\hyperref[sec:Wrapper]{{\icon \symbol{"002136}} {\icon \symbol{"0021B2}} {\icon \symbol{"0021BD}}}};
\node (newcomer) at (8, 3.25) {\hyperref[sec:Newcomer]{{\icon \symbol{"0021E5}} {\icon \symbol{"002180}}}};
\node[below = 1.9cm of wrapper] (scrapbook) {\hyperref[sec:Scrapbook]{{\icon \symbol{"002168}} {\icon \symbol{"0021B9}} {\icon \symbol{"00214C}}}};
\node[above = 1cm of aspecificproject] (peeragogyproject) {\hyperref[sec:Peeragogy]{{\icon \symbol{"00220A}} {\icon \symbol{"002158}}}};
%%%%%%%%%%%%%%%%%%%%%%%%%%%%%%%%%%%%%%%%%%%%%%%%%%%%%%%%%%%%%%%%%%%%%%%%%%%%%%%%%%%%%%%%%%%%%%%%%%%%%
\draw[draw=black,dashed,name path=line 1] (peeragogyproject) -- (aspecificproject)node[text opacity=0]                    {1};
\draw[draw=black,dashed,name path=line 2] (aspecificproject) -- (roadmap)node[text opacity=0]                             {2};
\draw[draw=black,dashed,name path=line 3] (aspecificproject.230) to[out=250,in=40] ([xshift=1mm]scrapbook.60)node[text opacity=0]        {3};
\draw[draw=black,dashed,name path=line 4] (aspecificproject) -- (carryingcapacity)node[text opacity=0]                    {4};
\draw[draw=black,dashed,name path=line 5] (carryingcapacity.335) -- (newcomer.160)node[text opacity=0]                    {5};
\draw[draw=black,dashed,name path=line 6] (carryingcapacity.343) -- ([xshift=1mm,yshift=-1mm]roadmap.150)node[text opacity=0]         {6};
\draw[draw=black,dashed,name path=line 7] (carryingcapacity) -- ([xshift=.4mm]peeragogyproject.west)node[text opacity=0]                    {7};
\draw[draw=black,dashed,name path=line 8] ([xshift=1mm]carryingcapacity.south) -- (scrapbook.140)node[text opacity=0]     {8};
\draw[draw=black,dashed,name path=line 9] (heartbeat) -- ([xshift=.4mm]aspecificproject.210)node[text opacity=0]                       {9};
\draw[draw=black,dashed,name path=line 10] (heartbeat) -- (carryingcapacity)node[text opacity=0]                          {10};
\draw[draw=black,dashed,name path=line 11] (heartbeat) -- (scrapbook.155)node[text opacity=0]                             {11};
\draw[draw=black,dashed,name path=line 12] (heartbeat) -- (reduce.225)node[text opacity=0]                                {12};
\draw[draw=black,dashed,name path=line 13] (newcomer) -- ([xshift=4mm]reduce.south)node[text opacity=0]                   {13};
\draw[draw=black,dashed,name path=line 14] (newcomer) -- (aspecificproject)node[text opacity=0]                           {14};
\draw[draw=black,dashed,name path=line 15] (newcomer) -- (roadmap.330)node[text opacity=0]                                {15};
\draw[draw=black,dashed,name path=line 16] (newcomer) -- (scrapbook.24)node[text opacity=0]                               {16};
\draw[draw=black,dashed,name path=line 17] (roadmap) -- ([xshift=1.2mm]peeragogyproject.205)node[text opacity=0]                        {17};
\draw[draw=black,dashed,name path=line 18] ([xshift=1.5mm,yshift=.5mm]roadmap.200) -- (heartbeat)node[text opacity=0]                               {18};
\draw[draw=black,dashed,name path=line 19] (scrapbook) -- (wrapper)node[text opacity=0]                                   {19};
\draw[draw=black,dashed,name path=line 20] (scrapbook.110) to[out=123,in=240] ([yshift=.4mm,xshift=.2mm]reduce.245) node[text opacity=0]            {20};
\draw[draw=black,dashed,name path=line 21] (scrapbook.70) to[out=43,in=310] (roadmap.315) node[text opacity=0]             {21};
\draw[draw=black,dashed,name path=line 22] (reduce) -- (carryingcapacity)node[text opacity=0]                             {22};
\draw[draw=black,dashed,name path=line 23] ([xshift=.7mm]reduce.270) -- ([xshift=.7mm,yshift=-.4mm]roadmap.90)node[text opacity=0]                                      {23};
\draw[draw=black,dashed,name path=line 24] ([xshift=.7mm]wrapper.180) -- (heartbeat.10)node[text opacity=0]               {24};
\draw[draw=black,dashed,name path=line 25] (wrapper.355) -- (newcomer.180)node[text opacity=0] {25};
\draw[draw=black,dashed,name path=line 26] (wrapper) -- ([xshift=2.3mm]carryingcapacity.south)node[text opacity=0]        {26};
\draw[draw=black,dashed,name path=line 27] (wrapper) -- (roadmap)node[text opacity=0]                                     {27};

%% \fill[fill=none,name intersections={of=line 7 and line 23,total=\t}]
%%     \foreach \s in {1,...,\t}{(intersection-\s) circle (2pt) node {$\star$}};

%% \fill[fill=none,name intersections={of=line 12 and line 7,total=\t}]
%%     \foreach \s in {1,...,\t}{(intersection-\s) circle (2pt) node {$\star$}};

%% \fill[fill=none,name intersections={of=line 20 and line 7,total=\t}]
%%     \foreach \s in {1,...,\t}{(intersection-\s) circle (2pt) node {$\star$}};

%% \fill[fill=none,name intersections={of=line 13 and line 7,total=\t}]
%%     \foreach \s in {1,...,\t}{(intersection-\s) circle (2pt) node {$\star$}};

%% \fill[fill=none,name intersections={of=line 12 and line 4,total=\t}]
%%     \foreach \s in {1,...,\t}{(intersection-\s) circle (2pt) node {$\star$}};

%% \fill[fill=none,name intersections={of=line 12 and line 6,total=\t}]
%%     \foreach \s in {1,...,\t}{(intersection-\s) circle (2pt) node {$\star$}};

%% \fill[fill=none,name intersections={of=line 12 and line 26,total=\t}]
%%     \foreach \s in {1,...,\t}{(intersection-\s) circle (2pt) node {$\star$}};

%% \fill[fill=none,name intersections={of=line 12 and line 8,total=\t}]
%%     \foreach \s in {1,...,\t}{(intersection-\s) circle (2pt) node {$\star$}};

%% \fill[fill=none,name intersections={of=line 18 and line 8,total=\t}]
%%     \foreach \s in {1,...,\t}{(intersection-\s) circle (2pt) node {$\star$}};

%% \fill[fill=none,name intersections={of=line 24 and line 8,total=\t}]
%%     \foreach \s in {1,...,\t}{(intersection-\s) circle (2pt) node {$\star$}};

%% \fill[fill=none,name intersections={of=line 9 and line 8,total=\t}]
%%     \foreach \s in {1,...,\t}{(intersection-\s) circle (2pt) node {$\star$}};

%% \fill[fill=none,name intersections={of=line 4 and line 23,total=\t}]
%%     \foreach \s in {1,...,\t}{(intersection-\s) circle (2pt) node {$\star$}};

%% \fill[fill=none,name intersections={of=line 4 and line 17,total=\t}]
%%     \foreach \s in {1,...,\t}{(intersection-\s) circle (2pt) node {$\star$}};

%% \fill[fill=none,name intersections={of=line 4 and line 13,total=\t}]
%%     \foreach \s in {1,...,\t}{(intersection-\s) circle (2pt) node {$\star$}};

%% \fill[fill=none,name intersections={of=line 17 and line 13,total=\t}]
%%     \foreach \s in {1,...,\t}{(intersection-\s) circle (2pt) node {$\star$}};

%% \fill[fill=none,name intersections={of=line 2 and line 13,total=\t}]
%%     \foreach \s in {1,...,\t}{(intersection-\s) circle (2pt) node {$\star$}};

%% \fill[fill=none,name intersections={of=line 9 and line 13,total=\t}]
%%     \foreach \s in {1,...,\t}{(intersection-\s) circle (2pt) node {$\star$}};

%% \fill[fill=none,name intersections={of=line 3 and line 13,total=\t}]
%%     \foreach \s in {1,...,\t}{(intersection-\s) circle (2pt) node {$\star$}};

%% \fill[fill=none,name intersections={of=line 3 and line 15,total=\t}]
%%     \foreach \s in {1,...,\t}{(intersection-\s) circle (2pt) node {$\star$}};

%% \fill[fill=none,name intersections={of=line 3 and line 5,total=\t}]
%%     \foreach \s in {1,...,\t}{(intersection-\s) circle (2pt) node {$\star$}};

%% \fill[fill=none,name intersections={of=line 3 and line 25,total=\t}]
%%     \foreach \s in {1,...,\t}{(intersection-\s) circle (2pt) node {$\star$}};

%% \fill[fill=none,name intersections={of=line 26 and line 18,total=\t}]
%%     \foreach \s in {1,...,\t}{(intersection-\s) circle (2pt) node {$\star$}};

%% \fill[fill=none,name intersections={of=line 15 and line 9,total=\t}]
%%     \foreach \s in {1,...,\t}{(intersection-\s) circle (2pt) node {$\star$}};

%% \fill[fill=none,name intersections={of=line 20 and line 4,total=\t}]
%%     \foreach \s in {1,...,\t}{(intersection-\s) circle (2pt) node {$\star$}};

%% \fill[fill=none,name intersections={of=line 20 and line 6,total=\t}]
%%     \foreach \s in {1,...,\t}{(intersection-\s) circle (2pt) node {$\star$}};

%% \fill[fill=none,name intersections={of=line 20 and line 26,total=\t}]
%%     \foreach \s in {1,...,\t}{(intersection-\s) circle (2pt) node {$\star$}};

%% \fill[fill=none,name intersections={of=line 20 and line 24,total=\t}]
%%     \foreach \s in {1,...,\t}{(intersection-\s) circle (2pt) node {$\star$}};

%% \fill[fill=none,name intersections={of=line 20 and line 9,total=\t}]
%%     \foreach \s in {1,...,\t}{(intersection-\s) circle (2pt) node {$\star$}};

%% \fill[fill=none,name intersections={of=line 20 and line 18,total=\t}]
%%     \foreach \s in {1,...,\t}{(intersection-\s) circle (2pt) node {$\star$}};

%% \fill[fill=none,name intersections={of=line 5 and line 9,total=\t}]
%%     \foreach \s in {1,...,\t}{(intersection-\s) circle (2pt) node {$\star$}};

%% \fill[fill=none,name intersections={of=line 5 and line 27,total=\t}]
%%     \foreach \s in {1,...,\t}{(intersection-\s) circle (2pt) node {$\star$}};

%% \fill[fill=none,name intersections={of=line 9 and line 27,total=\t}]
%%     \foreach \s in {1,...,\t}{(intersection-\s) circle (2pt) node {$\star$}};

%% \fill[fill=none,name intersections={of=line 21 and line 25,total=\t}]
%%     \foreach \s in {1,...,\t}{(intersection-\s) circle (2pt) node {$\star$}};

%% \fill[fill=none,name intersections={of=line 21 and line 5,total=\t}]
%%     \foreach \s in {1,...,\t}{(intersection-\s) circle (2pt) node {$\star$}};

%% \fill[fill=none,name intersections={of=line 21 and line 9,total=\t}]
%%     \foreach \s in {1,...,\t}{(intersection-\s) circle (2pt) node {$\star$}};

%% \fill[fill=none,name intersections={of=line 26 and line 9,total=\t}]
%%     \foreach \s in {1,...,\t}{(intersection-\s) circle (2pt) node {$\star$}};
\end{tikzpicture}

}
%\hspace{.1cm}
\vspace{-2cm}
\captionsetup{font=footnotesize,width=.45\textwidth,margin={0cm,.15cm}}
\caption{Mnemonic \label{mnemonic}}
\vspace{-.5cm}
\end{wrapfigure}

This paper presents nine patterns of peeragogy and connects them to
concrete next steps for the Peeragogy project.  In order to
demonstrate generality, we included examples that illustrate how the
patterns manifest in current Wikimedia projects, and how they could
inform the design of a future university rooted in the values and
methods of peer production.
%
% organization, motivation, and quality
The university metaphor need not constrain the future application of
the ideas, which cut across many modes of engagement (Figure \ref{mnemonic}).
Even so, a project to translate the free\slash libre\slash open
university from metaphor to reality would offer an intriguing
opportunity to test the generality of Hill's hypothesis on the
mobilizing potential of ``the combination of a familiar goal (e.g.,
`simply reproduce Encyclopedia Britannica') with innovative methods
(e.g., `anybody can edit anything')'' \cite[p.~13]{mako-thesis}.  We
close by reviewing our approach to the organization of learning, using
three dimensions of analysis that have been previously applied to
describe research on peer production \cite{benkler2015peer}.

\endgroup

\vspace{-.25\baselineskip}

\subsubsection*{Organization} 
Managing work on our project with design patterns that are augmented
with a ``What's next'' follow-through step allows us to ``set and execute goals in a decentralized manner'' \cite{benkler2015peer}, reintegrating
structure in the form of an emergent \patternname{Roadmap}.  Our
methods apply at varied levels of scale and degrees of formality,
inside or outside of institutional frameworks.  If you already write
patterns, you can add a ``What's next'' step to them and try the
approach for yourself.

\vspace{-.25\baselineskip}

\subsubsection*{Motivation}  The future of learning may be
the \emph{Chartres of programming} \cite{alexander1999origins}, but it will have plenty in common with the
bazaar \cite{raymond2001cathedral}.
%
Philipp Schmidt indicates that \emph{learning is at the core of peer
  production communities} \cite{schmidt+commons-based+2009}.  Our
patterns help to explicate the way these communities work, but more
importantly, we hope these patterns will help potentiate a global
culture of collaborative learning, inside and outside of institutions.

\vspace{-.25\baselineskip}

\subsubsection*{Quality} 
``By intervening in real communities, these efforts achieve a level of
external validity that lab-based experiments cannot''
\cite{benkler2015peer}.  The ``What's next'' annotation piloted here
will be helpful to other design pattern authors who aim to use
patterns as part of a research intervention.  Peer production is not guaranteed to
  out-compete proprietary solutions
\cite{benkler2015peer,free-software-better}: its potential for
success will depend on the way the problems are framed,
and our ability to follow through.

\subsection*{Acknowledgments}
We thank our PLoP shepherd David Kane, on-site shepherd Philipp
Bachmann, and workshop facilitator Mary Lynn Manns.  Marian Petre
offered helpful and motivating comments on a much earlier version of
several of the patterns.  Amanda Lyons and Fabrizio Terzi contributed
the images used in Figure \ref{dashboard}.  Photographs and icons were
sourced from the public domain, via the Wikimedia Commons and the
Symbola map-makers font.
%% respectively.


% \vspace{-.25\baselineskip}
%\vspace{.25\baselineskip}


%% The potential societal benefits of a free/open/libre approach to
%% learning and education seem huge.  This approach has certainly served
%% us well over the course of our three-and-a-half year long
%% collaboration.  However, serious challenges remain.  We envision the
%% Peeragogy project as mutual aid society for other collaborative
%% projects.  And yet, many educators are not eager to collaborate, and
%% many enthusiastic collaborators are not particularly eager to educate.
%% Nonetheless, there are numerous inspiring examples of peeragogy (by
%% any other name) -- ranging from institutional forms like co-op format
%% undergraduate programs to graduate-level or accelerator-sponsored
%% training in entrepreneurship, to more informal collaborations in
%% hacker-maker spaces and other highly diverse coauthoring, critique,
%% and creative groups around the world.
%, the NSF's Research Experiences for Undergraduates program


% from a backyard treehouse to Wikipedia's
%Teahouse.\footnote{\url{https://en.wikipedia.org/wiki/Wikipedia:Teahouse}}

% outperform traditional organizational forms under conditions of widespread access to networked communications technologies, a multitude of motivations driving contributions, and non-rival information capable of being broken down into granular, modular, and easy-to-integrate pieces

% structuring design changes as experiments

% governance and hierarchies tend to become more pronounced as peer production projects mature

% With habituation and practice, internalized prosocial behaviors may lead people to adopt a more, or less, cooperative stance in specific contexts based on their interpretation of the appropriate social practice and its coherence with their self-understanding of how to live well

% Observational work that estimates the effects of motivational drivers across real communities offers an especially promising avenue for future work.

% By focusing only on the projects that successfully mobilize contributors, researchers interested in when peer production occurs or the reasons why it succeeds at producing high quality outputs have systematically selected on their dependent variables. An important direction for peer production research will be to study these failures.

% Benkler (2006) speculated that peer production may be better at producing functional works like operating system software and encyclopedias than creative works like code or art.

% For example, while experienced editors may perceive low quality contributions from inexperienced editors as a nuisance, “newbie” edits also attract the attention of experienced community members to improve popular pages they might otherwise have ignored

% - Atomised learner as individual unit of production?workable model for education.
