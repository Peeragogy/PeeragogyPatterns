\section{Conclusion}\label{sec:Conclusion}

%% Architectual maverick Christopher Alexander asked the following
%% questions to an audience of computer programmers:
%% \begin{quote}
%% ``What is the Chartres of programming? What task is at a high enough level to inspire people writing programs, to reach for the stars?'' \cite{alexander1999origins}
%% \end{quote}

%% The future of learning seems to be a sufficient challenge to engage
%% the minds of programmers, designers, educators and the public at
%% large.  However it is as far from a \emph{``set up'' ready-made problem}
%% \cite[p.~15]{deleuze1991bergsonism} handed down from society as one could hope for.

This paper presents nine patterns of peeragogy and connects them to
concrete next steps for the Peeragogy project.  In order to
show the generality of these patterns, we included examples
showing how they manifest in current Wikimedia projects, and 
how they could inform the design of a future university rooted in the
values and methods of peer production.
%
% organization, motivation, and quality
%
The university metaphor that has guided our inquiry need not
constrain the future application of the ideas.
%
We close by reviewing our distributed
and emergent approach to the organization of learning with three
dimensions of analysis that have been previously applied to describe
research on peer production \cite{benkler2015peer}.

\vspace{-.25\baselineskip}

\subsubsection*{Organization} 
Managing work on our project with design patterns that are augmented
with a ``What's next'' follow-through step \emph{decentralizes both
  goal setting and execution} \cite{benkler2015peer}, reintegrating
structure in the form of an emergent \patternname{Roadmap}.  Our
methods apply at varied levels of scale and degrees of formality,
inside or outside of institutional frameworks.  If you already write
patterns, you can add a ``What's next'' step to them to try the
approach for yourself.

\vspace{-.25\baselineskip}

\subsubsection*{Motivation}  The future of learning may be
the \emph{Chartes of programming} \cite{alexander1999origins}, but it will have plenty in common with the
bazaar \cite{raymond2001cathedral}.
%
Philipp Schmidt indicates that \emph{learning is at the core of peer
  production communities} \cite{schmidt+commons-based+2009}.  Our
patterns help to explicate the way these communities work, but more
importantly, we hope they will potentiate a global culture of
learning, inside and outside of institutions.

\vspace{-.25\baselineskip}

\subsubsection*{Quality} 
``By intervening in real communities, these efforts achieve a level of
external validity that lab-based experiments cannot''
\cite{benkler2015peer}.  The ``What's next'' annotation piloted here
will be helpful to other design pattern authors who aim to use
patterns as part of a research intervention.  Peer production is not guaranteed to
  out-compete proprietary solutions
\cite{benkler2015peer,free-software-better}: its potential for
success will depend on the way our problems are framed,
and our ability to follow through.

% \vspace{-.25\baselineskip}
%\vspace{.25\baselineskip}


%% The potential societal benefits of a free/open/libre approach to
%% learning and education seem huge.  This approach has certainly served
%% us well over the course of our three-and-a-half year long
%% collaboration.  However, serious challenges remain.  We envision the
%% Peeragogy project as mutual aid society for other collaborative
%% projects.  And yet, many educators are not eager to collaborate, and
%% many enthusiastic collaborators are not particularly eager to educate.
%% Nonetheless, there are numerous inspiring examples of peeragogy (by
%% any other name) -- ranging from institutional forms like co-op format
%% undergraduate programs to graduate-level or accelerator-sponsored
%% training in entrepreneurship, to more informal collaborations in
%% hacker-maker spaces and other highly diverse coauthoring, critique,
%% and creative groups around the world.
%, the NSF's Research Experiences for Undergraduates program


% from a backyard treehouse to Wikipedia's
%Teahouse.\footnote{\url{https://en.wikipedia.org/wiki/Wikipedia:Teahouse}}

% outperform traditional organizational forms under conditions of widespread access to networked communications technologies, a multitude of motivations driving contributions, and non-rival information capable of being broken down into granular, modular, and easy-to-integrate pieces

% structuring design changes as experiments

% governance and hierarchies tend to become more pronounced as peer production projects mature

% With habituation and practice, internalized prosocial behaviors may lead people to adopt a more, or less, cooperative stance in specific contexts based on their interpretation of the appropriate social practice and its coherence with their self-understanding of how to live well

% Observational work that estimates the effects of motivational drivers across real communities offers an especially promising avenue for future work.

% By focusing only on the projects that successfully mobilize contributors, researchers interested in when peer production occurs or the reasons why it succeeds at producing high quality outputs have systematically selected on their dependent variables. An important direction for peer production research will be to study these failures.

% Benkler (2006) speculated that peer production may be better at producing functional works like operating system software and encyclopedias than creative works like code or art.

% For example, while experienced editors may perceive low quality contributions from inexperienced editors as a nuisance, “newbie” edits also attract the attention of experienced community members to improve popular pages they might otherwise have ignored

% - Atomised learner as individual unit of production?workable model for education.
