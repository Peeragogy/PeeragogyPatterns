\begin{wrapfigure}{l}{.42\textwidth}
\vspace{-.3cm}
\begin{center}
\includegraphics[width=.4\textwidth]{figures/Duchamp_Fountaine/Duchamp_Fountaine.jpg}
\end{center}
\caption{A paradigmatic example of found-art. Caption reads: ``Fountain by R. Mutt, Photograph by Alfred Stieglitz, THE EXHIBIT REFUSED BY THE INDEPENDENTS''. Public domain, via the Wikimedia Commons.\label{fountain}}
\vspace{-.9cm}
\end{wrapfigure}

\section{Reduce, reuse, recycle} \label{sec:Reduce, reuse, recycle}
% DK: I am not sure about the [old] title of this pattern. The bulk of the body of the pattern seems to be about the reuse. Maybe something like “Don’t Make what you can Use” might fit the spirit better. [fixed]

\subsubsection*{Context}
% DK: This seems like an explanation of the title, not a ``context'' in which a problem is observed.
In a peer production context, you are simultaneously ``making stuff'' and building on the work of others. 

\subsubsection*{Problem}
People are often very attached to their own projects and priorities and don't have a sense of how their initiatives can benefit from connection and relationship.  Many projects die because the cost of \patternnameext{\href{http://c2.com/cgi/wiki?ReinventingTheWheel}{Reinventing the Wheel}} [c2] is too high.  

\subsubsection*{Solution} While it's a great idea to ``steal like an artist'' don't forget to return the favor and make it possible for other people to build on your work too (Figure \ref{fountain}).  In the Peeragogy project, so far we have gotten away with writing very little new software, and have instead used off-the-shelf and hosted solutions suited to the task at hand (including: Drupal, Google+, Google Hangouts, Google Docs, Wordpress, pandoc, XeLaTeX, Authorea, and Github).  We've connected with many different contributors, each with different problems to solve and different ways of moving the project forward.  Early on we agreed to release our \emph{Peeragogy Handbook} under the terms of the Creative Commons Public Domain Dedication (CC0), the legal instrument that grants the greatest possible leeway to downstream users.\footnote{\url{https://creativecommons.org/publicdomain/zero/1.0/}}  This has allowed us and others to repurpose and improve \emph{Handbook} contents in other settings (including the current paper).

\subsubsection*{Rationale} 
Clearly we are not the first people to notice the problems with wheel-reinvention, including ``missing opportunities, repeating common mistakes, and working harder than we need to.''\footnote{\url{https://blog.wikimedia.org/2013/11/19/learning-patterns-new-way-share-important-lessons/}}  As Willow Brugh of Geeks without Bounds and the MIT Media Lab remarked as a guest in one of our hangouts\footnote{\url{https://www.youtube.com/watch?v=NpyQfYVKfBI}}: people often think that they need to build a community, and so fail to recognize that they are already part of a community.   David Kane, this paper's shepherd for PLoP'15, keenly observed that this pattern can be decomposed along the lines of the keywords in its title:  \emph{Reduce} the panoply of interesting interrelated ideas and methods to a functional core (e.g.~writing a book).  \emph{Reuse} whatever resources are relevant to this aim, factoring in ``things I was going to have to do anyway'' from everyone involved.  \emph{Recycle} what you've created in new connections and relationships.  

\subsubsection*{Resolution}  It's worth keeping in mind that peeragogy per se is not new, and it's not something we can bottle and sell. It appears in avocational, academic, and industrial contexts. All the more reason to understand it better!  Furthermore, as much as innovation is celebrated in our culture, there's something to be said for tradition, too.  Reweaving old material into new designs and new material into classic frameworks, we build a deeper understanding.

\begin{framed}
\emph{What's Next.}
We've spun off the pattern catalog from the \emph{Peeragogy Handbook} into this paper, sharing it with a new community and gaining new perspectives.  Let's look for other parts of the handbook we can spin off!
\end{framed}



    
    
