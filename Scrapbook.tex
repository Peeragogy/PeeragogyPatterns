\begingroup \color{BurntOrange}

\section{Scrapbook} \label{sec:Scrapbook}

\subsubsection*{Context} We've maintained and revised our pattern catalog over a period of years.  We're achieving
the ``What's Next'' steps they suggest.

\subsubsection*{Problem} Not all of the patterns are remain equally relevant, and some will become completely irrelevant as a project evolves.  In particular, some of the patterns no longer lead to concrete next steps.

\subsubsection*{Solution} In order to maintain focus, is important to ``tune'' and ``prune'' the collection of patterns receiving active attention.  Connect this to actions undertaken in the project by asking questions like these:
(1) Review what was supposed to happen.
(2) Establish what is happening/happened.
(3) Determine what’s right and wrong with what we are doing/have done.
(4) What did we learn or change? 
(5) What else should we change going forward?  \cite[Chapter 28]{peeragogy-handbook}
%
After reviewing an individual pattern or collection of patterns with
resepect to these issues, priorities will become clearer.  If a
particular pattern is no longer of current relevance, move it into a
scrapbook.\footnote{\url{http://paragogy.net/Pattern_Scrapbook}.}  The
scrapbook can also be used to maintain a backlog of proto-patterns, in
the form of outstanding problems and issues.

\subsubsection*{Rationale} We want to keep the attention focused on the most relevant issues.
We want our collection of patterns to be concretely useful and actively used.
Even if a pattern is not specifically useful or actionable at the
moment, sufficient time for reflection may offer a better
understanding of the essential attributes of the pattern, show when it
is appropriate, and how to implement it.

\subsubsection*{Resolution} 
Judicious use of the \patternname{Scrapbook} can help focus project participants, and can make it easier to present a project to others in a clear manner.  Patterns that are overly abstract, redundant, or that lack a concrete resolution are moved out of focus, and the pattern catalog is leaner and more action-oriented as a result.

\begin{framed}
\emph{What's Next.}
After significantly pruning back the pattern catalog, we want it to grow again: new patterns are needed.  Reviewing the contents of the \patternname{Scrapbook} will be one place to look for inspiration, but there are many others.
\end{framed}

\endgroup
