\section{Scrapbook} \label{sec:Scrapbook}

\subsubsection*{Motivation} This pattern describes another technique for maintaining focus.  

\subsubsection*{Context} We've maintained and revised our pattern catalog over a period of years.  We're achieving
the ``What's Next'' steps attached to some of the patterns.

\subsubsection*{Forces}~
\parbox[t]{.85\textwidth}{
\textbf{Attention}: due to limited energy, we need to ask: where should we set the focus?\\
\textbf{Interest}: new ideas catch our attention\\
\textbf{Meaning}: a history of working on things makes them meaningful.
}

\subsubsection*{Problem} Not all of the patterns we've noticed remain equally relevant.  In particular, some of the patterns no longer lead to concrete next steps.

\subsubsection*{Solution} In order to maintain focus, is important to ``tune'' and ``prune'' the collection of patterns receiving active attention.  Connect this understanding to concrete actions undertaken in the project by frequently asking questions like these:
\begin{quote}
(1) Review what was supposed to happen.
(2) Establish what is happening/happened.
(3) Determine what’s right and wrong with what we are doing/have done.
(4) What did we learn or change? 
(5) What else should we change going forward?  \cite[Chapter 28]{peeragogy-handbook}.
\end{quote}
%
%OSS: Who maintains the scrapbook? ... People say when you're learning, you should retain a learning log. Maybe scrapbook is like a shared notebook? All the process should be shared together, even if people take different paths, its all open. Journal of activities?
After reviewing our activities with respect to these questions, our
current priorities will become clearer.  If a particular pattern is no
longer of current relevance, move it to a
\patternname{Scrapbook}.\footnote{\url{http://paragogy.net/Scrapbook}.}  
%
In addition to retired patterns, use the scrapbook to maintain a
backlog, or ``parking lot,'' of proto-patterns, in the form of
outstanding problems, issues, and concerns.  Don't limit yourself to
\emph{your own} creativity: include bookmarks to or clippings from
patterns from other sources (see \patternname{Reduce, reuse, recycle}).  Ideally many people will contribute by describing their ideas and concerns, but in some cases a designated \patternname{Wrapper} may have to do further work to elicit and organize that material.

\subsubsection*{Rationale} 
We want our pattern catalog to be concretely useful and actively used,
and to keep attention focused on the most relevant issues.
If a pattern is not specifically useful or actionable at the
moment, sufficient time for reflection may offer a better
understanding, or it may prove useful in a different context.

\subsubsection*{Resolution} 
Judicious use of the \patternname{Scrapbook} can help focus project participants' \textbf{attention} on current concerns, without losing grasp of items of \textbf{interest}.  The currently active pattern catalog is leaner and more action-oriented as a result. If the \patternname{Roadmap} shows where we're going, it is the \patternname{Scrapbook} that shows most clearly where we've been, and collects the observations that are most \textbf{meaningful} to us.

\subsubsection*{Example 1} Now that new plans are being formed, the Wikimedia Foundation's previous ``five year plan'' somewhat
resembles a \patternname{Scrapbook} \cite{wikimedia2011plan}.

\subsubsection*{Example 2} 
In the future university, the patterns described here will continue to
shape the landscape, but considerable activity will be focused on new
problems and new patterns -- just as a university campus grows and
changes with the addition of new buildings.

\begin{framed}
\noindent 
\emph{What's Next.}
\definecollection{ScrapbookWN}
\begin{collectinmacro}{\ScrapbookWN}{}{}
After pruning back our pattern catalog, we want it to grow again: new patterns are needed.
One strategy would be to ``patternize'' the rest of the \emph{Peeragogy Handbook.}
\end{collectinmacro}
\ScrapbookWN
\end{framed}


