\section{Scrapbook} \label{sec:Scrapbook}

\subsubsection*{Context} We've maintained and revised our pattern catalog over a period of years.  We're achieving
the ``What's Next'' steps attached to some of the patterns.

\subsubsection*{Problem} Not all of the patterns are remain equally relevant.  In particular, some of the patterns no longer lead to concrete next steps.

\subsubsection*{Solution} In order to maintain focus, is important to ``tune'' and ``prune'' the collection of patterns receiving active attention.  Connect this understanding to concrete actions undertaken in the project by frequently asking questions like these:
(1) Review what was supposed to happen.
(2) Establish what is happening/happened.
(3) Determine what’s right and wrong with what we are doing/have done.
(4) What did we learn or change? 
(5) What else should we change going forward?  \cite[Chapter 28]{peeragogy-handbook}.
%
After reviewing our activities with resepect to these questions, our
current priorities will become clearer.  If a particular pattern is no
longer of current relevance, move it to a
\patternname{Scrapbook}.\footnote{\url{http://paragogy.net/Scrapbook}.}  In addition
to retired patterns, use the scrapbook to maintain a backlog, or
``parking lot,'' of proto-patterns, in the form of outstanding
problems, issues, concerns.  Don't limit yourself to \emph{your own}
creativity: include bookmarks to or clippings from patterns from other
sources (see \patternname{Reduce, Reuse, Recycle}).

\subsubsection*{Rationale} We want to keep attention focused on the most relevant issues.
We want our pattern catalog to be concretely useful and actively used.
If a pattern is not specifically useful or actionable at the
moment, sufficient time for reflection may offer a better
understanding, or it may prove useful in a different context.

\subsubsection*{Resolution} 
Judicious use of the \patternname{Scrapbook} can help focus project participants, and can make it easier to communicate current priorities to others in a clear manner.  The currently active pattern catalog is leaner and more action-oriented as a result.

\begin{framed}
\emph{What's Next.}
After significantly pruning back our pattern catalog, we want it to grow again: new patterns are needed.  Reviewing the contents of the \patternname{Scrapbook} will be one place to look for inspiration, but there are many others.
\end{framed}


