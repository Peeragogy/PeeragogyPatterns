\begingroup \color{OliveGreen}
\section{Wrapper}\label{sec:Wrapper}

\subsubsection*{Context} An active, long-running, and possibly quite complex project with more than a few participants, often including \patternname{Newcomers}.  

\subsubsection*{Problem} In an active project, it can be effectively impossible to stay up to date with all of the details.  Not everyone will be able to attend every meeting (see \patternname{\href{http://peeragogy.org/patterns/heartbeat/}{Heartbeat}}) or read every email.  Project participants can easily get lost and drift away.  The experience can be more difficult for \patternname{Newcomers}: joining an existing project can feel like trying to get aboard a rapidly moving vehicle.  If you've taken time off, you may feel like things have moved on so far that you cannot catch up.  Information overload is not the only concern: there is also problem with missing information.  If they aren't shared, key skills can quickly become bottlenecks; see \patternname{Carrying capacity}.

\subsubsection*{Solution}
% DK: Be more direct.  Don’t say what “can” be done…just say what to do. [also, typo -jc]
Create a wrap-up summary, distinct from other project communications, that make current activities comprehensible to participants who may not have been following all of the details.  Over time, the collected chronicle of activities can give \patternname{Newcomers} a feel for the sort of thing that happens in the project.  In addition, check other public summaries of the project (like the landing page, \patternname{Roadmap}, and documentation) from time to time, to make sure that they accurately represent the facts on the ground.  

\subsubsection*{Rationale}
According to the theory proposed by Yochai Benkler, for free/open ``commons-based'' projects to work, it is vital to have both (1) the ability to contribute small pieces; (2) something that stitches those pieces together \cite{coases-penguin}. The wrapper helps perform this integrative stitching function, which is often much more challenging than the job of breaking things down into pieces or just doing one of the small pieces.

\subsubsection*{Resolution} 
% DK: This sounds like Rationale
Wrap-up summaries can help to engage or re-engage members of a project, and can give an emotional boost to peeragogues who see their contributions and concerns mentioned: it's a sign that someone is paying attention.

\begin{framed}
\emph{What's Next.}
We have prototyped a visual ``dashboard'' that people can access to immediately get an idea of what work is ongoing in the project with links to ways to get further engaged (Figure \ref{dashboard}).  Let's deploy it.
\end{framed}    
\endgroup

\begin{figure}
\includegraphics[width=\textwidth,trim=0mm 135mm 0mm 0mm,clip=true]{figures/peeragogy_dashboard_draft1/peeragogy_dashboard_draft1.jpg}
\caption{Design sketch for possible updated Peeragogy project dashboard (design sketch by Amanda Lyons, used with permission).\label{dashboard}}
\end{figure}

