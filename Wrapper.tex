\begingroup \color{BurntOrange}
\section{Wrapper}\label{sec:Wrapper}

\subsubsection*{Context} An active project with more than a few participants, and possibly with \patternname{Newcomers} arriving frequently.  

\subsubsection*{Problem} In a very active project, it can be effectively impossible to stay up to date with all of the details.  Not everyone will be able to attend every meeting (see \patternname{\href{http://peeragogy.org/patterns/heartbeat/}{Heartbeat}}) or read every email, and project participants can easily get lost and drift away.  The experience can be even worse for \patternname{Newcomers}: joining a project already going can feel like trying to get aboard a rapidly moving vehicle.  If you've taken time off, you may feel like things have moved on so far that they cannot catch up. 

\subsubsection*{Solution}
A project contributor can summarizes what has happened recently in the project, making progress comprehensible to participants who have not been following all of the details.\footnote{In the Peeragogy project, this idea was initially suggested by Charlie Danoff, adapting an idea from his Indiana University class on EFL teaching led by Faridah Pawan. The idea was that someone take on the ``wrapper role'' -- do a weekly pre/post wrap, so that new (and existing) users could get a feel for the status of the project at any given point in time.}  If they are kept up to date, a project's \href{http://socialmediaclassroom.com/host/peeragogy/}{landing page} and \patternname{Roadmap} also serve as a sort of ``wrapper'', telling people what resources they can expect to find in the project and how they can participate.  

\subsubsection*{Rationale}
The wrapper must check the public summaries of the project from time to time to make sure that they accurately represent the facts on the ground.\footnote{In the first year of the Peeragogy project, the ``Weekly Roundup'' by Christopher Tillman Neal served to engage and re-engage members. Peeragogues began to eager watched for the weekly reports to see if our teams or our names had been mentioned. When there was a holiday or break, Chris would announce the hiatus, to keep the flow going. In the second year of the project, we did not routinely publish summaries of progress, and instead, we assumed that interested parties will stay tuned on Google+.  More recently, Charlie has begun publishing irregular wrap-up \href{http://peeragogy.org/peeragogy-wrapper-post-9-feb-5-apr-2015/}{blog posts} and e-mails again, which helps keep people who don't read Google+ up to date.}

\subsubsection*{Resolution} 
According to the theory proposed by Yochai Benkler, for free/open ``commons-based'' projects to work, it is vital to have both (1) the ability to contribute small pieces; (2) something that stitches those pieces together \cite{coases-penguin}. The wrapper helps perform this integrative stitching function, which is often much more challenging than the job of breaking things down into pieces or just doing one of the small pieces.

\subsubsection*{What's Next}
We need better practices for automating the wrapping-up process. One of the latest ideas is to develop a simple visual ``dashboard'' for the project that would be a web page people could access and immediately get an idea of what work is ongoing in the project with links for going more in depth and/or contributing.
\endgroup
    