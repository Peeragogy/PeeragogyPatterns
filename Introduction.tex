\begin{itemize}
\item[{\bf Week of June 8}] Peeragogy Project and Roadmap, comments \#17--\#27.  JC: I've made an initial revision, changing ``Peeragogy Project'' to ``Peeragogy''.  See text in green below.  Roadmap is orange, meaning ``on deck'' but no major changes have been made there yet.
\item[{\bf Week of June 15}] Use or Make and Carrying Capacity, comments up to \#36.  JC: I've made initial revisions, circulated by email.  Lisa has given some thoughtful comments there, some of that has been merged in here but not all. ``Use or Make'' now retitled as ``Reduce, Reuse, Recycle''.
\item[{\bf Week of June 22}] A Specific Project and Wrapper
\item[{\bf Week of June 29}] Heartbeat and Creating a Guide
\item[{\bf Week of July 6}] Newcomer and Pattern Audit Routine
\item[{\bf Week of July 13}] Scrapbook and Distributed Roadmap
\item[{\bf Week of July 20}] Make sure to submit the revised paper
\end{itemize}

\section{Introduction}\label{sec:Introduction}

Readers will have encountered \emph{peer production}, at least in applications like Wikipedia, StackExchange, and free/libre/open source software development.   In the Peeragogy project,  we aim to build upon these inspiring examples to help design the future of education.  
We have found design patterns tremendously useful for organizing our thinking about these matters.  However, there is a key difference between our pattern catalog and previous collections of design patterns that touch on similar domains -- like \emph{Liberating Voices: A Pattern Language for Communication Revolution} \cite{schuler2008liberating} and \emph{Pedagogical Patterns: Advice for Educators} \cite{bergin2012pedagogical}.  Our pattern catalog is our primary project management tool -- our way of ``synthesizing form'' \cite{alexander1964notes} in the Peeragogy project. A quite convincing implementation of Christopher Alexander’s idea of patterns as a ``living language'' \cite[p.~xvii]{alexander1977pattern} was realized with one of the earliest applications of wiki software developed by Ward Cunningham: the Portland Pattern Repository.  What we've developed is a further iteration of this idea.   As the patterns developed, the pattern template we used was revised and revised again, until ultimately we settled on something entirely traditional.  At the level of the pattern template, our one innovation is to add a ``What's next'' step to each pattern, which anticipates the way the pattern will continue to ``resolve'' in the context of our project. 

At a more philosophical level, our approach is all about human interaction, and the challenges, fluidity and lack of predictability that comes with it.  Something that works for one person may not work for another or may not even work for the same person in a slightly different situation.  It is easy to say ``just do X'' and somewhat easy for reasonable people to agree in general terms about what to do.  In our view, other pattern languages often achieve this sort of common sense rationality, and then stop.  In our experience, failure in the prescriptive model begins when people start trying to define things more carefully and make context-specific changes -- when they actually try to put ideas into practice, or understand things in a more tangible way. The patterns we introduce here describe ways to negotiate the execution and implementation of solutions in their practical context.  This often requires compromise, adjustments and even restarts.  

So, while we speak the same language as other developers of design patterns, our orientation is somewhat different, and our understanding of the word `pattern' is nuanced because we aim to take full account of the lifecycle of patterns.  We believe this gets to the heart of what design patterns can do.  We are not just thinking of \emph{abstract} patterns, but patterns that are contingent on our actions, and patterns that -- as we distinguish them -- shape who we are.   We believe that design patterns can be used to ``constitute and occupy practical or speculative problems as such'' and that this is precisely what defines learning \cite[p. 204]{deleuze1994difference}.   Something of along the lines of this dynamical view
is developed by Christian Kohls \cite{kohls2010structure,kohls2011structure}, but in a less ``embedded'' manner.
%
The result of this effort is a hands-on adjunct to existing sociological and historical research on peer production, surveyed in \cite{benkler2015peer}.  
% In practical terms, we believe the patterns that we introduce here will be useful for students and educators who want their work to have real-world relevance, to activists and policy-makers who want to develop practicable solutions to large-scale problems, and to employees and managers who, like it or not, find themselves working in distributed teams. 

The essence of peeragogy is to combine an accessible approach with reflective practice.   We seek to develop a richer language describing effective practices for collaborative interaction.  This work is relevant to giving the often vague idea of ``openness'' a more concrete meaning, and will have applications within and beyond traditional peer production settings.  Our approach to emergent organization is likely to be of interest to theorists in fields like organization studies and (perhaps surprisingly) computer science, where researchers are increasingly making use of social approaches to software design and agent-based models of learning \cite{minsky1967programming,poetry-workshop}.  The next section introduces \patternname{Peeragogy} more explicitly in the form of a design pattern.  Sections \ref{sec:Roadmap}--\ref{sec:Scrapbook} list the main patterns in our catalog.    Figure \ref{fig:connections} illustrates their interconnections.  Section \ref{sec:Distributed_Roadmap} collects our ``What's next'' steps and summarizes the outlook of our project.

\begin{figure}
\vspace{-.9in}
{\centering
\begin{tikzpicture}[dot/.style={circle,inner sep=1pt,fill,name=#1}]
%\draw[step=1cm,gray,very thin] (0,0) grid (10,10);
\node (assess) at (5, 9.75) {{\Large {\sc Assess}}};
\node (organize) at (5, 0) {{\Large {\sc Organize}}};
\node (cooperate)[text width=2cm,align=center,rotate=270] at (10, 5) {{\Large {\sc Convene}}};
\node (convene)[text width=15cm,align=center,rotate=90] at (0, 5) {{\Large {\sc Cooperate}}};

\node(legend)[draw,rectangle,text width=2.67cm] at (9.25,.75) 
{\begin{tabular}{p{2.7cm}@{\hspace{-1mm}}}
\textbf{Legend}\\ \hline\vspace{-2mm} \textbf{A}\hspace{.41in}\textbf{B}\\
if pattern \textbf{A} refers to pattern \textbf{B}.
  \end{tabular}};
\draw[-{Latex[width=2mm]},draw=gray] ([xshift=5mm,yshift=1.75mm]legend.west) -- ([xshift=-14.8mm,yshift=1.75mm]legend.east);

%%%%%%%%%%%%%%%%%%%%%%%%%%%%%%%%%%%%%%%%%%%%%%%%%%%%%%%%%%%%%%%%%%%%%%%%%%%%%%%%%%%%%%%%%%%%%%%%%%%%%
\node[below = 5cm of assess] (roadmap) {\ref{sec:Roadmap}. \hyperref[sec:Roadmap]{\emph{Roadmap}}};
\node (reduce) at (5, 8.75) {\ref{sec:Reduce, reuse, recycle}. \hyperref[sec:Reduce, reuse, recycle]{\emph{Reduce, reuse, recycle}}};
\node (carryingcapacity) at (1.25, 7.15) {\ref{sec:Carrying capacity}. \hyperref[sec:Carrying capacity]{\emph{Carrying capacity}}};
\node[below = 3.2cm of carryingcapacity] (heartbeat) {\ref{sec:Heartbeat}. \hyperref[sec:Heartbeat]{\emph{Heartbeat}}};
\node (aspecificproject) at (8.5, 6.5) {\ref{sec:A specific project}. \hyperref[sec:A specific project]{\emph{A specific project}}};
\node[below = 1cm of roadmap] (wrapper) {\ref{sec:Wrapper}. \hyperref[sec:Wrapper]{\emph{Wrapper}}};
\node (newcomer) at (8.5, 3.25) {\ref{sec:Newcomer}. \hyperref[sec:Newcomer]{\emph{Newcomer}}};
\node[below = 1.7cm of wrapper] (scrapbook) {\ref{sec:Scrapbook}. \hyperref[sec:Scrapbook]{\emph{Scrapbook}}};
\node[above = 1cm of aspecificproject] (peeragogyproject) {\ref{sec:Peeragogy}. \hyperref[sec:Peeragogy]{\emph{Peeragogy}}};
%%%%%%%%%%%%%%%%%%%%%%%%%%%%%%%%%%%%%%%%%%%%%%%%%%%%%%%%%%%%%%%%%%%%%%%%%%%%%%%%%%%%%%%%%%%%%%%%%%%%%
\draw[-{Latex[width=2mm]},draw=gray] (peeragogyproject) -- (aspecificproject);
% \draw[-{Latex[width=2mm]},draw=gray] (aspecificproject) -- (par);
\draw[-{Latex[width=2mm]},draw=gray] (aspecificproject) -- (roadmap);
\draw[-{Latex[width=2mm]},draw=gray] (aspecificproject.230) to[out=250,in=40] (scrapbook);
\draw[-{Latex[width=2mm]},draw=gray] (aspecificproject) -- (carryingcapacity);
\draw[-{Latex[width=2mm]},draw=gray] (carryingcapacity.337) -- (newcomer);
\draw[-{Latex[width=2mm]},draw=gray] (carryingcapacity.330) -- (roadmap);
\draw[-{Latex[width=2mm]},draw=gray] (carryingcapacity) -- (peeragogyproject);
\draw[-{Latex[width=2mm]},draw=gray] ([xshift=1mm]carryingcapacity.south) -- (scrapbook.140);
% \draw[-{Latex[width=2mm]},draw=gray] ([xshift=2mm]creatingaguide.160) to[out=-215,in=-67] (carryingcapacity);
\draw[-{Latex[width=2mm]},draw=gray] (heartbeat) -- (aspecificproject.185);
\draw[-{Latex[width=2mm]},draw=gray] (heartbeat) -- (carryingcapacity);
\draw[-{Latex[width=2mm]},draw=gray] (heartbeat) -- (scrapbook.155);
\draw[-{Latex[width=2mm]},draw=gray] (heartbeat) -- (reduce.215);
\draw[-{Latex[width=2mm]},draw=gray] (newcomer) -- ([xshift=4mm]reduce.south);
\draw[-{Latex[width=2mm]},draw=gray] (newcomer) -- (aspecificproject);
% \draw[-{Latex[width=2mm]},draw=gray] (newcomer) -- (creatingaguide.north);
\draw[-{Latex[width=2mm]},draw=gray] (newcomer) -- (roadmap.350);
\draw[-{Latex[width=2mm]},draw=gray] (newcomer) -- (scrapbook.24);
% \draw[-{Latex[width=2mm]},draw=gray] (par) -- (scrapbook);
\draw[-{Latex[width=2mm]},draw=gray] (roadmap) -- (peeragogyproject.195);
\draw[-{Latex[width=2mm]},draw=gray] (roadmap.350) -- (newcomer);
\draw[-{Latex[width=2mm]},draw=gray] (roadmap) -- (wrapper);
\draw[-{Latex[width=2mm]},draw=gray] ([yshift=.3mm]roadmap.west) -- (heartbeat);
\draw[-{Latex[width=2mm]},draw=gray] (roadmap) -- (aspecificproject);
% \draw[-{Latex[width=2mm]},draw=gray] (scrapbook) -- (par);
\draw[-{Latex[width=2mm]},draw=gray] (scrapbook) -- (wrapper);
\draw[-{Latex[width=2mm]},draw=gray] (scrapbook.110) to[out=123,in=250] (reduce.245);
\draw[-{Latex[width=2mm]},draw=gray] (scrapbook.70) to[out=43,in=305] (roadmap.330);
% \draw[-{Latex[width=2mm]},draw=gray] ([xshift=2mm,yshift=-.4mm]reduce.south) -- (creatingaguide);
\draw[-{Latex[width=2mm]},draw=gray] (reduce) -- (carryingcapacity);
\draw[-{Latex[width=2mm]},draw=gray] (reduce) -- (roadmap);
\draw[-{Latex[width=2mm]},draw=gray] ([xshift=.7mm]wrapper.175) -- (heartbeat);
\draw[-{Latex[width=2mm]},draw=gray] ([xshift=-.7mm,yshift=-.3mm]wrapper.360) -- (newcomer);
\draw[-{Latex[width=2mm]},draw=gray] (wrapper) -- ([xshift=2.3mm]carryingcapacity.south);
\draw[-{Latex[width=2mm]},draw=gray] (wrapper) -- (roadmap);

\end{tikzpicture}


\par
}
\vspace{-.9in}
\caption{Connections between the patterns of peeragogy.  An arrow points from pattern \textbf{A} to pattern \textbf{B} if the description of pattern \textbf{A} references pattern \textbf{B}. Labels at the borders of the figure correspond to the main sections of the \emph{Peeragogy Handbook}.\label{fig:connections}}
\end{figure}


    
    
    
    
    
    
    

    
    
    
    
    
    
    
    
    
    
    
    
    
    
    
    
    
    
    
    