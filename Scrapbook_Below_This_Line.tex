\section{Scrapbook: Below This Line}

\paragraph{Context:} We've been working with the pattern catalog for a while.
\paragraph{Problem:} We've created more patterns than we can keep track of.
\paragraph{Solution:} We'll create a ``Scrapbook'' for patterns that are no longer part of the active catalog, including any for which there are no more ``what's next'' steps.   To honor the old patterns, we will try to summarize what they said and why we no longer need them explicitly.
\paragraph{Rationale:} We want to look at the ways we used to think about things.
\paragraph{Resolution:} For now, any patterns below this line can be considered part of the Scrapbook.  In particular, we're ``retiring'' the anti-patterns, in order to keep things simple.
\paragraph{What's next:} For example, we describe the key ideas from the anti-pattern collection using our \emph{actual} patterns?

\subsection{Exercise: Redescribe the What's Next Summary of the Anti-Patterns}

\begin{quote}
As a way to check whether we can really get away with retiring all of the anti-patterns, are the following steps adequately covered by the patterns ``above the line''?
\end{quote}

\paragraph{Isolation:} We recently submitted an abstract called “Escape from Peeragogy Island”
to a geography conference talking about the spatiality of peer
production. The idea behind this article is that we feel like we’ve come
up with something great with the Peeragogy project, but we’re going to
be a bit isolated if it’s not transparently useful to others. If we
can’t explain why it’s a great idea, then it’s not entirely clear how
great of an idea it actually is.

\paragraph{Magical Thinking:}  Fast-forwarding a few years from the DIY Math experiment: as part of the
PlanetMath project, we are hoping to build a well-thought-through
example of a peer learning space for mathematics. One of the ideas we’re
exploring is to use patterns and antipatterns (exactly like the ones in
this catalog) as a way not only of designing a learning space, but also
of talking about the difficulties that people frequently run into when
studying mathematics. Building an initial collection of Calculus
Patterns may help give people the guide-posts they need to start
effectively self-organizing.

\paragraph{Messy With Lurkers:} What comes out of thinking about the anti-pattern is that we need to be
careful about how we think about “virtues” in a peer production setting.
It is not just a question of being a “good contributor” to an existing
project, but of continually improving the methods that this project uses
to make meaning.

\paragraph{Misunderstanding Power:} As Paul Graham wrote about programming languages – programmers are
typically “satisfied with whatever language they happen to use, because
it dictates the way they think about programs” – so too are people often
“satisfied” with their social environments, because these tend to
dictate the way they think and act in life. Nevertheless, if we put our
minds to it, we can become more “literate” in the patterns that make up
our world and the ways we can effect change.

\paragraph{Moderation:} We recently ran a Paragogical Action Review to elicit feedback from
participants in the Peeragogy project. Some of them brought up
dissatisfactions, and some of them brought up confusion. Can we find
ways to bring these concerns front-and-center, without embarrassing the
people who brought them up?

\paragraph{Navel Gazing:} We have hinted that, in this project, effective criticism is very
welcome! But understanding what makes criticism effective is, in
general, still a research problem.

\paragraph{Stasis:} We’re working on a new handbook chapter about the relationship of open
source software and peeragogy. This will include some more specific
ideas about ways of making change.

\paragraph{Stuck:} If we are actively engaging with other people, then this is a foundation
for strong ties. In this case of deep learning, our aims are neither
instrumental nor informational, but “interactional”. Incidentally, the
“One of us” quoted above has been one of the most consistently engaged
peeragogues over the years of the project. Showing up is a good step –
you can always help someone else move their washing machine!